%! TEX program = xelatex
\PassOptionsToPackage{dvipsnames}{xcolor}
\documentclass{lebhart}

\UseLanguage{Chinese}

%%================================
%% TeX logo 与 网址
%%================================
\usepackage{hologo}
\usepackage{url}

%%================================
%% 排版代码
%%================================
\usepackage{listings}
\definecolor{lightergray}{gray}{0.99}
\lstset{language=[LaTeX]TeX,
    keywordstyle=\color{RoyalBlue},
    basicstyle=\ttfamily,
    commentstyle=\color{ForestGreen}\ttfamily,
    stringstyle=\rmfamily,
    showstringspaces=false,
    breaklines=true,
    frame=lines,
    backgroundcolor=\color{lightergray},
    flexiblecolumns=true,
    escapeinside={(*}{*)},
    % numbers=left,
    numberstyle=\scriptsize, stepnumber=1, numbersep=5pt,
    firstnumber=last,
} 
\providecommand{\meta}[1]{$\langle${\normalfont\itshape#1}$\rangle$}
\lstset{morekeywords=%
    {CreateTheorem,proofideanameCN,cref,dnf,needgraph,UseLanguage,
    linenumbers,nolinenumbers,subsection,maketitle
    }
}
\lstnewenvironment{code}% 
{\setkeys{lst}{columns=fullflexible,keepspaces=true}}{}

\providecommand{\colorist}{\textsf{colorist}}
\providecommand{\lebhart}{\textsf{lebh\-art}}
\providecommand{\lebhartfast}{\textsf{lebh\-art\-fast}}
\providecommand{\beaulivre}{\textsf{beau\-livre}}
\providecommand{\beaulivrefast}{\textsf{beau\-livre\-fast}}

%%================================
%% 正文
%%================================
\begin{document}

\title{\colorist{},多彩风格的文档类系列\thanks{对应版本. \texttt{\colorist{} 2021/03/12}}}
\author{锦文}
\date{2021年3月,北京}

\maketitle

\begin{abstract}
    \colorist{}文档类系列包含用于排版文章的\lebhart{}、用于排版书的\beaulivre{},以及对应的两个fast版本。我设计这一系列的初衷是为了撰写草稿与笔记,使之多彩而不缭乱。

    这些文档类支持英文、法文、中文三种语言,并且同一篇文档中三种语言可以很好地协调。由于采用了自定义字体,需要采用 \hologo{XeLaTeX} 或 \hologo{LuaLaTeX} 进行编译。
    
    最后,这篇说明文档是用\lebhart{}排版的,你可以把它看作一份简短的说明与演示。
\end{abstract}

\begin{tcolorbox}[enhanced jigsaw,pad at break*=1mm,breakable,colback=yellow!25!paper,boxrule=0pt,frame hidden]
    由于\colorist{}主体是从\textsf{minimalist}系列修改而来的,因而一些页面元素还未完全重新设计,特别是目录、part和chapter的样式。这些内容会在将来逐渐加入。
\end{tcolorbox}

\tableofcontents

\section{关于这些文档类的名称与区别}
\lebhart{}取自于德文的lebhaft (活泼),并取了artikel (文章)的前三个字母组合而成。

\beaulivre{}取自于法文的beau (美丽),以及livre (书),由二者组合而成。

\lebhartfast{}与\beaulivrefast{}是更快速但略微粗糙的\lebhart{}与\beaulivre{}。主要区别是:
\begin{itemize}
    \item 使用较为简单的数学字体设置;
    \item 不使用hyperref;
    \item 所有tcolorbox使用draft模式;
    \item 使用polyglossia而不是babel来支持多语言。(使用polyglossia编译速度会略有提高,但目前对于中文的兼容不太完善,在它更加稳定后,将会考虑全面切换到polyglossia)
\end{itemize}

在文章的撰写阶段,建议使用fast版本以加快编译速度,改善写作时的流畅程度。在最后,可以把fast标记去除,从而得到正式的版本。

\section{一些使用说明}

\subsection{定理,以及引用}

定义、定理等环境已经被预定义,可以直接使用,例如:

\begin{code}
  \begin{definition}[奇异物品] \label{def: strange} ...
\end{code}

将会生成
\begin{definition}[奇异物品]\label{def: strange}
    这是奇异物品的定义。
\end{definition}

定理类环境的前后有一行左右的间距。在定义结束的时候会有一个符号来标记。

引用时,可以直接使用智能引用 \lstinline|\cref{标签名称}|,例如:\lstinline|\cref{def: strange}| 会显示为:\cref{def: strange}。

\bigskip
下面是定理类环境的其他几种样式:

\begin{theorem}
    Theorem style: theorem, proposition, lemma, corollary
\end{theorem}

\begin{proof}
    Proof style
\end{proof}

\begin{remark}
    Remark style
\end{remark}

\begin{conjecture}
    Conjecture style
\end{conjecture}

\begin{example*}
    Example style: example, fact
\end{example*}

\begin{problem}
    Problem style
\end{problem}

\subsection{定义新的定理型环境}

首先定义这个环境在所用语言下的名称 \lstinline|\(name of environment)(language name)|,其中 \\\lstinline|(language name)| 是 \lstinline|EN|、\lstinline|FR|、\lstinline|CN| 等,然后用下面四种方式之一定义这一环境:
\begin{itemize}
    \item \lstinline|\CreateTheorem*{(name of environment)}|
    \item \lstinline|\CreateTheorem{(name of environment)}[(numbered like)]|
    \item \lstinline|\CreateTheorem{(name of environment)}<(numbered within)>|
    \item \lstinline|\CreateTheorem{(name of environment)}|
\end{itemize}

\def\proofideanameCN{思路}
\CreateTheorem*{proofidea}

例如,

\begin{code}
  \def\proofideanameCN{思路}
  \CreateTheorem*{proofidea}
\end{code}
可以定义不编号的环境 \lstinline|proofidea|,它支持在中文环境中使用,效果如下所示:

\begin{proofidea}
    ...
\end{proofidea}

\subsection{未完成标记}

你可以通过 \lstinline|\dnf| 来标记尚未完成的部分。例如:
\begin{itemize}
    \item \lstinline|\dnf|: \quad \dnf
    \item \lstinline|\dnf<还需加入…>|: \quad \dnf<还需加入…>
\end{itemize}

类似的,还有 \lstinline|\needgraph| :
\begin{itemize}
    \item \lstinline|\needgraph|: \needgraph
    \item \lstinline|\needgraph<关于…>|: \needgraph<关于…>
\end{itemize}

\subsection{语言设置}
可以随时使用 \lstinline|\UseLanguage{语言名称}| 更改语言,语言名称包括Chinese、English、French(首字母大小写随意,例如chinese亦可)。这样,各种指令和环境的效果也会随之变动。

例如,使用 \lstinline|\UseLanguage{English}| 后,定理与未完成标记会显示为:

\UseLanguage{English}
\begin{theorem}[Useless]\label{thm}
    Some theorem in English. \dnf
\end{theorem}

引用时,定理的名称总是与定理所在区域的语言匹配,例如,开头的定义在现在的英文模式下依然显示为中文:\cref{def: strange}和\cref{thm}。

% \subsection{关于行号}
% 行号可以随时开启和关闭。\lstinline|\linenumbers| 用来开启行号,\lstinline|\nolinenumbers| 用来关闭行号。标题、目录、索引等位置为了美观,不进行编号。

\subsection{关于字体}
\lebhart{}与\beaulivre{}使用Palatino Linotype作为英文字体,方正悠宋、悠黑简体作为中文字体,并部分使用了Neo Euler作为数学字体:
\begin{itemize}
    \item English main font. \textsf{English sans serif font}.
    \item 中文主要字体,\textsf{中文无衬线字体}
    \item 数学示例: \( \alpha, \beta, \gamma, \delta, 1,2,3,4, a,b,c,d \), \[\mathrm{li}(x)\coloneqq \int_2^{\infty} \frac{1}{\log t}\,\mathrm{d}t \]
\end{itemize}

其中,Neo Euler可以在 \url{https://github.com/khaledhosny/euler-otf} 下载。

其他字体不是免费字体,需要自行购买使用(你可以在方正字库网站查询详细资料:\url{https://www.foundertype.com})。

在没有安装相应的字体时,将采用TeX Live中自带的字体来代替,效果可能会有所折扣。

\section{文档示例}

\begin{minipage}{0.4\textwidth}
\begin{code}
%! TEX program = xelatex
\documentclass{lebhartfast}

\UseLanguage{French}

\begin{document}

\title{Titre}
\author{Nom}
\date{03 / 2021, Lieu}

\maketitle

%% Texte ici

\end{document}
\end{code}
\end{minipage}
%
\hfill
%
\begin{minipage}{0.4\textwidth}
\begin{code}
%! TEX program = xelatex
\documentclass{lebhartfast}

\UseLanguage{Chinese}

\begin{document}

\title{标题}
\author{姓名}
\date{2021年3月,地点}

\maketitle

%% 正文部分

\end{document}
\end{code}
\end{minipage}
\par

\bigskip
(\lstinline|\UseLanguage| 既可以放在导言中,也可以放在正文部分,并且可以按照需要反复使用)

\end{document}
