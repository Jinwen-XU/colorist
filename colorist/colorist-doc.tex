%! TEX program = xelatex
\documentclass[allowbf]{colorart}

%%================================
%% Import toolkit
%%================================
\usepackage{PJLtoolkit}
\usepackage{longtable}  % breakable tables
\usepackage{hologo}     % more TeX logo

%%================================
%% For typestting code
%%================================
\usepackage{listings}
\definecolor{maintheme}{RGB}{70,130,180}
\definecolor{forestgreen}{RGB}{21,122,81}
\definecolor{lightergray}{gray}{0.99}
\lstset{language=[LaTeX]TeX,
    keywordstyle=\color{maintheme},
    basicstyle=\ttfamily,
    commentstyle=\color{forestgreen}\ttfamily,
    stringstyle=\rmfamily,
    showstringspaces=false,
    breaklines=true,
    frame=lines,
    backgroundcolor=\color{lightergray},
    flexiblecolumns=true,
    escapeinside={(*}{*)},
    % numbers=left,
    numberstyle=\scriptsize, stepnumber=1, numbersep=5pt,
    firstnumber=last,
} 
\providecommand{\meta}[1]{$\langle${\normalfont\itshape#1}$\rangle$}
\lstset{moretexcs=%
    {linenumbers,nolinenumbers,subsection,
    color,NameTheorem,CreateTheorem,proofideanameEN,cref,dnf,needgraph,UseLanguage,UseOtherLanguage,AddLanguageSetting,maketitle,address,curraddr,email,keywords,subjclass,thanks,dedicatory,PJLdate,ProjLib
    }
}
\lstnewenvironment{code}% 
{\setkeys{lst}{columns=fullflexible,keepspaces=true}}{}

%%================================
%% remind
%%================================
\usepackage[many]{tcolorbox}
\newenvironment{remind}[1][Remind]{%
    \begin{tcolorbox}[breakable,
        enhanced,
        width = \textwidth,
        colback = white, colbacktitle = paper,
        colframe = gray!50, boxrule=0.2mm,
        coltitle = black,
        fonttitle = \sffamily,
        attach boxed title to top left = {yshift=-\tcboxedtitleheight/2,  xshift=\tcboxedtitlewidth/4},
        boxed title style = {boxrule=0pt, colframe=paper},
        before skip = 0.3cm,
        after skip = 0.3cm,
        top = 3mm,
        bottom = 3mm,
        title={\sffamily #1}]%
}{\end{tcolorbox}}

%%================================
%% Names
%%================================
\providecommand{\colorist}{{\normalfont\textsf{colorist}}}
\providecommand{\colorart}{{\normalfont\textsf{colorart}}}
\providecommand{\colorbook}{{\normalfont\textsf{colorbook}}}
\providecommand{\lebhart}{{\normalfont\textsf{lebhart}}}
\providecommand{\beaulivre}{{\normalfont\textsf{beaulivre}}}


%%================================
%% Main text
%%================================
\begin{document}

\title{{\normalfont\textbf{\textsf{colorist}}}, write your articles or books in a colorful way}
\author{Jinwen XU}
\thanks{Corresponding to: \texttt{\colorist{} 2021/05/24}}
\date{May 2021, Beijing}

\maketitle

\begin{abstract}
    \colorist{} is a series of styles and classes for you to typeset your articles or books in a colorful manner. My original intention in designing this series is to write drafts and notes that look colorful yet not dazzling.

    The entire collection includes \verb|colorist.sty|, which is the main style shared by all of the following classes; \verb|colorart.cls| for typesetting articles and \verb|colorbook.cls| for typesetting books. They compile with any major \TeX{} engine, with native support to English, French and German typesetting via \lstinline|\UseLanguage| (see the instruction below for detail).

    You can also found \lebhart{} and \beaulivre{} on CTAN. They are the enhanced version of \colorart{} and \colorbook{} with unicode support. With this, they can access to more beautiful fonts, and also have native support for Chinese, Japanese and Russian typesetting. On the other hand, they need to be compiled with \hologo{XeLaTeX} or \hologo{LuaLaTeX} (not pdf\LaTeX).
    
    This documentation is typeset using \colorart{} (with the option \texttt{allowbf}). You can think of it as a short introduction and demonstration.
\end{abstract}

\tableofcontents

\bigskip\bigskip
\begin{remind}
    Multi-language support, theorem-like environments, draft marks and some other features are provided by the \ProjLib{} toolkit. Here we only briefly discuss how to use it with this document class. For more detailed information, you can refer to the documentation of \ProjLib{}.
\end{remind}

\clearpage
\section{Initialization}

\subsection{How to load it}
You can directly use \colorart{} or \colorbook{} as your document class. In this way, you can directly begin writing your document, without having to worry about the configurations.

\begin{code}
  \documentclass{colorart} (*{\normalfont or}*) \documentclass{colorbook}
\end{code}

And of course, you can also use the default classes \textsf{article} or \textsf{book}, and load the \colorist{} package. This way, only the basic styles are set, and you can thus use your preferred fonts and page layout. All the features mentioned in this article are provided.
\begin{code}
  \usepackage{colorist}
\end{code}

\subsection{Options}

\colorist{} offers the following options: 

\begin{itemize}
    \item \texttt{draft} or \texttt{fast}
        \begin{itemize}
            \item The option \verb|fast| enables a faster but slightly rougher style, main differences are:
            \begin{itemize}
                \item Use simpler math font configuration; 
                \item Do not use hyperref; 
                \item Enable the fast mode of \ProjLib{} toolkit.
            \end{itemize}
        \end{itemize}
    \item \texttt{allowbf}
        \begin{itemize}
            \item Allow boldface. When this option is enabled, the title, titles of all levels and names of theorem-like environments will be bolded.
        \end{itemize}
\end{itemize}

\begin{remind}
    \begin{itemize}
        \item During the draft stage, it is recommended to use the \verb|fast| option to speed up compilation. At the end, one should remove the ``fast'' option to get the final version. When in \verb|fast| mode, there will be a watermark "DRAFT" to indicate that you are currently in the draft mode.
    \end{itemize}
\end{remind}

\bigskip
Additionally, \colorart{} and \colorbook{} offers the following options: 
\begin{itemize}
    \item \texttt{a4paper} or \texttt{b5paper}
        \begin{itemize}
            \item Optional paper size. The default paper size is 8.5in $\times$ 11in.
        \end{itemize}
\end{itemize}


\section{Some instructions}

Many of the features described next are provided by the \ProjLib{} toolkit. Only the basic usage is mentioned here. For more details, please refer to its user documentation.

\subsection{Language configuration}

\colorist{} has multi-language support, including simplified Chinese, traditional Chinese, English, French, German, Japanese, and Russian. The language can be selected by the following macros:

\begin{itemize}
    \item \lstinline|\UseLanguage{|\meta{language name}\lstinline|}| is used to specify the language. The corresponding setting of the language will be applied after it. It can be used either in the preamble or in the main body. When no language is specified, ``English" is selected by default.
    \item \lstinline|\UseOtherLanguage{|\meta{language name}\lstinline|}{|\meta{content}\lstinline|}|, which uses the specified language settings to typeset \meta{content}. Compared with \lstinline|\UseLanguage|, it will not modify the line spacing, so line spacing would remain stable when CJK and Western texts are mixed.
\end{itemize}

\clearpage
\meta{language name} can be (it is not case sensitive, for example, \texttt{French} and \texttt{french} have the same effect):
\begin{itemize}
    \item Simplified Chinese: \texttt{CN}, \texttt{Chinese}, \texttt{SChinese} or \texttt{SimplifiedChinese}
    \item Traditional Chinese: \texttt{TC}, \texttt{TChinese} or \texttt{TraditionalChinese}
    \item English: \texttt{EN} or \texttt{English}
    \item French: \texttt{FR} or \texttt{French}
    \item German: \texttt{DE}, \texttt{German} or \texttt{ngerman}
    \item Japanese: \texttt{JP} or \texttt{Japanese}
    \item Russian: \texttt{RU} or \texttt{Russian}
\end{itemize}

\medskip
In addition, you can also add new settings to selected language:
\begin{itemize}
    \item \lstinline|\AddLanguageSetting{|\meta{settings}\lstinline|}|
    \begin{itemize}
        \item Add \meta{settings} to all supported languages.
    \end{itemize}
    \item \lstinline|\AddLanguageSetting(|\meta{language name}\lstinline|){|\meta{settings}\lstinline|}|
    \begin{itemize}
        \item Add \meta{settings} to the selected language \meta{language name}.
    \end{itemize}
\end{itemize}
For example, \lstinline|\AddLanguageSetting(German){\color{orange}}| can make all German text displayed in orange (of course, one then need to add \lstinline|\AddLanguageSetting{\color{black}}| in order to correct the color of the text in other languages).


\subsection{Theorems and how to reference them}

Environments such as definitions and theorems have been pre-defined and can be used directly. 

More specifically, preset environments include: 
\texttt{assumption}, \texttt{axiom}, \texttt{conjecture}, \texttt{convention}, \texttt{corollary}, \texttt{definition}, \texttt{definition-proposition}, \texttt{definition-theorem}, \texttt{example}, \texttt{exercise}, \texttt{fact}, \texttt{hypothesis}, \texttt{lemma}, \texttt{notation}, \texttt{problem}, \texttt{property}, \texttt{proposition}, \texttt{question}, \texttt{remark}, \texttt{theorem}, and the corresponding unnumbered version with an asterisk \lstinline|*| in the name. The display of these environments will change according to the current language.

\medskip
When referencing a theorem-like environment, it is recommended to use clever reference \lstinline|\cref{|\meta{label}\texttt{\}}. In this way, there is no need to explicitly write down the name of the corresponding environment every time.

\begin{remind}[Example]
\begin{code}
  \begin{definition}[Strange things] \label{def: strange} ...
\end{code}

will produce
\begin{definition}[Strange things]\label{def: strange}
    This is the definition of some strange objects.
\end{definition}

\lstinline|\cref{def: strange}| will be displayed as: \cref{def: strange}.

\medskip
After using \lstinline|\UseLanguage{French}|, a theorem will be displayed as:

\UseLanguage{French}
\begin{theorem}[Inutile]\label{thm}
    Un théorème en français.
\end{theorem}

By default, when referenced, the name of the theorem always matches the language of the context in which the theorem is located. For example, the definition above is still displayed in English in the current French mode: \cref{def: strange} and \cref{thm}. If you want the name of the theorem to match the current context when referencing, you can add \texttt{regionalref} to the global options.
\end{remind}


\clearpage %\bigskip
The following are the main styles of theorem-like environments:
\begin{theorem}
    Theorem style: theorem, proposition, lemma, corollary, ...
\end{theorem}

\begin{proof}
    Proof style
\end{proof}

\begin{remark}
    Remark style
\end{remark}

\begin{conjecture}
    Conjecture style
\end{conjecture}

\begin{example*}
    Example style: example, fact, ...
\end{example*}

\begin{problem}
    Problem style: problem, question, ...
\end{problem}

\medskip
For aesthetics, adjacent definitions will be connected together automatically:
\begin{definition}
    First definition.
\end{definition}

\begin{definition}
    Second definition.
\end{definition}

\UseLanguage{English}

% \clearpage
\subsection{Define a new theorem-like environment}

If you need to define a new theorem-like environment, you must first define the name of the environment in the language to use:
\begin{itemize}
    \item \lstinline|\NameTheorem[|\meta{language name}\lstinline|]{|\meta{name of environment}\lstinline|}{|\meta{name string}\lstinline|}|
\end{itemize}
For \meta{language name}, please refer to the section on language configuration. When \meta{language name} is not specified, the name will be set for all supported languages. In addition, environments with or without asterisk share the same name, therefore, \lstinline|\NameTheorem{envname*}{...}| has the same effect as \lstinline|\NameTheorem{envname}{...}| .

\medskip
And then define this environment in one of following five ways:
\begin{itemize}
    \item \lstinline|\CreateTheorem*{|\meta{name of environment}\lstinline|}|
        \begin{itemize}
            \item Define an unnumbered environment \meta{name of environment}
        \end{itemize}
    \item \lstinline|\CreateTheorem{|\meta{name of environment}\lstinline|}|
        \begin{itemize}
            \item Define a numbered environment \meta{name of environment}, numbered in order 1,2,3,\dots
        \end{itemize}
    \item \lstinline|\CreateTheorem{|\meta{name of environment}\lstinline|}[|\meta{numbered like}\lstinline|]|
        \begin{itemize}
            \item Define a numbered environment \meta{name of environment}, which shares the counter \meta{numbered like}
        \end{itemize}
    \item \lstinline|\CreateTheorem{|\meta{name of environment}\lstinline|}<|\meta{numbered within}\lstinline|>|
        \begin{itemize}
            \item Define a numbered environment \meta{name of environment}, numbered within the \\counter \meta{numbered within}
        \end{itemize}
    \clearpage
    \item \lstinline|\CreateTheorem{|\meta{name of environment}\lstinline|}(|\meta{existed environment}\lstinline|)|\\
    \lstinline|\CreateTheorem*{|\meta{name of environment}\lstinline|}(|\meta{existed environment}\lstinline|)|
        \begin{itemize}
            \item Identify \meta{name of environment} with \meta{existed environment} or \meta{existed environment}\lstinline|*|.
            \item This method is usually useful in the following two situations:
                \begin{enumerate}
                    \item To use a more concise name. For example, with \lstinline|\CreateTheorem{thm}|\\\lstinline|(theorem)|, one can then use the name \texttt{thm} to write theorem.
                    \item To remove the numbering. For example, one can remove the numbering of the \texttt{remark} environment with \lstinline|\CreateTheorem{remark}(remark*)|.
                \end{enumerate}
        \end{itemize}
\end{itemize}

\medskip
\begin{remind}
    It uses \textsf{amsthm} internally, so the traditional \texttt{theoremstyle} is also applicable to it. One only needs declare the style before the relevant definitions.
\end{remind}

% \def\proofideanameEN{Idea}
\NameTheorem[EN]{proofidea}{Idea}
\CreateTheorem*{proofidea*}
\CreateTheorem{proofidea}<subsection>

\bigskip
Here is an example. The following code:
\begin{code}
  \NameTheorem[EN]{proofidea}{Idea}
  \CreateTheorem*{proofidea*}
  \CreateTheorem{proofidea}<subsection>
\end{code}
defines an unnumbered environment \lstinline|proofidea*| and a numbered environment \lstinline|proofidea| (numbered within subsection) respectively. They can be used in English context. The effect is as follows:

\begin{proofidea*}
    The \lstinline|proofidea*| environment.
\end{proofidea*}

\begin{proofidea}
    The \lstinline|proofidea| environment.
\end{proofidea}

\bigskip
\subsection{Draft mark}

You can use \lstinline|\dnf| to mark the unfinished part. For example:
\begin{itemize}
    \item \lstinline|\dnf| or \lstinline|\dnf<...>|. The effect is: \dnf~ or \dnf<...>. \\The prompt text changes according to the current language. For example, it will be displayed as \UseOtherLanguage{French}{\dnf} in French mode.
\end{itemize}

Similarly, there is \lstinline|\needgraph| : 
\begin{itemize}
    \item \lstinline|\needgraph| or \lstinline|\needgraph<...>|. The effect is: \needgraph or \needgraph<...>The prompt text changes according to the current language. For example, in French mode, it will be displayed as \UseOtherLanguage{French}{\needgraph}
\end{itemize}

\clearpage
\subsection{On the title, abstract and keywords}

\colorart{} has both the features of standard classes and that of the \AmS{} classes.

Therefore, the title part can either be written in the usual way, in accordance with the standard class \textsf{article}:

\begin{code}
  \title{(*\meta{title}*)}
  \author{(*\meta{author}*)\thanks{(*\meta{text}*)}}
  \date{(*\meta{date}*)}
  \maketitle
  \begin{abstract}
      (*\meta{abstract}*)
  \end{abstract}
  \begin{keyword}
      (*\meta{keywords}*)
  \end{keyword}
\end{code}

or written in the way of \AmS{} classes:

\begin{code}
  \title{(*\meta{title}*)}
  \author{(*\meta{author}*)}
  \thanks{(*\meta{text}*)}
  \address{(*\meta{address}*)}
  \email{(*\meta{email}*)}
  \date{(*\meta{date}*)}
  \keywords{(*\meta{keywords}*)}
  \subjclass{(*\meta{subjclass}*)}
  \begin{abstract}
      (*\meta{abstract}*)
  \end{abstract}
  \maketitle
\end{code}

\medskip
The author information can contain multiple groups, written as:

\begin{code}
  \author{(*\meta{author 1}*)}
  \address{(*\meta{address 1}*)}
  \email{(*\meta{email 1}*)}
  \author{(*\meta{author 2}*)}
  \address{(*\meta{address 2}*)}
  \email{(*\meta{email 2}*)}
  ...
\end{code}

Among them, the mutual order of \lstinline|\address|, \lstinline|\curraddr|, \lstinline|\email| is not important.


\section{Known issues}

\begin{itemize}[itemsep=.6em]
    \item The font settings are still not perfect.
    \item Since many features are based on the \ProjLib{} toolkit, \colorist{} inherits all its problems. For details, please refer to the "Known Issues" section of the \ProjLib{} documentation.
    \item The error handling mechanism is incomplete: there is no corresponding error prompt when some problems occur.
    \item There are still many things that can be optimized in the code.
\end{itemize}

\clearpage
\section{Document templates}

\subsection{The standard way}

If you want to write in the standard way, you can refer to the following example:

\begin{code}
\documentclass{colorart}
\usepackage{PJLtoolkit} % Load ProjLib toolkit

\UseLanguage{French} % Use French from here

\begin{document}

\title{Le Titre}
\author{Auteur}
\date{\PJLdate{2022-04-01}}

\maketitle

\begin{abstract}
    Ceci est un résumé. \dnf<Plus de contenu est nécessaire.>
\end{abstract}
\begin{keyword}
    AAA, BBB, CCC, DDD, EEE
\end{keyword}

\section{Un théorème}

%% Theorem-like environments can be used directly
\begin{theorem}\label{thm:abc}
    Ceci est un théorème.
\end{theorem}

Référence du théorème: \cref{thm:abc} 
    % It is recommended to use clever reference

\end{document}
\end{code}

If you wish to switch to the standard class later, just replace the first two lines with:

\begin{code}
\documentclass{article}
\usepackage[a4paper,margin=1in]{geometry}
\usepackage[hidelinks]{hyperref}
\usepackage[palatino]{PJLtoolkit} % Load ProjLib toolkit
\end{code}


\clearpage
\subsection{The \texorpdfstring{\AmS{}}{AMS} way}

If you intend to switch to the journal template in the future and thus want to use the writing style as in the \AmS{} classes, you can refer to the following example:

\begin{code}
\documentclass{colorart}
\usepackage{PJLtoolkit} % Load ProjLib toolkit

\UseLanguage{French} % Use French from here

\begin{document}

\title{Le Titre}
\author{Auteur 1}
\address{Adresse 1}
\email{\href{Courriel 1}{Courriel 1}}
\author{Auteur 1}
\address{Adresse 1}
\email{\href{Courriel 2}{Courriel 2}}
\date{\PJLdate{2022-04-01}}
\subjclass{*****}
\keywords{...}

\begin{abstract}
    Ceci est un résumé. \dnf<Plus de contenu est nécessaire.>
\end{abstract}

\maketitle

\section{Première section}

%% Theorem-like environments can be used directly
\begin{theorem}\label{thm:abc}
    Ceci est un théorème.
\end{theorem}

Référence du théorème: \cref{thm:abc} 
    % It is recommended to use clever reference

\end{document}
\end{code}

In this way, if you wish to switch to \AmS{} class later, just replace the first two lines with:

\begin{code}
\documentclass{amsart}
\usepackage[a4paper,margin=1in]{geometry}
\usepackage[hidelinks]{hyperref}
\usepackage[palatino]{PJLtoolkit} % Load ProjLib toolkit
\end{code}

\end{document}
