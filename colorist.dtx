% \iffalse meta-comment
%
% Copyright (C) 2021-2023 by Jinwen XU
% ------------------------------------
%
% This file may be distributed and/or modified under the conditions of the LaTeX
% Project Public License, either version 1.3c of this license or (at your option)
% any later version. The latest version of this license is in:
%
%    http://www.latex-project.org/lppl.txt
%
% \fi
%
%<*driver>
\ProvidesFile{colorist.dtx}
%</driver>
\NeedsTeXFormat{LaTeX2e}[2022-06-01]
%
%<*colorart>
\ProvidesExplClass
  {colorart}
  {2023/07/14} {}
  {A colorful article style}

\tl_const:Nn \l__colorclass_base_class_tl { article }
%</colorart>
%
%<*colorbook>
\ProvidesExplClass
  {colorbook}
  {2023/07/14} {}
  {A colorful book style}

\tl_const:Nn \l__colorclass_base_class_tl { book }
%</colorbook>
%
%<*lebhart>
\ProvidesExplClass
  {lebhart}
  {2023/07/14} {}
  {A colorful article style}

\tl_const:Nn \l__colorclass_base_class_tl { article }
%</lebhart>
%
%<*beaulivre>
\ProvidesExplClass
  {beaulivre}
  {2023/07/14} {}
  {A colorful book style}

\tl_const:Nn \l__colorclass_base_class_tl { book }
%</beaulivre>
%
%<*colorist>
\ProvidesExplPackage
  {colorist}
  {2023/07/14} {}
  {A colorful style for articles and books}
%</colorist>
%
%<*colorist-fancy>
\ProvidesExplPackage
  {colorist-fancy}
  {2023/07/14} {}
  {The fancy style of colorist}
%</colorist-fancy>

%<*class>

\bool_new:N \l__colorclass_load_custom_font_file_bool
\bool_set_false:N \l__colorclass_load_custom_font_file_bool

%<*lebhart|beaulivre>
\bool_new:N \l__colorclass_load_custom_font_file_latin_bool
\bool_set_false:N \l__colorclass_load_custom_font_file_latin_bool

\bool_new:N \l__colorclass_load_custom_font_file_cjk_bool
\bool_set_false:N \l__colorclass_load_custom_font_file_cjk_bool

\bool_new:N \l__colorclass_load_custom_font_file_math_bool
\bool_set_false:N \l__colorclass_load_custom_font_file_math_bool
%</lebhart|beaulivre>

\keys_define:nn { colorclass }
  {
    , draft                   .bool_set:N         = \l__colorclass_fast_bool
    , draft                   .initial:n          = { false }
    , fast                    .bool_set:N         = \l__colorclass_fast_bool

    , print                   .bool_set:N         = \l__colorclass_print_mode_bool
    , print                   .initial:n          = { false }
    , print mode              .bool_set:N         = \l__colorclass_print_mode_bool
    , print~mode              .bool_set:N         = \l__colorclass_print_mode_bool
    , print-mode              .bool_set:N         = \l__colorclass_print_mode_bool
    , print version           .bool_set:N         = \l__colorclass_print_mode_bool
    , print~version           .bool_set:N         = \l__colorclass_print_mode_bool
    , print-version           .bool_set:N         = \l__colorclass_print_mode_bool

%<*none>
    , fancy                   .bool_set:N         = \l__colorclass_fancy_bool
    , fancy                   .initial:n          = { false }
%</none>

    , load custom font file   .code:n             = {
                                                      \bool_set_true:N \l__colorclass_load_custom_font_file_bool
                                                      \str_set:Nn \l__colorclass_custom_font_file_str { #1 }
%<*lebhart|beaulivre>
                                                      \bool_set_true:N \l__colorclass_load_custom_font_file_latin_bool
                                                      \str_set:Nn \l__colorclass_custom_font_file_latin_str { colorist.font.latin }
                                                      \bool_set_true:N \l__colorclass_load_custom_font_file_cjk_bool
                                                      \str_set:Nn \l__colorclass_custom_font_file_cjk_str   { colorist.font.cjk }
                                                      \bool_set_true:N \l__colorclass_load_custom_font_file_math_bool
                                                      \str_set:Nn \l__colorclass_custom_font_file_math_str  { colorist.font.math }
%</lebhart|beaulivre>
                                                    }
    , load custom font file   .default:n          = { colorist.font }
    , load~custom~font~file   .meta:n             = { load custom font file = { #1 } }
    , load~custom~font~file   .default:n          = { colorist.font }
    , load-custom-font-file   .meta:n             = { load custom font file = { #1 } }
    , load-custom-font-file   .default:n          = { colorist.font }

%<*lebhart|beaulivre>
    , load custom latin font file   .code:n       = {
                                                      \bool_set_true:N \l__colorclass_load_custom_font_file_latin_bool
                                                      \str_set:Nn \l__colorclass_custom_font_file_latin_str { #1 }
                                                    }
    , load custom latin font file   .default:n    = { colorist.font.latin }
    , load~custom~latin~font~file   .meta:n       = { load custom latin font file = { #1 } }
    , load~custom~latin~font~file   .default:n    = { colorist.font.latin }
    , load-custom-latin-font-file   .meta:n       = { load custom latin font file = { #1 } }
    , load-custom-latin-font-file   .default:n    = { colorist.font.latin }

    , load custom cjk font file     .code:n       = {
                                                      \bool_set_true:N \l__colorclass_load_custom_font_file_cjk_bool
                                                      \str_set:Nn \l__colorclass_custom_font_file_cjk_str { #1 }
                                                    }
    , load custom cjk font file     .default:n    = { colorist.font.cjk }
    , load~custom~cjk~font~file     .meta:n       = { load custom cjk font file = { #1 } }
    , load~custom~cjk~font~file     .default:n    = { colorist.font.cjk }
    , load-custom-cjk-font-file     .meta:n       = { load custom cjk font file = { #1 } }
    , load-custom-cjk-font-file     .default:n    = { colorist.font.cjk }

    , load custom math font file    .code:n       = {
                                                      \bool_set_true:N \l__colorclass_load_custom_font_file_math_bool
                                                      \str_set:Nn \l__colorclass_custom_font_file_math_str { #1 }
                                                    }
    , load custom math font file    .default:n    = { colorist.font.math }
    , load~custom~math~font~file    .meta:n       = { load custom math font file = { #1 } }
    , load~custom~math~font~file    .default:n    = { colorist.font.math }
    , load-custom-math-font-file    .meta:n       = { load custom math font file = { #1 } }
    , load-custom-math-font-file    .default:n    = { colorist.font.math }
%</lebhart|beaulivre>

    , a4paper                 .bool_set:N         = \l__colorclass_a_four_paper_bool
    , a4paper                 .initial:n          = { false }
    , b5paper                 .bool_set:N         = \l__colorclass_b_five_paper_bool
    , b5paper                 .initial:n          = { false }

    , oneside                 .code:n             = { \PassOptionsToClass { \CurrentOption } { \l__colorclass_base_class_tl } }
    , twoside                 .code:n             = { \PassOptionsToClass { \CurrentOption } { \l__colorclass_base_class_tl } }

    , 11pt                    .code:n             = { \PassOptionsToClass { \CurrentOption } { \l__colorclass_base_class_tl } }
    , 12pt                    .code:n             = { \PassOptionsToClass { \CurrentOption } { \l__colorclass_base_class_tl } }

    , unknown                 .code:n             = {
                                                      \PassOptionsToPackage { \CurrentOption } { colorist }
                                                    }
  }
\ProcessKeyOptions [ colorclass ]

\LoadClass{\l__colorclass_base_class_tl}


\NewDocumentCommand \IfPrintModeTF { m m }
  {
    \bool_if:NTF \l__colorclass_print_mode_bool { #1 } { #2 }
  }
\NewDocumentCommand \IfPrintModeT { m }
  {
    \bool_if:NT \l__colorclass_print_mode_bool { #1 }
  }
\NewDocumentCommand \IfPrintModeF { m }
  {
    \bool_if:NF \l__colorclass_print_mode_bool { #1 }
  }


%%================================
%%  Page layout
%%================================
\RequirePackage { silence }
\WarningFilter { geometry } { Over-specification }

\PassOptionsToPackage { heightrounded } { geometry }
\RequirePackage { geometry }

\geometry
  {
    papersize = { 8.5in, 11in },
    total = { 6.500in, 9.130in },
    centering,
    footnotesep = 2em plus 2pt minus 2pt,
    footskip = .5in,
  }

\bool_if:NT \l__colorclass_b_five_paper_bool
  {
    \geometry
      {
        b5paper,
        total = { 5.535in, 8.160in },
        centering,
        footnotesep = 2em plus 2pt minus 2pt,
        footskip = .5in,
      }
  }

\bool_if:NT \l__colorclass_a_four_paper_bool
  {
    \geometry
      {
        a4paper,
        total = { 6.500in, 9.685in },
        centering,
        footnotesep = 2em plus 2pt minus 2pt,
        footskip = .5in,
      }
  }

\bool_if:NT \l__colorclass_fast_bool
  {
    \PassOptionsToPackage { fast } { colorist }
    \RequirePackage { draftwatermark }
    \DraftwatermarkOptions { text = { \normalfont DRAFT }, color = paper!97!-paper }
  }

\RequirePackage { indentfirst }

\RequirePackage { colorist }

% \raggedbottom
% \emergencystretch=2pt
% \hfuzz=2pt
% \vfuzz=2pt

%%================================
%%  Fonts
%%================================
\WarningFilter { latexfont } { Font~shape }
\WarningFilter { latexfont } { Some~font  }

\hook_gput_code:nnn { begindocument/before } { colorclass }
  {
    \IfPackageLoadedTF { biblatex }
      {
        \PassOptionsToPackage { biblatex } { embrac }
      } {}
    \RequirePackage { embrac }
  }

\cs_new_protected:Nn \__colorclass_load_file_or_config:Nnn
  {
    \bool_if:NT #1
      {
        \exp_args:Nx \file_if_exist:nT { #2 }
          {
            \exp_args:Nx \file_input:n { #2 }
            \use_none:nn
          }
      }
    \use:n { #3 }
  }

%<*lebhart|beaulivre>
\cs_new_protected:Nn \__colorclass_if_font_exist:nnn
  {
    \bool_if:NTF \l__colorclass_fast_bool
      { #3 }
      { \fontspec_font_if_exist:nTF { #1 } { #2 } { #3 } }
  }
%</lebhart|beaulivre>

\__colorclass_load_file_or_config:Nnn \l__colorclass_load_custom_font_file_bool { \l__colorclass_custom_font_file_str }
  {
    \RequirePackage { projlib-font }

%<*colorart|colorbook>
    \bool_if:NF \g_projlib_font_already_set_bool
      {
        \RequirePackage { mathpazo }
        \RequirePackage { newpxtext }
        \bool_if:NT \l__projlib_font_useosf_bool { \useosf }
        \RequirePackage { amssymb }
        \sys_if_engine_pdftex:F
          {
            \setsansfont { texgyreheros }
              [
                Extension      = .otf ,
                Scale          = MatchUppercase ,
                UprightFont    = *-regular ,
                BoldFont       = *-bold ,
                ItalicFont     = *-italic ,
                BoldItalicFont = *-bolditalic ,
              ]
          }
        % Adjusting the kerning of 'embrac' for 'newpxtext'
        \hook_gput_code:nnn { begindocument/before } { colorclass }
          {
            \RenewEmph{[}{]}
            \RenewEmph{(}{)}
          }
      }
%</colorart|colorbook>
%
%<*lebhart|beaulivre>
    \bool_if:NF \g_projlib_font_already_set_bool
      {
        \bool_if:NT \l__colorclass_fast_bool
          {
            \RequirePackage { mathpazo }
          }

        \PassOptionsToPackage { no-math,quiet } { fontspec }
        \RequirePackage { fontspec }

        \__colorclass_load_file_or_config:Nnn \l__colorclass_load_custom_font_file_latin_bool { \l__colorclass_custom_font_file_latin_str }
          {
            \bool_if:NTF \l__projlib_font_useosf_bool
              {
                \setmainfont { TeXGyrePagellaX }
                  [
                    Extension      = .otf,
                    UprightFont    = *-Regular,
                    BoldFont       = *-Bold,
                    ItalicFont     = *-Italic,
                    BoldItalicFont = *-BoldItalic,
                    Numbers        = OldStyle ,
                  ]
                \setsansfont { texgyreheros }
                  [
                    Extension      = .otf ,
                    Scale          = MatchUppercase ,
                    UprightFont    = *-regular ,
                    BoldFont       = *-bold ,
                    ItalicFont     = *-italic ,
                    BoldItalicFont = *-bolditalic ,
                    Numbers        = OldStyle ,
                  ]
                % \setsansfont { SourceSansPro }
                %   [
                %     Extension      = .otf,
                %     UprightFont    = *-Regular,
                %     BoldFont       = *-Semibold,
                %     ItalicFont     = *-RegularIt,
                %     BoldItalicFont = *-SemiboldIt,
                %     WordSpace      = {1.25, 1, 1} ,
                %     Numbers        = OldStyle ,
                %   ]
              }
              {
                \setmainfont { TeXGyrePagellaX }
                  [
                    Extension      = .otf,
                    UprightFont    = *-Regular,
                    BoldFont       = *-Bold,
                    ItalicFont     = *-Italic,
                    BoldItalicFont = *-BoldItalic,
                  ]
                \setsansfont { texgyreheros }
                  [
                    Extension      = .otf ,
                    Scale          = MatchUppercase ,
                    UprightFont    = *-regular ,
                    BoldFont       = *-bold ,
                    ItalicFont     = *-italic ,
                    BoldItalicFont = *-bolditalic ,
                  ]
                % \setsansfont { SourceSansPro }
                %   [
                %     Extension      = .otf,
                %     UprightFont    = *-Regular,
                %     BoldFont       = *-Semibold,
                %     ItalicFont     = *-RegularIt,
                %     BoldItalicFont = *-SemiboldIt,
                %     WordSpace      = {1.25, 1, 1} ,
                %   ]
              }
            \setmonofont { NewCMMono10 }
              [
                Scale          = 1.05 ,
                Extension      = .otf,
                UprightFont    = *-Regular,
                BoldFont       = *-Bold,
                ItalicFont     = *-Italic,
                BoldItalicFont = *-BoldOblique,
              ]

            \projlib_language_set_linespacing_latin:n { \setstretch { 1.07 } }

            \hook_gput_code:nnn { begindocument/before } { colorclass }
              {
                \RenewEmph{[}{]}
                \RenewEmph{(}{)}
              }
          }
      }


    \PassOptionsToPackage { fontset = none, scheme = plain } { ctex }
    \RequirePackage { ctex }


    \__colorclass_load_file_or_config:Nnn \l__colorclass_load_custom_font_file_cjk_bool { \l__colorclass_custom_font_file_cjk_str }
      {
        \__colorclass_if_font_exist:nnn { SourceHanSerifSC-Regular }
          {
            \setCJKmainfont { SourceHanSerifSC }
              [
                UprightFont    = *-Regular,
                BoldFont       = *-Bold,
                ItalicFont     = *-Regular,
                BoldItalicFont = *-Bold,
              ]
          }
          {
            \setCJKmainfont { FandolSong }
              [
                Extension      = .otf,
                UprightFont    = *-Regular,
                BoldFont       = *-Bold,
                ItalicFont     = FandolKai-Regular ,
                BoldItalicFont = FandolKai-Regular ,
                BoldItalicFeatures = { FakeBold = 4 } ,
              ]
          }

        \__colorclass_if_font_exist:nnn { SourceHanSansSC-Regular }
          {
            \setCJKsansfont { SourceHanSansSC }
              [
                UprightFont    = *-Regular,
                BoldFont       = *-Bold,
                ItalicFont     = *-Regular,
                BoldItalicFont = *-Bold,
              ]
          }
          {
            \setCJKsansfont { FandolHei }
              [
                Extension      = .otf,
                UprightFont    = *-Regular,
                BoldFont       = *-Bold,
                ItalicFont     = *-Regular,
                BoldItalicFont = *-Bold,
              ]
          }

        \__colorclass_if_font_exist:nnn { SourceHanMonoSC-Regular }
          {
            \setCJKmonofont { SourceHanMonoSC }
              [
                UprightFont    = *-Regular,
                BoldFont       = *-Medium,
                ItalicFont     = *-Regular,
                BoldItalicFont = *-Medium,
              ]
          }
          {
            \setCJKmonofont { FandolFang-Regular.otf }
              [
                BoldFont       = * ,
                BoldFeatures   = { FakeBold = 4 } ,
                ItalicFont     = * ,
                BoldItalicFont = * ,
                BoldItalicFeatures = { FakeBold = 4 } ,
              ]
          }

        \bool_if:NT \g__projlib_language_enabled_schinese_bool
          {
            \__colorclass_if_font_exist:nnn { SourceHanSerifSC-Regular }
              {
                \setCJKfamilyfont { SCmain } { SourceHanSerifSC }
                  [
                    UprightFont    = *-Regular,
                    BoldFont       = *-Bold,
                    ItalicFont     = *-Regular,
                    BoldItalicFont = *-Bold,
                  ]
              }
              {
                \setCJKfamilyfont { SCmain } { FandolSong }
                  [
                    Extension      = .otf,
                    UprightFont    = *-Regular,
                    BoldFont       = *-Bold,
                    ItalicFont     = FandolKai-Regular ,
                    BoldItalicFont = FandolKai-Regular ,
                    BoldItalicFeatures = { FakeBold = 4 } ,
                  ]
              }
            \__colorclass_if_font_exist:nnn { SourceHanSansSC-Regular }
              {
                \setCJKfamilyfont { SCsans } { SourceHanSansSC }
                  [
                    UprightFont    = *-Regular,
                    BoldFont       = *-Bold,
                    ItalicFont     = *-Regular,
                    BoldItalicFont = *-Bold,
                  ]
              }
              {
                \setCJKfamilyfont { SCsans } { FandolHei }
                  [
                    Extension      = .otf,
                    UprightFont    = *-Regular,
                    BoldFont       = *-Bold,
                    ItalicFont     = *-Regular,
                    BoldItalicFont = *-Bold,
                  ]
              }
            \__colorclass_if_font_exist:nnn { SourceHanMonoSC-Regular }
              {
                \setCJKfamilyfont { SCmono } { SourceHanMonoSC }
                  [
                    UprightFont    = *-Regular,
                    BoldFont       = *-Medium,
                    ItalicFont     = *-Regular,
                    BoldItalicFont = *-Medium,
                  ]
              }
              {
                \setCJKfamilyfont { SCmono } { FandolFang-Regular.otf }
                  [
                    BoldFont       = * ,
                    BoldFeatures   = { FakeBold = 4 } ,
                    ItalicFont     = * ,
                    BoldItalicFont = * ,
                    BoldItalicFeatures = { FakeBold = 4 } ,
                  ]
              }
          }

        \bool_if:NT \g__projlib_language_enabled_tchinese_bool
          {
            \__colorclass_if_font_exist:nnn { SourceHanSerifTC-Regular }
              {
                \setCJKfamilyfont { TCmain } { SourceHanSerifTC }
                  [
                    UprightFont    = *-Regular,
                    BoldFont       = *-Bold,
                    ItalicFont     = *-Regular,
                    BoldItalicFont = *-Bold,
                  ]
              }
              {
                \setCJKfamilyfont { TCmain } { FandolSong }
                  [
                    Extension      = .otf,
                    UprightFont    = *-Regular,
                    BoldFont       = *-Bold,
                    ItalicFont     = FandolKai-Regular ,
                    BoldItalicFont = FandolKai-Regular ,
                    BoldItalicFeatures = { FakeBold = 4 } ,
                  ]
              }
            \__colorclass_if_font_exist:nnn { SourceHanSansTC-Regular }
              {
                \setCJKfamilyfont { TCsans } { SourceHanSansTC }
                  [
                    UprightFont    = *-Regular,
                    BoldFont       = *-Bold,
                    ItalicFont     = *-Regular,
                    BoldItalicFont = *-Bold,
                  ]
              }
              {
                \setCJKfamilyfont { TCsans } { FandolHei }
                  [
                    Extension      = .otf,
                    UprightFont    = *-Regular,
                    BoldFont       = *-Bold,
                    ItalicFont     = *-Regular,
                    BoldItalicFont = *-Bold,
                  ]
              }
            \__colorclass_if_font_exist:nnn { SourceHanMonoTC-Regular }
              {
                \setCJKfamilyfont { TCmono } { SourceHanMonoTC }
                  [
                    UprightFont    = *-Regular,
                    BoldFont       = *-Medium,
                    ItalicFont     = *-Regular,
                    BoldItalicFont = *-Medium,
                  ]
              }
              {
                \setCJKfamilyfont { TCmono } { FandolFang-Regular.otf }
                  [
                    BoldFont       = * ,
                    BoldFeatures   = { FakeBold = 4 } ,
                    ItalicFont     = * ,
                    BoldItalicFont = * ,
                    BoldItalicFeatures = { FakeBold = 4 } ,
                  ]
              }
          }

        \bool_if:NT \g__projlib_language_enabled_japanese_bool
          {
            \__colorclass_if_font_exist:nnn { SourceHanSerif-Regular }
              {
                \setCJKfamilyfont { JPmain } { SourceHanSerif }
                  [
                    UprightFont    = *-Regular,
                    BoldFont       = *-Bold,
                    ItalicFont     = *-Regular,
                    BoldItalicFont = *-Bold,
                  ]
              }
              {
                \setCJKfamilyfont { JPmain } { FandolSong }
                  [
                    Extension      = .otf,
                    UprightFont    = *-Regular,
                    BoldFont       = *-Bold,
                    ItalicFont     = FandolKai-Regular ,
                    BoldItalicFont = FandolKai-Regular ,
                    BoldItalicFeatures = { FakeBold = 4 } ,
                  ]
              }
            \__colorclass_if_font_exist:nnn { SourceHanSans-Regular }
              {
                \setCJKfamilyfont { JPsans } { SourceHanSans }
                  [
                    UprightFont    = *-Regular,
                    BoldFont       = *-Bold,
                    ItalicFont     = *-Regular,
                    BoldItalicFont = *-Bold,
                  ]
              }
              {
                \setCJKfamilyfont { JPsans } { FandolHei }
                  [
                    Extension      = .otf,
                    UprightFont    = *-Regular,
                    BoldFont       = *-Bold,
                    ItalicFont     = *-Regular,
                    BoldItalicFont = *-Bold,
                  ]
              }
            \__colorclass_if_font_exist:nnn { SourceHanMono-Regular }
              {
                \setCJKfamilyfont { JPmono } { SourceHanMono }
                  [
                    UprightFont    = *-Regular,
                    BoldFont       = *-Medium,
                    ItalicFont     = *-Regular,
                    BoldItalicFont = *-Medium,
                  ]
              }
              {
                \setCJKfamilyfont { JPmono } { FandolFang-Regular.otf }
                  [
                    BoldFont       = * ,
                    BoldFeatures   = { FakeBold = 4 } ,
                    ItalicFont     = * ,
                    BoldItalicFont = * ,
                    BoldItalicFeatures = { FakeBold = 4 } ,
                  ]
              }
          }

        \cs_new:Nn \colorclass_cjk_sffamily: {}
        \cs_new:Nn \colorclass_cjk_ttfamily: {}

        \hook_gput_code:nnn { cmd/sffamily/after } { colorclass } { \colorclass_cjk_sffamily: }
        \hook_gput_code:nnn { cmd/ttfamily/after } { colorclass } { \colorclass_cjk_ttfamily: }

        \AddLanguageSetting [schinese]
          {
            \cs_set:Nn \colorclass_cjk_sffamily: { \CJKfamily { SCsans } }
            \cs_set:Nn \colorclass_cjk_ttfamily: { \CJKfamily { SCmono } }
            \CJKfamily { SCmain }
          }
        \AddLanguageSetting [tchinese]
          {
            \cs_set:Nn \colorclass_cjk_sffamily: { \CJKfamily { TCsans } }
            \cs_set:Nn \colorclass_cjk_ttfamily: { \CJKfamily { TCmono } }
            \CJKfamily { TCmain }
          }
        \AddLanguageSetting [japanese]
          {
            \cs_set:Nn \colorclass_cjk_sffamily: { \CJKfamily { JPsans } }
            \cs_set:Nn \colorclass_cjk_ttfamily: { \CJKfamily { JPmono } }
            \CJKfamily { JPmain }
          }
      }


    \__colorclass_load_file_or_config:Nnn \l__colorclass_load_custom_font_file_math_bool { \l__colorclass_custom_font_file_math_str }
      {
        \bool_if:NF \g_projlib_font_already_set_bool
          {
            \RequirePackage { amssymb }
            \bool_if:NF \l__colorclass_fast_bool
              {
                \PassOptionsToPackage { warnings-off = { mathtools-colon, mathtools-overbracket } } { unicode-math }
                \RequirePackage { unicode-math }
                \unimathsetup { math-style = ISO, partial = upright, nabla = upright }
                \setmathfont { KpMath-Regular.otf }
                \setmathfont { KpMath-Regular.otf }
                  [
                    range = \amalg ,
                    Scale = 0.84625
                  ]
                \setmathfont { KpMath-Sans.otf }
                  [
                    range = { \sum }
                  ]
                \setmathfont { latinmodern-math.otf }
                  [
                    range = { frak, bffrak, \mitvarpi, \mupvarpi, \ast } ,
                    Scale = 1.10
                  ]
                \setmathfont { texgyrepagella-math.otf }
                  [
                    range = { `(, `) } ,
                    Scale = 1.10
                  ]
                \setmathfont { texgyrepagella-math.otf }
                  [
                    range = { \mathcomma, \mathsemicolon } ,
                  ]
                \setmathfont { texgyrepagella-math.otf }
                  [
                    range = { it / { Latin, latin }, bfit / { Latin, latin }, up / num, bfup / num }
                  ]
                \setmathfont { KpMath-Regular.otf } [ range = { cal, bfcal }, RawFeature=+ss01 ]
                % \setmathfont { KpMath-Regular.otf } [ range = {} ]

                % https://tex.stackexchange.com/a/646715
                \sys_if_engine_luatex:T
                  {
                    \mathitalicsmode=1
                  }

                \hook_gput_code:nnn { begindocument } { colorclass }
                  {
                    \cs_gset_eq:NN \overline \wideoverbar
                    \cs_gset_eq:NN \square \mdwhtsquare

                    % https://tex.stackexchange.com/a/678611
                    \newcommand{\limstrut}{\vrule depth0.2ex width 0pt}
                    \renewcommand{\varprojlim}{\mathop{\underleftarrow{{\lim}\limstrut}}}
                    \renewcommand{\varinjlim}{\mathop{\underrightarrow{{\lim}\limstrut}}}
                  }

                % https://tex.stackexchange.com/a/647789
                \hook_gput_code:nnn { begindocument } { colorclass }
                  {
                    \NewCommandCopy\unicodevdots\vdots
                    \RenewDocumentCommand{\vdots}{}{\mathrel{\loweredvdots}}
                  }
                \newcommand{\loweredvdots}{\mathpalette\loweredvdots@\relax}
                \newcommand{\loweredvdots@}[2]{%
                  \begingroup
                  \sbox\z@{$\m@th#1\unicodevdots$}%
                  \vrule width \z@ height 2.25\ht\z@ depth 0.012\ht\z@
                  \raisebox{0.25\height}{\usebox\z@}%
                  \endgroup
                }

                % https://tex.stackexchange.com/a/678998
                \hook_gput_code:nnn { begindocument } { colorclass }
                  {
                    \NewCommandCopy\standarddashv\dashv
                    \NewCommandCopy\standardvdash\vdash
                    \RenewDocumentCommand{\dashv}{}{\mathrel{\mathpalette\raisesymbol\standarddashv}}
                    \RenewDocumentCommand{\vdash}{}{\mathrel{\mathpalette\raisesymbol\standardvdash}}
                  }
                \newcommand{\raisesymbol}[2]{%
                  \begingroup
                  \sbox\z@{$\m@th#1A$}%
                  \sbox\tw@{$\m@th#1#2$}%
                  \raisebox{\dimexpr(\ht\z@-\ht\tw@)/2}{\usebox{\tw@}}%
                  \endgroup
                }

                \RequirePackage { tikz-cd }
                \tikzcdset { arrow~style = tikz, diagrams = { >={Stealth[round,length=3.4pt,width=6.15pt,inset=2.25pt]} } }

                \box_new:N \l__colorclass_xarrows_above_box
                \box_new:N \l__colorclass_xarrows_below_box
                \dim_new:N \l__colorclass_xarrows_length_dim
                \cs_new_protected:Nn \colorclass_xarrows_generic:nnnn
                  % #3 = option of \tikz
                  % #4 = edge of \draw
                  {
                    \hbox_set:Nn \l__colorclass_xarrows_below_box { \ensuremath { \scriptstyle #1 } }
                    \hbox_set:Nn \l__colorclass_xarrows_above_box { \ensuremath { \scriptstyle #2 } }
                    \dim_set:Nn \l__colorclass_xarrows_length_dim
                      { \dim_eval:n { \dim_max:nn { \box_wd:N \l__colorclass_xarrows_below_box } { \box_wd:N \l__colorclass_xarrows_above_box } + 1em } }
                    \mathrel
                      {
                        \tikz [ #3, line~width = .6pt, baseline = -.5ex, every~node/.style = { inner~sep = 0pt }, >={Stealth[round,length=3.4pt,width=6.15pt,inset=2.25pt]} ]
                          \draw (0,0) #4
                            node [ below = 3pt ] { \box_use:N \l__colorclass_xarrows_below_box }
                            node [ above = 2pt ] { \box_use:N \l__colorclass_xarrows_above_box }
                            ( \l__colorclass_xarrows_length_dim ,0) ;
                      }
                  }

                \RenewDocumentCommand \xrightarrow { O{} m }
                  {
                    \colorclass_xarrows_generic:nnnn { #1 } { #2 } { -> } { -- }
                  }
                \RenewDocumentCommand \xleftarrow { O{} m }
                  {
                    \colorclass_xarrows_generic:nnnn { #1 } { #2 } { <- } { -- }
                  }
                \RenewDocumentCommand \xleftrightarrow { O{} m }
                  {
                    \colorclass_xarrows_generic:nnnn { #1 } { #2 } { <-> } { -- }
                  }
                \RenewDocumentCommand \xhookrightarrow { O{} m }
                  {
                    \colorclass_xarrows_generic:nnnn { #1 } { #2 } {} { edge [ commutative~diagrams/hookrightarrow ] }
                  }
                \RenewDocumentCommand \xhookleftarrow { O{} m }
                  {
                    \colorclass_xarrows_generic:nnnn { #1 } { #2 } {} { edge [ commutative~diagrams/hookleftarrow ] }
                  }
                \RenewDocumentCommand \xmapsto { O{} m }
                  {
                    \colorclass_xarrows_generic:nnnn { #1 } { #2 } {} { edge [ commutative~diagrams/mapsto ] }
                  }
                \NewDocumentCommand \xlongequal { O{} m }
                  {
                    \colorclass_xarrows_generic:nnnn { #1 } { #2 } {} { edge [ commutative~diagrams/equal ] }
                  }
                \NewDocumentCommand \xtwoheadrightarrow { O{} m }
                  {
                    \colorclass_xarrows_generic:nnnn { #1 } { #2 } {} { edge [ commutative~diagrams/twoheadrightarrow ] }
                  }
                \NewDocumentCommand \xtwoheadleftarrow { O{} m }
                  {
                    \colorclass_xarrows_generic:nnnn { #1 } { #2 } {} { edge [ commutative~diagrams/twoheadleftarrow ] }
                  }

                \cs_new_protected:Nn \colorclass_xarrows_double_generic:nnnnnn
                  % #3 = option of \tikz
                  % #4 = edge 1 of \draw
                  % #5 = edge 2 of \draw
                  % #6 = sep/2
                  {
                    \hbox_set:Nn \l__colorclass_xarrows_below_box { \ensuremath { \scriptstyle #1 } }
                    \hbox_set:Nn \l__colorclass_xarrows_above_box { \ensuremath { \scriptstyle #2 } }
                    \dim_set:Nn \l__colorclass_xarrows_length_dim
                      { \dim_eval:n { \dim_max:nn { \box_wd:N \l__colorclass_xarrows_below_box } { \box_wd:N \l__colorclass_xarrows_above_box } + 1em } }
                    \mathrel
                      {
                        \tikz [ #3, line~width = .6pt, baseline = -.5ex, every~node/.style = { inner~sep = 0pt }, >={Stealth[round,length=3.4pt,width=6.15pt,inset=2.25pt]} ]
                          {
                            \draw (0,#6) #4
                              node [ below = 4pt+#6 ] { \box_use:N \l__colorclass_xarrows_below_box }
                              ( \l__colorclass_xarrows_length_dim ,#6) ;
                            \draw (0,-#6) #5
                              node [ above = 3pt+#6 ] { \box_use:N \l__colorclass_xarrows_above_box }
                              ( \l__colorclass_xarrows_length_dim ,-#6) ;
                          }
                      }
                  }

                \NewDocumentCommand \xrightrightarrows { O{} m }
                  {
                    \colorclass_xarrows_double_generic:nnnnnn { #1 } { #2 } {} { edge [ commutative~diagrams/rightarrow ] } { edge [ commutative~diagrams/rightarrow ] } { .2em }
                  }

                \NewDocumentCommand \xleftleftarrows { O{} m }
                  {
                    \colorclass_xarrows_double_generic:nnnnnn { #1 } { #2 } {} { edge [ commutative~diagrams/leftarrow ] } { edge [ commutative~diagrams/leftarrow ] } { .2em }
                  }

                \hook_gput_code:nnn { begindocument/end } { colorclass }
                  {
                    \RenewDocumentCommand \twoheadrightarrow {}
                      {
                        \colorclass_xarrows_generic:nnnn {
                          \mathchoice { \,\, } { \, } { } { }
                        } {} {} { edge [ commutative~diagrams/twoheadrightarrow ] }
                      }
                    \RenewDocumentCommand \twoheadleftarrow {}
                      {
                        \colorclass_xarrows_generic:nnnn {
                          \mathchoice { \,\, } { \, } { } { }
                        } {} {} { edge [ commutative~diagrams/twoheadleftarrow ] }
                      }
                  }
              }
          }
      }
%</lebhart|beaulivre>
  }

%<*lebhart|beaulivre>
% https://tex.stackexchange.com/a/419287
\char_set_catcode_active:n { `\· }
\cs_new_protected:Npn · { \ensuremath\cdot }
%</lebhart|beaulivre>

%%================================
%%  Graphics
%%================================
\RequirePackage { graphicx }
\graphicspath { { images/ } }
\RequirePackage { wrapfig }
\RequirePackage { float }
\RequirePackage { caption }
\captionsetup { font = small }

%%================================
%%  Icing on the cake
%%================================
\bool_if:NT \l__colorclass_fast_bool { \endinput }

\sys_if_engine_luatex:TF
  {
    \RequirePackage { lua-widow-control }
    \lwcsetup { balanced }
  }
  {
    \PassOptionsToPackage { all } { nowidow }
    \RequirePackage { nowidow }
  }

% https://tex.stackexchange.com/a/641824
\sys_if_engine_xetex:T
  {
    \RequirePackage { regexpatch }
    \skip_new:N \g_colorclass_parfillskip_skip
    \xpatchcmd{\@trivlist}{\@flushglue}{\g_colorclass_parfillskip_skip}{}{}
    \hook_gput_code:nnn { begindocument } { colorclass }
      {
        \skip_gset:Nn \g_colorclass_parfillskip_skip { 0pt plus \dim_eval:n { \linewidth - 3em } }
        \skip_gset_eq:NN \parfillskip \g_colorclass_parfillskip_skip
      }
  }
%</class>
%
%
%<*colorist>
\keys_define:nn { colorist }
  {
    , draft             .bool_set:N         = \l__colorist_fast_bool
    , draft             .initial:n          = { false }
    , fast              .bool_set:N         = \l__colorist_fast_bool

    , style             .str_set:N          = \l__colorist_style_str
    , style             .initial:n          = { fancy }
    , use-style         .str_set:N          = \l__colorist_style_str
    , use~style         .str_set:N          = \l__colorist_style_str
    , use style         .str_set:N          = \l__colorist_style_str
    , fancy             .meta:n             = { style = fancy }

    , use-boldface      .bool_set:N         = \l__colorist_use_boldface_bool
    , use-boldface      .initial:n          = { false }
    , use~boldface      .bool_set:N         = \l__colorist_use_boldface_bool
    , use boldface      .bool_set:N         = \l__colorist_use_boldface_bool
    , title-in-boldface .bool_set:N         = \l__colorist_use_boldface_bool
    , title~in~boldface .bool_set:N         = \l__colorist_use_boldface_bool
    , title in boldface .bool_set:N         = \l__colorist_use_boldface_bool
    , title-in-bold     .bool_set:N         = \l__colorist_use_boldface_bool
    , title~in~bold     .bool_set:N         = \l__colorist_use_boldface_bool
    , title in bold     .bool_set:N         = \l__colorist_use_boldface_bool

    , use-scshape       .bool_set:N         = \l__colorist_use_scshape_bool
    , use-scshape       .initial:n          = { false }
    , use~scshape       .bool_set:N         = \l__colorist_use_scshape_bool
    , use scshape       .bool_set:N         = \l__colorist_use_scshape_bool
    , title-in-scshape  .bool_set:N         = \l__colorist_use_scshape_bool
    , title~in~scshape  .bool_set:N         = \l__colorist_use_scshape_bool
    , title in scshape  .bool_set:N         = \l__colorist_use_scshape_bool

    , runin             .bool_set:N         = \l__colorist_runin_bool
    , runin             .initial:n          = { false }

    , indent-items      .dim_set:N         = \l__colorist_item_indentation_dim
    , indent-items      .initial:n         = { 0pt }
    , indent-items      .default:n         = { \parindent }
    , indent~items      .dim_set:N         = \l__colorist_item_indentation_dim
    , indent~items      .default:n         = { \parindent }
    , indent items      .dim_set:N         = \l__colorist_item_indentation_dim
    , indent items      .default:n         = { \parindent }
    , indent-lists      .dim_set:N         = \l__colorist_item_indentation_dim
    , indent-lists      .default:n         = { \parindent }
    , indent~lists      .dim_set:N         = \l__colorist_item_indentation_dim
    , indent~lists      .default:n         = { \parindent }
    , indent lists      .dim_set:N         = \l__colorist_item_indentation_dim
    , indent lists      .default:n         = { \parindent }

    , theorem-in-new-line .bool_set:N       = \l__colorist_theorem_in_new_line_bool
    , theorem-in-new-line .initial:n        = { false }
    , theorem~in~new~line .bool_set:N       = \l__colorist_theorem_in_new_line_bool
    , theorem in new line .bool_set:N       = \l__colorist_theorem_in_new_line_bool

    , unknown           .code:n             = {
                                                \PassOptionsToPackage { \CurrentOption } { projlib-language }
                                                \PassOptionsToPackage { \CurrentOption } { projlib-author }
                                                \PassOptionsToPackage { \CurrentOption } { projlib-datetime }
                                                \PassOptionsToPackage { \CurrentOption } { projlib-draft }
                                                \PassOptionsToPackage { \CurrentOption } { projlib-font }
                                                \PassOptionsToPackage { \CurrentOption } { projlib-logo }
                                                \PassOptionsToPackage { \CurrentOption } { projlib-math }
                                                \PassOptionsToPackage { \CurrentOption } { projlib-paper }
                                                \PassOptionsToPackage { \CurrentOption } { projlib-theorem }
                                              }
  }
\ProcessKeyOptions [ colorist ]

\bool_new:N \l__colorist_is_book_bool
\cs_if_exist:cTF { c@chapter }
  {
    \bool_set_true:N \l__colorist_is_book_bool
  }
  {
    \bool_set_false:N \l__colorist_is_book_bool
  }

%%================================
%%  Paper configuration
%%================================
\RequirePackage { projlib-paper }

%%================================
%%  Multi-language support
%%================================
\RequirePackage { projlib-language }

%%================================
%%  Loading the style
%%================================
\exp_args:No \RequirePackage { colorist- \l__colorist_style_str }
%</colorist>
%
%<*style>
\IfPackageLoadedTF { colorist } {}
  {
    \msg_new:nnn { \@currname }
      { colorist-not-loaded }
      { "#1"~is~an~internal~style~of~"colorist".~To~use~it,~you~must~load~the~package~"colorist"~first. }
    \msg_warning:nnx { \@currname } { colorist-not-loaded } { \@currname }
    \endinput
  }

%%================================
%%  Title fonts
%%================================
\RequirePackage { relsize }
\RequirePackage { anyfontsize }

\NewCommandCopy \colorist_original_bfseries: \bfseries
\bool_new:N \l_colorist_is_under_bfseries_bool
\bool_set_false:N \l_colorist_is_under_bfseries_bool
\RenewDocumentCommand \bfseries { }
  {
    \bool_if:NF \l_colorist_is_under_bfseries_bool
      {
        \colorlet{colorist-temp-color}{.}
        \color{colorist-temp-color!90!paper}
      }
    \colorist_original_bfseries:
    \bool_set_true:N \l_colorist_is_under_bfseries_bool
  }
\bool_if:NTF \l__colorist_use_boldface_bool
  {
    \cs_new:Nn \colorist_bfseries: { \bfseries }
  }
  {
    \cs_new:Nn \colorist_bfseries: {}
  }

\bool_if:NTF \l__colorist_use_scshape_bool
  {
    \cs_new:Nn \colorist_scshape: { \scshape }
  }
  {
    \cs_new:Nn \colorist_scshape: {}
  }


\tl_new:N \g_colorist_title_font_common_tl

\tl_new:N \g_colorist_title_font_part_tl
\tl_new:N \g_colorist_title_font_chapter_tl
\tl_new:N \g_colorist_title_font_section_tl
\tl_new:N \g_colorist_title_font_subsection_tl
\tl_new:N \g_colorist_title_font_subsubsection_tl
\tl_new:N \g_colorist_title_font_paragraph_tl

%<*colorist-fancy>
\tl_gset:Nn \g_colorist_title_font_common_tl        { \sffamily }
\tl_gset:Nn \g_colorist_title_font_part_tl          { \colorist_bfseries: \g_colorist_title_font_common_tl }
\tl_gset:Nn \g_colorist_title_font_chapter_tl       { \colorist_bfseries: \g_colorist_title_font_common_tl }
\tl_gset:Nn \g_colorist_title_font_section_tl       { \colorist_bfseries: \g_colorist_title_font_common_tl }
\tl_gset:Nn \g_colorist_title_font_subsection_tl    { \colorist_bfseries: \g_colorist_title_font_common_tl }
\tl_gset:Nn \g_colorist_title_font_subsubsection_tl { \colorist_bfseries: \g_colorist_title_font_common_tl }
\tl_gset:Nn \g_colorist_title_font_paragraph_tl     { \colorist_bfseries: \g_colorist_title_font_common_tl }
%</colorist-fancy>

%%================================
%% Color
%%================================
%<*colorist-fancy>
\definecolor{maintheme}{RGB}{70,130,180}
\definecolor{forestgreen}{RGB}{21,122,81}
\definecolor{lightorange}{RGB}{255,185,88}
%</colorist-fancy>

%%================================
%%  Footer
%%================================
\RequirePackage { geometry }
\RequirePackage { fancyhdr }
\RequirePackage { extramarks }

\hook_gput_code:nnn { begindocument/before } { colorist }
  {
    \fancyhfoffset { 0pt }
  }

\tl_new:N \l_colorist_leftmark_tl
\tl_new:N \l_colorist_rightmark_tl

\tl_set:Nn \l_colorist_leftmark_tl
  {
    \begin{minipage}[t]{.833\textwidth}
      \lastleftmark
    \end{minipage}
  }
\tl_set:Nn \l_colorist_rightmark_tl
  {
    \begin{minipage}[t]{.833\textwidth}
      \filleft
      \lastrightmark
    \end{minipage}
  }

\fancypagestyle { fancy }
  {
    \fancyhf { }
    \if@twoside
      \fancyfoot[RO]
        {
          \sffamily
          \textcolor { main-text!30!paper } { \small \l_colorist_rightmark_tl }
          \rlap
            {
              \nobreakspace \nobreakspace \nobreakspace \nobreakspace
              \textcolor { main-text!75!paper } { \colorist_bfseries: \thepage }
            }
        }
      \fancyfoot[LE]
        {
          \leavevmode
          \sffamily
          \llap
            {
              \textcolor { main-text!75!paper } { \colorist_bfseries: \thepage }
              \nobreakspace \nobreakspace \nobreakspace \nobreakspace
            }
          \textcolor { main-text!30!paper } { \l_colorist_leftmark_tl }
        }
    \else
      \fancyfoot[R]
        {
          \sffamily
          \textcolor { main-text!30!paper } { \small \l_colorist_rightmark_tl }
          \rlap
            {
              \nobreakspace \nobreakspace \nobreakspace \nobreakspace
              \textcolor { main-text!75!paper } { \colorist_bfseries: \thepage }
            }
        }
    \fi
    \renewcommand { \headrulewidth } { 0pt }
  }
\pagestyle { fancy }

\fancypagestyle { plain }
  {
    \fancyhf { }
    \if@twoside
      \fancyfoot[RO]
        {
          \sffamily
          \nobreakspace
          \rlap
            {
              \nobreakspace \nobreakspace \nobreakspace \nobreakspace
              \textcolor { main-text!75!paper } { \colorist_bfseries: \thepage }
            }
        }
      \fancyfoot[LE]
        {
          \leavevmode
          \sffamily
          \llap
            {
              \textcolor { main-text!75!paper } { \colorist_bfseries: \thepage }
              \nobreakspace \nobreakspace \nobreakspace \nobreakspace
            }
        }
    \else
      \fancyfoot[R]
        {
          \sffamily
          \nobreakspace
          \rlap
            {
              \nobreakspace \nobreakspace \nobreakspace \nobreakspace
              \textcolor { main-text!75!paper } { \colorist_bfseries: \thepage }
            }
        }
    \fi
    \renewcommand { \headrulewidth } { 0pt }
  }

\bool_if:NTF \l__colorist_is_book_bool
  {
    \if@twoside
        \renewcommand{\chaptermark}[1]{\markboth{\textsc{#1}}{}}
    \else
        \renewcommand{\chaptermark}[1]{\markboth{\textsc{#1}}{\textsc{#1}}}
    \fi
    \renewcommand*{\sectionmark}[1]{
      \markright{\thesection\nobreakspace\nobreakspace#1}}
  }
  {
    \if@twoside
        \renewcommand*{\sectionmark}[1]{\markboth{\textsc{#1}}{}}
    \else
        \renewcommand*{\sectionmark}[1]{\markboth{\textsc{#1}}{\textsc{#1}}}
    \fi
  }

\renewcommand*{\thefootnote}{\textcolor{main-text!45!paper}{\arabic{footnote}}}

\bool_if:NT \l__colorist_is_book_bool
  {
    \hook_gput_code:nnn { cmd/frontmatter/before } { colorist }
      {
        \renewcommand*{\thefootnote}{\textcolor{main-text!45!paper}{\fnsymbol{footnote}}}
      }
    \hook_gput_code:nnn { cmd/mainmatter/before } { colorist }
      {
        \setcounter{footnote}{0}
        \renewcommand*{\thefootnote}{\textcolor{main-text!45!paper}{\arabic{footnote}}}
      }
  }

%%================================
%%  Title format
%%================================
\RequirePackage [ explicit, newparttoc ] { titlesec }
% \renewcommand{\bottomtitlespace}{.1\textheight}
\PassOptionsToPackage { normalem } { ulem }
\RequirePackage { ulem }

\PassOptionsToPackage { many } { tcolorbox }
\RequirePackage { tcolorbox }
\bool_if:NT \l__colorist_fast_bool { \tcbstartdraftmode }

\newcommand{\partstring}{\MakeUppercase{{\partname\nobreakspace\protect\thepart}}}

\AddLanguageSetting
  {
    \renewcommand{\partstring}{\MakeUppercase{{\partname\nobreakspace\protect\thepart}}}
  }
\AddLanguageSetting [ schinese ]
  {
    \renewcommand{\partstring}{第 \nobreakspace\thepart\nobreakspace 部分}
  }
\AddLanguageSetting [ tchinese ]
  {
    \renewcommand{\partstring}{第 \nobreakspace\thepart\nobreakspace 部分}
  }
\AddLanguageSetting [ japanese ]
  {
    \renewcommand{\partstring}{第 \nobreakspace\thepart\nobreakspace 部}
  }

\bool_if:NTF \l__colorist_is_book_bool
  {
    \setcounter{secnumdepth}{3}

    %% Part
    \titleclass{\part}{top} % make part like a chapter
    \titleformat{\part}[display]
      { \g_colorist_title_font_part_tl \filleft}
      {
        \thispagestyle{empty}
        \begin{tikzpicture}[remember~picture,overlay]
          \fill[maintheme!10!paper] (current~page.north~west) rectangle (current~page.south~east);
          \node at ($(current~page.north~west)+(15em,-15em)$) {\normalfont\textcolor{maintheme}{\scalebox{12}{\thepart}}};
        \end{tikzpicture}
      }
      {1em}
      {\fontsize{20}{24}\selectfont\MakeUppercase{#1}}
    \titleformat{name=\part,numberless}[display]
      {% \phantomsection\addcontentsline{toc}{part}{#1}%
         \g_colorist_title_font_part_tl \filleft}
      {
        \thispagestyle{empty}
        \begin{tikzpicture}[remember~picture,overlay]
          \fill[maintheme!10!paper] (current~page.north~west) rectangle (current~page.south~east);
          \node at ($(current~page.north~west)+(15em,-15em)$) {\normalfont\textcolor{maintheme}{\scalebox{12}{$*$}}};
        \end{tikzpicture}
      }
      {1em}
      {\fontsize{20}{24}\selectfont\MakeUppercase{#1}}
    \titlespacing*{\part}{0pt}{5em}{6em}
    %% Text after part
    \newcommand{\parttext}[1]{
        \vfill
        \begin{flushright}
            \begin{minipage}{0.833\textwidth}
                \color{main-text!80!paper}\raggedleft#1
            \end{minipage}
        \end{flushright}
        \vfill\vfill
        \cleardoublepage
    }

    %% Chapter
    % Numbered chapter title
    \cs_new_protected:Nn \colorist_chapter_inner:nn
      % #1 = number
      % #2 = title
      {
        \tcbsidebyside[enhanced,sidebyside~adapt=right,sidebyside~align=bottom,
        colback=paper,frame~hidden,
        segmentation~code={
            \filldraw[maintheme] (segmentation.north)
                -- ($(segmentation.east)-(12pt,0)$)
                -- ($(segmentation.west)+(12pt,0)$)
                -- (segmentation.south);}
        ]{\filleft#2}{\normalfont\textcolor{maintheme}{\scalebox{4}{#1}}}
      }
    % Numberless chapter title
    \cs_new_protected:Nn \colorist_chapter_inner:n
      % #1 = title
      {
        \begin{tcolorbox}[
            enhanced,
            width = 0.67\textwidth,
            colback=paper,frame~hidden,
            halign=center]
            #1
            \vspace{-.6em}
            \begin{center}
                \begin{tikzpicture}
                    \filldraw[maintheme] (-4em,0) -- (4em,0) -- (0,-.1em) -- (0,.1em);
                \end{tikzpicture}
            \end{center}
        \end{tcolorbox}
      }

    \titleformat{name=\chapter}
      { \g_colorist_title_font_chapter_tl \colorist_scshape:\huge} % Format
      {} % Label
      {0mm} % Sep
      { \colorist_chapter_inner:nn { \thechapter } { #1 } } % Before-code
    \titlespacing*{name=\chapter}
      {0em}{*2}{0em} % {left}{before-sep}{after-sep}

    \titleformat{name=\chapter, numberless}
      {\filcenter \g_colorist_title_font_chapter_tl \colorist_scshape:\huge}
      {}
      {0mm}
      { \colorist_chapter_inner:n { #1 } }
    \titlespacing*{name=\chapter, numberless}
      {0em}{*2}{0em}

    %% Section
    \titleformat{\section}
      {\color{maintheme} \g_colorist_title_font_section_tl \large}
      {\thesection}{.75em}{#1}

    %% Subsection
    \titleformat{\subsection}
      { \g_colorist_title_font_subsection_tl }{\thesubsection}{.75em}
      {#1}
  }
  {
    %% Part
    \titleformat{\part}[display]
      { \g_colorist_title_font_part_tl \filleft}
      {\partstring}
      {.3em}
      {\fontsize{16}{0}\selectfont\MakeUppercase{#1}}
    \titleformat{name=\part,numberless}[display]
      {% \phantomsection\addcontentsline{toc}{part}{#1}
         \g_colorist_title_font_part_tl \filleft}
      {\phantom{\MakeUppercase{\partname}}}
      {.3em}
      {\fontsize{16}{0}\selectfont\MakeUppercase{#1}}
    %% Text after part
    \newcommand{\parttext}[1]
      {
        \begin{flushright}
            \begin{minipage}{0.833\textwidth}
                \color{main-text!80!paper}\raggedleft#1
            \end{minipage}
        \end{flushright}
      }

    %% Section
    \titleformat{\section}
      {\color{maintheme} \g_colorist_title_font_section_tl \large}
      {\thesection}{.75em}{\colorist_scshape: #1}

    %% Subsection
    \titleformat{\subsection}
      { \g_colorist_title_font_subsection_tl }{\thesubsection}{.75em}
      {#1}
  }

%% Subsubsection
%<*colorist-fancy>
\bool_if:NTF \l__colorist_runin_bool
  {
    \titleformat{\subsubsection}[runin]
      {\color{main-text!70!paper} \g_colorist_title_font_subsubsection_tl }
      {\thesubsubsection}
      {.5em}
      {#1.}
      [\hspace*{.3em}]
  }
  {
    \titleformat{\subsubsection}
      {\color{main-text!70!paper} \g_colorist_title_font_subsubsection_tl }
      {\thesubsubsection}
      {.5em}
      {#1}
  }
%</colorist-fancy>

%% Paragraph
\titleformat{\paragraph}[runin]
  {\color{main-text!50!paper} \g_colorist_title_font_paragraph_tl }{\theparagraph}{1em}{#1}

\titlespacing{\section}{0pt}{1\baselineskip plus .5\baselineskip minus .2\baselineskip}{.6\baselineskip plus .3\baselineskip minus .2\baselineskip}
\titlespacing{\subsection}{0pt}{.75\baselineskip plus .3\baselineskip minus .2\baselineskip}{.4\baselineskip plus .2\baselineskip minus .1\baselineskip}
\bool_if:NTF \l__colorist_runin_bool
  {
    \titlespacing{\subsubsection}{0pt}{.5\baselineskip plus .2\baselineskip minus .1\baselineskip}{0pt}
  }
  {
    \titlespacing{\subsubsection}{0pt}{.5\baselineskip plus .2\baselineskip minus .1\baselineskip}{.3\baselineskip plus .2\baselineskip minus .1\baselineskip}
  }

%%================================
%%  ToC format
%%================================
\RequirePackage { titletoc }
\titlecontents{part}
  [0em]
  {\addvspace{1.5pc}\large\filcenter\sffamily \colorist_bfseries: }
  {\textcolor{maintheme}{\bfseries\thecontentslabel}\nopagebreak\\\nopagebreak\uppercase}
  {}
  {} % without page number
  [\addvspace{.5pc}]

\bool_if:NTF \l__colorist_is_book_bool
  {
    \titlecontents{chapter}
      [2em] % i.e., 0em (part) + 2em
      {\addvspace{1pc} \color{maintheme} \normalfont \sffamily \colorist_bfseries: \colorist_scshape: }
      {\contentslabel[ \raisebox{-.03\baselineskip}{ \large \normalfont \sffamily \colorist_bfseries: \thecontentslabel } ]{2em}}
      {\hspace*{-2em}}
      {\titlerule*[10pt]{\parbox{3pt}{\hspace*{-.25pt}\textcolor{main-text!15!paper}{.}}}\color{maintheme}\normalfont\sffamily\contentspage}
    \titlecontents{section}
      [5.25em] % i.e., 2em (chapter) + 2.75em + 0.5em
      {\addvspace{.3pc}\normalfont\color{main-text}\sffamily}
      {\contentslabel{2.75em}}
      {\hspace*{-2.75em}}
      {\titlerule*[10pt]{\parbox{3pt}{\textcolor{main-text!15!paper}{.}}}\color{main-text}\contentspage}
    \titlecontents{subsection}
      [9em] % i.e., 5.25em (section) + 3.5em + 0.25em
      {\addvspace{.15pc}\normalfont\color{main-text!60!paper}\sffamily}
      {\contentslabel{3.50em}}
      {\hspace*{-3.50em}}
      {\titlerule*[10pt]{\parbox{3pt}{\textcolor{main-text!15!paper}{.}}}\color{main-text!60!paper}\contentspage}
    \titlecontents{subsubsection}
      [13em] % i.e., 9em (subsection) + 3.75em + 0.25em
      {\normalfont\color{main-text!45!paper}\sffamily}
      {\contentslabel{3.75em}}
      {\hspace*{-3.75em}}
      {\titlerule*[10pt]{\parbox{3pt}{\textcolor{main-text!15!paper}{.}}}\color{main-text!45!paper}\contentspage}
  }
  {
    \titlecontents{section}
      [2em] % i.e., 0em (part) + 2em
      {\addvspace{.3pc} \color{maintheme} \normalfont \sffamily \colorist_scshape: }
      {\contentslabel[ { \normalfont \sffamily \thecontentslabel } ]{1.75em}}
      {\hspace*{-1.75em}}
      {\titlerule*[10pt]{\parbox{3pt}{\textcolor{main-text!15!paper}{.}}}\color{maintheme}\normalfont\sffamily\contentspage}
    \titlecontents{subsection}
      [5em] % i.e., 2em (section) + 2.5em + 0.5em
      {\addvspace{.15pc}\normalfont\sffamily}
      {\contentslabel{2.5em}}
      {\hspace*{-2.5em}}
      {\titlerule*[10pt]{\parbox{3pt}{\textcolor{main-text!15!paper}{.}}}\color{main-text!45!paper}\contentspage}
    \titlecontents{subsubsection}
      [9em] % i.e., 5em (subsection) + 3.25em + 0.75em
      {\normalfont\sffamily}
      {\contentslabel{3.25em}}
      {\hspace*{-3.25em}}
      {\titlerule*[10pt]{\parbox{3pt}{\textcolor{main-text!15!paper}{.}}}\color{main-text!45!paper}\contentspage}
  }

%%================================
%%  Lists
%%================================
\PassOptionsToPackage { inline } { enumitem }
\RequirePackage { enumitem }
\setlistdepth{10}
\setlist{noitemsep, topsep=.33\topsep-.5\parskip}
% \setlist{topsep = .1\baselineskip, parsep=0pt, itemsep=.05\baselineskip}
\setlist[enumerate]{labelsep=*, leftmargin=*}
\setlist[enumerate,1]{label = \normalfont\arabic*$\mskip-.5mu\big)$,
    ref = \normalfont\color{.!45!paper}\arabic*$\mskip-.5mu\big)$,
    leftmargin= \l__colorist_item_indentation_dim + \maxof{\parindent}{1.5em} }
    % labelindent= \l__colorist_item_indentation_dim }
\setlist[enumerate,2]{label = \normalfont\roman*$\mskip-.5mu\big)$,
    ref = \normalfont\color{.!45!paper}\arabic{enumi}.\roman*$\mskip-.5mu\big)$}
\setlist[enumerate,3]{label = \normalfont\emph{\alph*}$\mskip-.5mu\big)$,
    ref = \normalfont\color{.!45!paper}\arabic{enumi}.\roman{enumii}.\emph{\alph*}$\mskip-.5mu\big)$}

\setlist[description]{font=\normalfont\colorist_bfseries: ,
    labelindent= \l__colorist_item_indentation_dim }

\renewlist{itemize}{itemize}{10}
\setlist[itemize]{leftmargin=*,label=\textcolor{.!27!paper}{$\cdot$}}
\AddLanguageSetting { \setlist[itemize,1]{label=\textcolor{.!27!paper}{$\bullet$},leftmargin= \l__colorist_item_indentation_dim + \maxof{\parindent}{1.5em}} }
\AddLanguageSetting [french] { \setlist[itemize,1]{label=\textcolor{.!39!paper}{\rule[.2\baselineskip]{.8em}{.75pt}},leftmargin= \l__colorist_item_indentation_dim + \maxof{\parindent}{1.5em} } }
\setlist[itemize,2]{label=\textcolor{.!27!paper}{\rule[.2\baselineskip]{.55em}{.75pt}}}
% \setlist[itemize,3]{label=\textcolor{.!27!paper}{$\diamond$}}
\setlist[itemize,3]{label=\textcolor{.!27!paper}{$\circ$}}
\setlist[itemize,4]{label=\textcolor{.!27!paper}{$\ast$}}

%%================================
%%  Blank page
%%================================
\projlib_langauge_define_multilingual_text:Nn \bl@nkpagetext
  {
    , EN = This~page~is~intentionally~left~blank
    , FR = Cette~page~est~intentionnellement~laissée~vide
    , DE = Diese~Seite~wurde~absichtlich~leer~gelassen
    , IT = Questa~pagina~è~stata~lasciata~vuota~intenzionalmente
    , PT = Esta~página~foi~intencionalmente~deixada~em~branco
    , BR = Esta~página~foi~intencionalmente~deixada~em~branco
    , ES = Esta~página~se~ha~dejado~intencionadamente~en~blanco
    , CN = \ziju{0.2} 此页为有意留为空白
    , TC = \ziju{0.2} 此頁為有意留為空白
    , JP = このページは意図的に空白にしてあります
    , RU = Эта~страница~намеренно~оставлена~пустой
  }
\renewcommand{\cleardoublepage}{
  \relax
  \clearpage
  \if@twoside\ifodd\c@page\relax\else
  \thispagestyle{empty}
  \hook_gput_next_code:nn { shipout/background }
    {
      \put(0.5\paperwidth,-0.5\paperheight){
      \makebox[0pt]{\large\color{main-text!10!paper}\g_colorist_title_font_common_tl\bl@nkpagetext}}
    }
  \null\newpage\fi\fi
}

%%================================
%%  Index
%%================================
\RequirePackage { imakeidx }
\makeindex[intoc]

\RequirePackage { silence }
\ExplSyntaxOff
\WarningFilter{latex}{Writing or overwriting file}
\begin{filecontents*}[overwrite]{\jobname.mst}
delim_0 "\\IndexDotfill " % Filler between section heading and page number
delim_1 "\\IndexDotfill " % Filler between subsection heading and page number
headings_flag 1
heading_prefix "\\IndexHeading{"
heading_suffix "}\n"
\end{filecontents*}
\ExplSyntaxOn

\projlib_langauge_define_multilingual_text:Nn \index_symbols_name
  {
    , EN = Symbols
    , FR = Symboles
    , DE = Symbole
    , IT = Simboli
    , PT = Símbolos
    , BR = Símbolos
    , ES = Símbolos
    , CN = 符号
    , TC = 符號
    , JP = 記号
    , RU = Символы
  }

\newcommand*{\IndexDotfill}
  {
    \null\nobreak
    \leaders \hbox to .67em {\hss \textcolor{main-text!15!paper}{.} \hss} \hskip1em plus1fill
  }
\newcommand*{\IndexLinebreak}
  {
    \nobreakspace\textcolor{main-text!45!paper}{\raisebox{.4ex}{.}\raisebox{.2ex}{.}}
    \item\hspace*{\hangindent}
    \textcolor{main-text!45!paper}{\raisebox{.45ex}{.}\raisebox{.25ex}{.}}\:
    \unskip
  }
% % Below is another version that can produce a better automatic result,
% % however if you are willing to make some manual interventions, you would be able to get even better result,
% % thus this version is only left here for reference
% \newcommand*{\IndexDotfill}
%   {
%     \leaders \hbox to .67em {\hss \textcolor{main-text!15!paper}{.} \hss}\hfil
%     \penalty0\vadjust{}
%     \leaders \hbox to .67em {\hss \textcolor{main-text!15!paper}{.} \hss}\hskip1em plus1fill
%   }
% \newcommand*{\IndexLinebreak}{~}
% \hook_gput_code:nnn { cmd/theindex/after } { colorist } { \rightskip=1pc \parfillskip=-\rightskip }

\newcommand*{\IndexHeading}[1]
  {
    \str_if_eq:nnTF { #1 } { Symbols }
      { \tl_set:Nn \l_tmpa_tl { \index_symbols_name } }
      { \tl_set:Nn \l_tmpa_tl { #1 } }
    \tikz\node[
      rounded~corners=5pt,
      draw=maintheme,
      fill=maintheme!10,
      line~width=1pt,
      inner~sep=5pt,
      align=center,
      font=\large\sffamily\colorist_bfseries:,
      minimum~width=\linewidth-\pgflinewidth,
    ] { \l_tmpa_tl };
    \nopagebreak
    \par
    \vspace{.3\baselineskip}
  }

\renewcommand*{\indexspace}
  {
    \par
    \vspace{2pc plus .5pc minus .3pc}
  }

% Prevent column break before the first sub-entry in the index
% See https://tex.stackexchange.com/a/132415
\bool_new:N \l__colorist_if_first_subitem_bool
\renewcommand*{\@idxitem}
  {
    \par\hangindent40\p@
    \bool_set_true:N \l__colorist_if_first_subitem_bool
  }
\renewcommand*{\subitem}
  {
    \par\hangindent40\p@
    \bool_if:NT \l__colorist_if_first_subitem_bool
      {
        \nobreak
        \bool_set_false:N \l__colorist_if_first_subitem_bool
      }
    \hspace*{20\p@}
  }

\hook_gput_code:nnn { begindocument/before } { colorist }
  {
    \bool_if:NF \l__colorist_fast_bool
      {
        \hook_gput_code:nnn { cmd/printindex/before } { colorist } { \bookmarksetup{startatroot} }
      }
  }

%%================================
%%  Draft mark
%%================================
\RequirePackage { projlib-draft }

%%================================
%% Icons
%%================================
\RequirePackage { tikz }
\NewDocumentCommand \colorist_icon_ideabulb:w { O{0.15} m }
  % #1 = scale
  % #2 = color
  {
    \scalebox{#1}{
    \begin{tikzpicture}
        \filldraw[draw=#2,fill=#2] (0,0) circle [radius=1cm];
        \filldraw[draw=paper,fill=paper,rounded~corners=0.8pt]
            [rotate=20] (-0.26,-0.66) rectangle (-0.06,-0.6)
            [xshift=-0.4mm,yshift=1mm] (-0.26,-0.66) rectangle (0.02,-0.6)
            [xshift=0.4mm,yshift=1mm] (-0.26,-0.66) rectangle (-0.06,-0.6);
        \draw[draw=paper,line~width=0.7mm] (-0.18,-0.46)
            .. controls (-0.18,-0.28) and (-0.28,-0.12) ..(-0.4,0.1)
            .. controls (-0.55,0.4) and (-0.3,0.64) ..(0,0.64)
            .. controls (0.3,0.64) and (0.55,0.4) ..(0.4,0.1)
            .. controls (0.28,-0.12) and (0.18,-0.28) ..(0.18,-0.46);
    \end{tikzpicture}}
  }

\NewDocumentCommand \colorist_icon_questionmark:w { O{0.15} m }
  % #1 = scale
  % #2 = color
  {
    \scalebox{#1}{
    \begin{tikzpicture}
        \filldraw[draw=#2,fill=#2] (0,0) circle [radius=1cm];
        \filldraw[paper,yshift=0.5mm,scale=0.9] (-0.4,0.1) circle [radius=0.77mm];
        \draw[draw=paper,line~width=1.5mm,yshift=0.5mm,scale=0.9] (-0.4,0.1)
            .. controls (-0.55,0.4) and (-0.3,0.64) ..(0,0.64)
            .. controls (0.3,0.64) and (0.55,0.4) ..(0.4,0.1)
            .. controls (0.28,-0.12) and (0.05,-0.28) ..(0.05,-0.3)
            .. controls (0,-0.36) and (0.0,-0.45) ..(0.0,-0.5);
        \fill[fill=paper,rounded~corners=0.6mm]
            (-0.09,-0.75) rectangle (0.09,-0.53);
    \end{tikzpicture}}
  }

%%================================
%%  Theorems
%%================================
\RequirePackage { mathtools }
\RequirePackage { amsthm }

% Change equation numbers to gray
% \def\tagform@#1{\maketag@@@{\textcolor{main-text!39!paper}{(\ignorespaces#1\unskip\@@italiccorr)}}}
\def\tagform@#1{\maketag@@@{\textcolor{.!39!paper}{(\ignorespaces#1\unskip\@@italiccorr)}}}

\PassOptionsToPackage { nopatch = eqnum } { microtype }


\bool_if:NTF \l__colorist_theorem_in_new_line_bool
  {
    \newtheoremstyle{simple}
      {}{}
      {\normalfont}{}
      {\normalfont}{}
      {0pt}
      {
        \rlap{\vbox{\hbox{\parbox{\linewidth}{
          {\thmname{#1}\thmnumber{\nobreakspace #2}}
          {\color{main-text!50!paper}\thmnote{\hspace{.4em}$($#3$)$}}
        }}\hbox{\strut}\vspace{0pt}}}
      }
  }
  {
    \newtheoremstyle{simple}
      {}{}
      {\normalfont}{}
      {\normalfont}{}
      {0pt}
      {{\thmname{#1}\nobreakspace\thmnumber{#2}}
        {\color{main-text!50!paper}\thmnote{\hspace{.4em}$($#3$)$}}\nobreakspace\nobreakspace{\normalfont\textcolor{main-text!27!paper}{---}}\nobreakspace\nobreakspace}
  }

\newcommand{\customqedsymbol}{
  \makebox[1em]{\color{.!27!paper}\rule[-0.1em]{.95em}{.95em}}}
\let\qedsymbol\customqedsymbol

% The style of the theorem-like environment that will be wrapped into the color box
\bool_if:NTF \l__colorist_theorem_in_new_line_bool
  {
    \newtheoremstyle{basic}
      {}{}
      {\normalfont}{}
      {}{}
      {0pt}
      {
        \rlap{\vbox{\hbox{\parbox{\linewidth}{
          {\thmname{#1}\nobreakspace\thmnumber{\textup{#2}}}
          \thmnote{\normalfont\sffamily\color{main-text}\nobreakspace(#3)}
        }}\hbox{\strut}\vspace{0pt}}}
      }
  }
  {
    \newtheoremstyle{basic}
      {0pt}{0pt}{\normalfont}{0pt}
      {}{\;}{0.25em}
      {{\thmname{#1}\nobreakspace\thmnumber{\textup{#2}}}
      \thmnote{\normalfont\sffamily\color{main-text}\nobreakspace(#3)}}
  }

\theoremstyle{basic}

% The style of remark-like environments
\newtheoremstyle{emphasis}
    {0pt}{0pt}{\itshape}{0pt}{}{}{0pt}
    {\thmnote{\normalfont\sffamily\color{main-text}#3\hspace*{0.5em}}}

% Custom proof style
\renewenvironment{proof}[1][\proofname]{\par
  \pushQED{\qed}
  \normalfont \topsep6\p@\@plus6\p@\relax
  \trivlist
  \item[\hskip\labelsep
        \itshape \sffamily \colorist_bfseries:
    #1\hspace{.4em}
    \textcolor{main-text!27!paper}{$|$}]\ignorespaces
}{%
  \popQED\endtrivlist\@endpefalse
}

\bool_if:NTF \l__colorist_fast_bool
  {
    \RequirePackage { hyperref }
    \hypersetup { draft }
  }
  {
    \RequirePackage { hyperref }
    \RequirePackage { bookmark }
    \hypersetup{ hidelinks, linktoc = all }
    \bookmarksetup{ numbered }
    \renewcommand\Hy@numberline[1]{#1.~}
    % https://tex.stackexchange.com/a/1821
    % Add the bookmark of ToC
    \bool_if:NTF \l__colorist_is_book_bool
      {
        \hook_gput_code:nnn { cmd/tableofcontents/before } { colorist }
          {
            \if@openright\cleardoublepage\else\clearpage\fi
            \pdfbookmark[0]{\contentsname}{toc}
          }
      }
      {
        \hook_gput_code:nnn { cmd/tableofcontents/before } { colorist }
          {
            \pdfbookmark[1]{\contentsname}{toc}
          }
      }
  }

% \PassOptionsToPackage { simplename } { projlib-theorem }

\PassOptionsToPackage
  {
    theorem style = {
      , remark = emphasis
      , observation = emphasis
    }
  }
  { projlib-theorem }

\RequirePackage { projlib-theorem }

\tl_gset:Nn \g_crthm_combined_name_sep_tl { \textcolor{main-text}{-} }

\SetTheorem { proof, proof* } { qed-symbol = { \customqedsymbol } }

\SetTheorem { theorem, lemma, proposition, corollary, property, axiom, construction, theorem-with-name }
  {
    name style = {
      heading style = { \color{orange}\colorist_bfseries:\g_colorist_title_font_common_tl\textsc }
    }
  }

\SetTheorem { definition, assumption, convention, hypothesis, notation }
  {
    name style = {
      heading style = { \color{forestgreen}\colorist_bfseries:\g_colorist_title_font_common_tl\textsc }
    }
  }

\SetTheorem { application, claim, example, exercise, fact, problem, question, recall }
  {
    name style = {
      heading style = { \color{main-text}\colorist_bfseries:\g_colorist_title_font_common_tl\textsc }
    }
  }

\SetTheorem { conjecture }
  {
    name style = {
      heading style = { \color{purple}\colorist_bfseries:\g_colorist_title_font_common_tl\textsc }
    }
  }


\bool_if:NF \l__projlib_theorem_complexname_bool
  {
    \SetTheorem { theorem, lemma, proposition, corollary, property, axiom, construction, theorem-with-name }
      {
        name style = {
          , crefname style = { \color{orange}\colorist_bfseries:\g_colorist_title_font_common_tl\textsc }
          , Crefname style = { \color{orange}\colorist_bfseries:\g_colorist_title_font_common_tl\textsc }
          , numbering style = { \color{orange}\colorist_bfseries:\g_colorist_title_font_common_tl }
        }
      }

    \SetTheorem { definition, assumption, convention, hypothesis, notation }
      {
        name style = {
          , crefname style = { \color{forestgreen}\colorist_bfseries:\g_colorist_title_font_common_tl\textsc }
          , Crefname style = { \color{forestgreen}\colorist_bfseries:\g_colorist_title_font_common_tl\textsc }
          , numbering style = { \color{forestgreen}\colorist_bfseries:\g_colorist_title_font_common_tl }
        }
      }

    \SetTheorem { application, claim, example, exercise, fact, problem, question, recall }
      {
        name style = {
          , crefname style = { \color{main-text}\colorist_bfseries:\g_colorist_title_font_common_tl\textsc }
          , Crefname style = { \color{main-text}\colorist_bfseries:\g_colorist_title_font_common_tl\textsc }
          , numbering style = { \color{main-text}\colorist_bfseries:\g_colorist_title_font_common_tl }
        }
      }

    \SetTheorem { conjecture }
      {
        name style = {
          , crefname style = { \color{purple}\colorist_bfseries:\g_colorist_title_font_common_tl\textsc }
          , Crefname style = { \color{purple}\colorist_bfseries:\g_colorist_title_font_common_tl\textsc }
          , numbering style = { \color{purple}\colorist_bfseries:\g_colorist_title_font_common_tl }
        }
      }
  }


\RequirePackage { marginnote }
\RequirePackage { ifoddpage }
\cs_new_protected:Nn \colorist_mpar_adjust:n
  {
    \renewcommand* { \marginnotevadjust } { #1 }
  }
\hook_gput_code:nnn { env/remark/begin } { colorist }
  {
    \if@twoside\checkoddpage
        \ifoddpage\reversemarginpar\fi
    \else
        \reversemarginpar
    \fi
    \colorist_mpar_adjust:n {-.25em}
    \marginnote{
      \colorist_icon_ideabulb:w [0.3] {orange}
      \bool_if:NTF \l__colorist_is_book_bool
        {
          \hspace*{-.2em}
        }
        {
          \hspace*{-.5em}
        }
    }
    \normalmarginpar
  }
\hook_gput_code:nnn { env/conjecture/begin } { colorist }
  {
    \if@twoside\checkoddpage
        \ifoddpage\reversemarginpar\fi
    \else
        \reversemarginpar
    \fi
    \bool_if:NTF \l__colorist_theorem_in_new_line_bool
      {
        \colorist_mpar_adjust:n {-.75em}
        \null
      }
      {
        \colorist_mpar_adjust:n {-.25em}
      }
    \marginnote{
      \colorist_icon_questionmark:w [0.3] {purple}
      \bool_if:NTF \l__colorist_is_book_bool
        {
          \hspace*{-.2em}
        }
        {
          \hspace*{-.5em}
        }
    }
    \normalmarginpar
  }

\ExplSyntaxOff
\RequirePackage{iftex}
\ifXeTeX
\def\pgfsys@hboxsynced#1{%
{%
    \pgfsys@beginscope%
    \setbox\pgf@hbox=\hbox{%
    \hskip\pgf@pt@x%
    \raise\pgf@pt@y\hbox{%
        \pgf@pt@x=0pt%
        \pgf@pt@y=0pt%
        \special{pdf: content q}%
        \pgflowlevelsynccm%
        \pgfsys@invoke{q -1 0 0 -1 0 0 cm}%
        \special{pdf: content -1 0 0 -1 0 0 cm q}
        % translate to original coordinate system
        \pgfsys@invoke{0 J [] 0 d}% reset line cap and dash
        \wd#1=0pt%
        \ht#1=0pt%
        \dp#1=0pt%
        \box#1%
        \pgfsys@invoke{n Q Q Q}%
    }%
    \hss%
    }%
    \wd\pgf@hbox=0pt%
    \ht\pgf@hbox=0pt%
    \dp\pgf@hbox=0pt%
    \pgfsys@hbox\pgf@hbox%
    \pgfsys@endscope%
}}
\fi
\ExplSyntaxOn


\cs_new_protected:Nn \colorist_add_colorbox:nn
  % #1 = list of environments
  % #2 = settings of tcolorbox
  {
    \clist_map_inline:nn { #1 }
      {
        \__colorist_add_colorbox_do:nn { ##1 } { #2 }
        \__colorist_add_colorbox_do:nn { ##1* } { #2 }
      }
  }
\cs_new_protected:Nn \__colorist_add_colorbox_do:nn
  % #1 = name of environment
  % #2 = settings of tcolorbox
  {
    \tcolorboxenvironment { #1 } { #2 }
  }

\colorist_add_colorbox:nn { theorem, lemma, proposition, corollary, property, axiom, construction, definition-corollary, definition-proposition, definition-theorem, theorem-with-name }
  {
    enhanced~jigsaw, breakable, lines~before~break=3,
    left=3.5mm, right=3.5mm,
    colback=main-text!3!paper,
    opacityframe=0.9, colframe=orange, arc=.7mm
  }

  \colorist_add_colorbox:nn { definition, assumption, convention, hypothesis, notation, corollary-definition, proposition-definition, theorem-definition }
  {
    enhanced~jigsaw, breakable, lines~before~break=3,
    left=4mm, right=4mm, top=1mm, bottom=1mm,
    colback=lightorange!10!paper, boxrule=0pt, frame~hidden,
    borderline~west={1.5mm}{0mm}{forestgreen}, arc=.7mm
  }

\colorist_add_colorbox:nn { application, claim, example, fact, recall }
  {
    enhanced~jigsaw, breakable, lines~before~break=3,
    colback=main-text!5!paper,
    boxrule=0pt, frame~hidden, arc=.7mm
  }

\colorist_add_colorbox:nn { conjecture }
  {
    enhanced~jigsaw, breakable, lines~before~break=3,
    left=3.5mm, right=3.5mm,
    colback=main-text!3!paper,
    opacityframe=0.7, colframe=purple, arc=.7mm
  }

\colorist_add_colorbox:nn { problem }
  {
    enhanced~jigsaw, breakable, lines~before~break=3,
    colback=yellow!25!paper,
    boxrule=0pt, frame~hidden, arc=.7mm
  }

\colorist_add_colorbox:nn { question, exercise, remark, observation }
  {
    enhanced~jigsaw, breakable, lines~before~break=3,
    oversize,
    top=0mm, bottom=0mm,
    opacityframe=0, opacityback=0
  }

% Connect adjacent definition-like environments
% From https://tex.stackexchange.com/a/587023
\NewDocumentCommand \AfterEnvEnd { +m }
  { \colorist_after_env_end:nw { #1 } }
\cs_new_protected:Npn \colorist_after_env_end:nw #1 #2
       \if@ignore\@ignorefalse\ignorespaces\fi
  { #2 \if@ignore\@ignorefalse\ignorespaces\fi #1 }
\NewDocumentCommand \ScanEnv { s m +m +m }
  {
    \IfBooleanTF { #1 }
      { \colorist_scan_env_ignore_par:nTF }
      { \colorist_scan_env:nTF }
      { #2 } { #3 } { #4 }
  }
\cs_new_protected:Npn \colorist_scan_env:nTF
  { \__colorist_scan_env:NnTF \c_false_bool }
\cs_new_protected:Npn \colorist_scan_env_ignore_par:nTF
  { \__colorist_scan_env:NnTF \c_true_bool }
\tl_new:N \l__colorist_collected_tl
\cs_new_protected:Npn \__colorist_scan_env:NnTF #1 #2 #3 #4
  {
    \tl_clear:N \l__colorist_collected_tl
    \peek_analysis_map_inline:n
      {
        \tl_put_right:Nn \l__colorist_collected_tl { ##1 }
        \int_compare:nNnTF { "##3 } = { 0 }
          {
            \exp_args:No \token_if_eq_meaning:NNTF { ##1 } \begin
              { \peek_analysis_map_break:n { \__colorist_chk_env:nTFn { #2 } { #3 } { #4 } } }
              {
                \bool_lazy_and:nnF { #1 }
                    { \exp_args:No \token_if_eq_meaning_p:NN { ##1 } \par }
                  { \__colorist_scan_env_end:n { #4 } }
              }
          }
          { \int_compare:nNnF { "##3 } = { 10 } { \__colorist_scan_env_end:n { #4 } } }
      }
  }
\cs_new_protected:Npn \__colorist_scan_env_end:n #1
  { \peek_analysis_map_break:n { \__colorist_reinsert_tokens:nn { #1 } { } } }
\cs_new_protected:Npn \__colorist_reinsert_tokens:nn #1 #2
  {
    \use:x
      {
        \tl_clear:N \exp_not:N \l__colorist_collected_tl
        \exp_not:n { #1 } \l__colorist_collected_tl #2
      }
  }
\cs_new_protected:Npn \__colorist_chk_env:nTFn #1 #2 #3 #4
  {
    \exp_args:Nx \__colorist_reinsert_tokens:nn
      { \str_if_eq:nnTF { #1 } { #4 } { \exp_not:n { #2 } } { \exp_not:n { #3 } } } { { #4 } }
  }

\cs_new_protected:Nn \colorist_add_scan_env:n
  {
    \clist_map_inline:nn { #1 }
      {
        \__colorist_add_scan_env_do:n { ##1 }
      }
  }
\cs_new_protected:Nn \__colorist_add_scan_env_do:n
  {
    \tl_const:cn { l__colorist_scan_env_ #1 _tl }
      {
        \AfterEnvEnd
          {
            \ScanEnv* { #1 }
              { \vspace { -1.15\baselineskip } }
              {
                \ScanEnv* { #1* }
                  { \vspace { -1.15\baselineskip } }
                  {}
              }
          }
      }
    \hook_gput_code:nnn { env/#1/end } { colorist } { \tl_use:c { l__colorist_scan_env_ #1 _tl } }
    \hook_gput_code:nnn { env/#1*/end } { colorist } { \tl_use:c { l__colorist_scan_env_ #1 _tl } }
  }

\colorist_add_scan_env:n { definition, assumption, convention, hypothesis, notation }

\theoremstyle{simple}


\NewDocumentEnvironment { emphasis } { }
  {
    \enlargethispage{2mm}
    \begin{tcolorbox}
        [
          enhanced ~ jigsaw, enforce~breakable, oversize,
          % nobeforeafter,
          left = 1em, right=0mm, top=.5mm, bottom=0mm, boxrule=0pt,
          colback=maintheme!3!paper, frame ~ hidden,
          borderline ~ west = {.15em} {0mm} {maintheme, double, double ~ distance=.12em},
          arc = 0.2mm,
        ]
  }
  {
    \end{tcolorbox}
  }

\hook_gput_code:nnn { env/quote/begin } { colorist } { \small }

%%================================
%%  Title block style
%%================================
\bool_if:NTF \l__colorist_is_book_bool
  {
    \hook_gput_code:nnn { package/projlib-author/after } { colorist }
      {
        \tl_gset:Nn \g__projlib_author_font_author_tl      { \normalfont \colorist_scshape: }
        \tl_gset:Nn \g__projlib_author_font_institute_tl   { \large \normalfont }
        \tl_gset:Nn \g__projlib_author_font_address_tl     { \large \normalfont \itshape }
        \tl_gset:Nn \g__projlib_author_font_curraddr_tl    { \large \normalfont \itshape }
        \tl_gset:Nn \g__projlib_author_font_email_tl       { \large \normalfont \ttfamily }
      }

    \RequirePackage { projlib-titlepage }
    \RenewDocumentCommand \maketitle { O{} }
      {
        \ProjLibTitlePage[#1]
          {
            , title  = \@title
            , author = \@author
            , date   = \@date
          }
      }
  }
  {
%<*colorist-fancy>
    \renewcommand{\@maketitle}
      {
        \begin{center}
            \color{maintheme}
            {\Large\sffamily\colorist_scshape: \colorist_bfseries: \@title}\\\bigskip
            \color{main-text!80!paper}
            {\colorist_scshape:\@author}\par\smallskip
            {\@date}
        \end{center}
        \medskip
      }
%</colorist-fancy>

    \hook_gput_code:nnn { cmd/maketitle/after } { colorist } { \thispagestyle{fancy} }

%%================================
%%  Abstract style
%%================================
%<*colorist-fancy>
    \renewenvironment{abstract}
    {\small{\centerline{\textsc{ \colorist_bfseries: \sffamily\abstractname}}\vspace{-0.3\baselineskip}}
        \color{main-text!80!paper}\begin{quotation}}
    {\end{quotation}\medskip}
%</colorist-fancy>

%%================================
%%  Keyword environment
%%================================
    \DefineMultilingualText { \keywordname }
      {
        EN = Keywords                               ,
        FR = Mots~clés                              ,
        DE = Schlüsselwörter                        ,
        IT = Parole~chiave                          ,
        PT = Palavras~chave                         ,
        BR = Palavras~chave                         ,
        ES = Palabras~clave                         ,
        CN = 关键词                                 ,
        TC = 關鍵詞                                 ,
        JP = キーワード                             ,
        RU = Ключевые~слова                         ,
      }

%<*colorist-fancy>
    \newenvironment{keyword}
    {\small\centerline{{ \colorist_bfseries: \sffamily\keywordname}}\vspace{-0.3\baselineskip}
        \color{main-text!80!paper}\begin{center}}
    {\end{center}\medskip}
%</colorist-fancy>
  } % end of \bool_if:NTF \l__colorist_is_book_bool

%%================================
%%  Simulate features of amsart
%%================================
\PassOptionsToPackage { amsfashion } { projlib-author }
\RequirePackage { projlib-author }

%%================================
%%  Special adjustment
%%================================
%<*colorist-fancy>
\crefformat { chapter } { \nobreak \crefthemark { \crefthe_retrieve_space: } \nobreak #2 \normalfont\textcolor{maintheme} {\larger #1} #3 }
\labelcrefformat { chapter } { #2 \normalfont\textcolor{maintheme} {\larger #1} #3 }
\crefformat { section } { \nobreak \crefthemark { \crefthe_retrieve_space: } \nobreak #2 \color{maintheme} \g_colorist_title_font_section_tl {#1} #3 }
\labelcrefformat { section } { #2 \color{maintheme} \g_colorist_title_font_section_tl {#1} #3 }
\crefformat { subsection } { \nobreak \crefthemark { \crefthe_retrieve_space: } \nobreak #2 \g_colorist_title_font_subsection_tl {#1} #3 }
\labelcrefformat { subsection } { #2 \g_colorist_title_font_subsection_tl {#1} #3 }
\crefformat { subsubsection } { \nobreak \crefthemark { \crefthe_retrieve_space: } \nobreak #2 \color{main-text!70!paper} \g_colorist_title_font_subsubsection_tl {#1} #3 }
\labelcrefformat { subsubsection } { #2 \color{main-text!70!paper} \g_colorist_title_font_subsubsection_tl {#1} #3 }
%</colorist-fancy>
%</style>

\endinput