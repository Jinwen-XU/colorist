%! TEX program = xelatex
\documentclass[allowbf,puretext]{lebhart}

\theoremstyle{basic}
\CreateTheorem{definition}<highest>
\CreateTheorem{theorem}<highest>
\CreateTheorem{conjecture}<highest>
\CreateTheorem*{example*}
\CreateTheorem{problem}<highest>

\theoremstyle{emphasis}
\CreateTheorem{remark}<highest>

\theoremstyle{simple}

%%================================
%% 引入工具集
%%================================
\usepackage{PJLtoolkit}
\usepackage{longtable}  % breakable tables
\usepackage{hologo}     % more TeX logo

\UseLanguage{Chinese}

%%================================
%% 排版代码
%%================================
\usepackage{listings}
\definecolor{maintheme}{RGB}{70,130,180}
\definecolor{forestgreen}{RGB}{21,122,81}
\definecolor{lightergray}{gray}{0.99}
\lstset{language=[LaTeX]TeX,
    keywordstyle=\color{maintheme},
    basicstyle=\ttfamily,
    commentstyle=\color{forestgreen}\ttfamily,
    stringstyle=\rmfamily,
    showstringspaces=false,
    breaklines=true,
    frame=lines,
    backgroundcolor=\color{lightergray},
    flexiblecolumns=true,
    escapeinside={(*}{*)},
    % numbers=left,
    numberstyle=\scriptsize, stepnumber=1, numbersep=5pt,
    firstnumber=last,
} 
\providecommand{\meta}[1]{$\langle${\normalfont\itshape#1}$\rangle$}
\lstset{moretexcs=%
    {href,color,NameTheorem,CreateTheorem,proofideanameCN,cref,dnf,needgraph,UseLanguage,UseOtherLanguage,AddLanguageSetting,maketitle,address,curraddr,email,keywords,subjclass,thanks,dedicatory,PJLdate,ProjLib
    }
}
\lstnewenvironment{code}% 
{\setkeys{lst}{columns=fullflexible,keepspaces=true}}{}

%%================================
%% tip
%%================================
\newenvironment{tip}[1][提示]{%
    \begin{tcolorbox}[breakable,
        enhanced,
        width = \textwidth,
        colback = paper, colbacktitle = paper,
        colframe = gray!50, boxrule=0.2mm,
        coltitle = black,
        fonttitle = \sffamily,
        attach boxed title to top left = {yshift=-\tcboxedtitleheight/2, xshift=.5cm},
        boxed title style = {boxrule=0pt, colframe=paper},
        before skip = 0.3cm,
        after skip = 0.3cm,
        top = 3mm,
        bottom = 3mm,
        title={\sffamily #1}]%
}{\end{tcolorbox}}

%%================================
%% 名称
%%================================
\providecommand{\colorist}{{\normalfont\textsf{colorist}}}
\providecommand{\colorart}{{\normalfont\textsf{colorart}}}
\providecommand{\colorbook}{{\normalfont\textsf{colorbook}}}
\providecommand{\lebhart}{{\normalfont\textsf{lebhart}}}
\providecommand{\beaulivre}{{\normalfont\textsf{beaulivre}}}

%%================================
%% 正文
%%================================
\begin{document}

\title{{\normalfont\textbf{\textsf{lebhart}}},以多彩的方式排版你的文章}
\author{许锦文}
\thanks{对应版本. \texttt{\lebhart{} 2021/06/07}}
\date{2021年6月,北京}

\maketitle

\begin{abstract}
    \lebhart{} 是 \colorist{} 文档类系列的成员之一,其名称取自于德文的lebhaft (活泼),并取了artikel (文章)的前三个字母组合而成。整个 \colorist{} 系列包含用于排版文章的 \colorart{}、\lebhart{} 以及用于排版书的 \colorbook{}、\beaulivre{}。我设计这一系列的初衷是为了撰写草稿与笔记,使之多彩而不缭乱。

    \lebhart{} 支持英语、法语、德语、意大利语、葡萄牙语、巴西葡萄牙语、西班牙语、简体中文、繁体中文、日文、俄文,并且同一篇文档中这些语言可以很好地协调。由于采用了自定义字体,需要用 \hologo{XeLaTeX} 或 \hologo{LuaLaTeX} 引擎进行编译。
    
    这篇说明文档即是用 \lebhart{} 排版的 (使用了参数 \texttt{allowbf}),你可以把它看作一份简短的说明与演示。
\end{abstract}

\tableofcontents

\bigskip\bigskip
\begin{tip}
    多语言支持、定理类环境、未完成标记等功能是由 \ProjLib{} 工具箱提供的,这里只给出了将其与本文档类搭配使用的要点。如需获取更详细的信息,可以参阅 \ProjLib{} 的说明文档。
\end{tip}

\clearpage
\section{初始化}

\subsection{如何载入}

只需要在第一行写:

\begin{code}
  \documentclass{lebhart}
\end{code}

即可使用 \lebhart{} 文档类。请注意,要使用 \hologo{XeLaTeX} 或 \hologo{LuaLaTeX} 引擎才能编译。

\subsection{选项}

\lebhart{} 文档类有下面几个选项:

\begin{itemize}
    \item \texttt{draft} 或 \texttt{fast}
        \begin{itemize}
            \item 你可以使用选项 \verb|fast| 来启用快速但略微粗糙的样式,主要区别是:
            \begin{itemize}
                \item 使用较为简单的数学字体设置;
                \item 不启用超链接;
                \item 启用 \ProjLib{} 工具箱的快速模式。
            \end{itemize}
        \end{itemize}
    \item \texttt{a4paper} 或 \texttt{b5paper}
        \begin{itemize}
            \item 可选的纸张大小。默认的纸张大小为 8.5in $\times$ 11in。
        \end{itemize}
    \item \texttt{palatino}、\texttt{times}、\texttt{garamond}、\texttt{biolinum}
        \begin{itemize}
            \item 字体选项。顾名思义,会加载相应名称的字体。
        \end{itemize}
    \item \texttt{allowbf}
        \begin{itemize}
            \item 允许加粗。启用这一选项时,题目、各级标题、定理类环境名称会被加粗。
        \end{itemize}
\end{itemize}

\begin{tip}
    \begin{itemize}
        \item 在文章的撰写阶段,建议使用 \verb|fast| 选项以加快编译速度,改善写作时的流畅程度。在最后,可以把 \verb|fast| 标记去除,从而得到正式的版本。使用 \verb|fast| 模式时会有“DRAFT”字样的水印,以提示目前处于草稿阶段。
    \end{itemize}
\end{tip}

\medskip
\section{关于文档类中使用的字体}
\lebhart{} 默认使用 Palatino Linotype 作为英文字体,方正悠宋、悠黑GBK作为中文字体,并部分使用了 Neo Euler 作为数学字体。其中,Neo Euler可以在 \url{https://github.com/khaledhosny/euler-otf} 下载。其他字体不是免费的,需要自行购买使用,可在方正字库网站查询详细信息:\url{https://www.foundertype.com}。

\begin{tip}[字体演示]
    \begin{itemize}\setstretch{1.15}
        \item English main font. \textsf{English sans serif font}. \texttt{English typewriter font}.
        \item 中文主要字体,\textsf{中文无衬线字体}
        \item 数学示例: \( \alpha, \beta, \gamma, \delta, 1,2,3,4, a,b,c,d \), \[\mathrm{li}(x)\coloneqq \int_2^{\infty} \frac{1}{\log t}\,\mathrm{d}t \]
    \end{itemize}
\end{tip}

在没有安装相应的字体时,将采用TeX Live中自带的字体来代替,效果可能会有所折扣。


\section{使用说明}

接下来介绍的许多功能是由 \ProjLib{} 工具箱提供的。这里只介绍了基本使用方法,更多细节可以直接参阅其用户文档。

\medskip
\subsection{语言设置}

\lebhart{} 提供了多语言支持,包括英语、法语、德语、意大利语、葡萄牙语、巴西葡萄牙语、西班牙语、简体中文、繁体中文、日文、俄文。可以通过下列命令来选定语言:
\begin{itemize}
    \item \lstinline|\UseLanguage{|\meta{language name}\lstinline|}|,用于指定语言,在其后将使用对应的语言设定。
    \begin{itemize}
        \item 既可以用于导言部分,也可以用于正文部分。在不指定语言时,默认选定 “English”。
    \end{itemize}
    \item \lstinline|\UseOtherLanguage{|\meta{language name}\lstinline|}{|\meta{content}\lstinline|}|,用指定的语言的设定排版 \meta{content}。
    \begin{itemize}
        \item 相比较 \lstinline|\UseLanguage|,它不会对行距进行修改,因此中西文字混排时能够保持行距稳定。
    \end{itemize}
\end{itemize}

\meta{language name} 有下列选择 (不区分大小写,如 \texttt{French} 或 \texttt{french} 均可):
\begin{itemize}\setstretch{1.15}
    \item 简体中文:\texttt{CN}、\texttt{Chinese}、\texttt{SChinese} 或 \texttt{SimplifiedChinese}
    \item 繁体中文:\texttt{TC}、\texttt{TChinese} 或 \texttt{TraditionalChinese}
    \item 英文:\texttt{EN} 或 \texttt{English}
    \item 法文:\texttt{FR} 或 \texttt{French}
    \item 德文:\texttt{DE}、\texttt{German} 或 \texttt{ngerman}
    \item 意大利语:\texttt{IT} 或 \texttt{Italian}
    \item 葡萄牙语:\texttt{PT} 或 \texttt{Portuguese}
    \item 巴西葡萄牙语:\texttt{BR} 或 \texttt{Brazilian}
    \item 西班牙语:\texttt{ES} 或 \texttt{Spanish}
    \item 日文:\texttt{JP} 或 \texttt{Japanese}
    \item 俄文:\texttt{RU} 或 \texttt{Russian}
\end{itemize}

另外,还可以通过下面的方式来填加相应语言的设置:
\begin{itemize}
    \item \lstinline|\AddLanguageSetting{|\meta{settings}\lstinline|}|
    \begin{itemize}
        \item 向所有支持的语言增加设置 \meta{settings}。
    \end{itemize}
    \item \lstinline|\AddLanguageSetting(|\meta{language name}\lstinline|){|\meta{settings}\lstinline|}|
    \begin{itemize}
        \item 向指定的语言 \meta{language name} 增加设置 \meta{settings}。
    \end{itemize}
\end{itemize}
例如,\lstinline|\AddLanguageSetting(German){\color{orange}}| 可以让所有德语以橙色显示(当然,还需要再加上 \lstinline|\AddLanguageSetting{\color{black}}| 来修正其他语言的颜色)。

\medskip
\subsection{定理类环境及其引用}

定义、定理等环境已经被预定义,可以直接使用。

具体来说,预设的定理类环境包括:
\texttt{assumption}、\texttt{axiom}、\texttt{conjecture}、\texttt{convention}、\texttt{corollary}、\texttt{definition}、\texttt{definition-proposition}、\texttt{definition-theorem}、\texttt{example}、\texttt{exercise}、\texttt{fact}、\texttt{hypothesis}、\texttt{lemma}、\texttt{notation}、\texttt{observation}、\texttt{problem}、\texttt{property}、\texttt{proposition}、\texttt{question}、\texttt{remark}、\texttt{theorem},以及相应的带有星号 \lstinline|*| 的无编号版本。

\medskip
在引用定理类环境时,建议使用智能引用 \lstinline|\cref{|\meta{label}\lstinline|}|。这样就不必每次都写上相应环境的名称了。

\begin{tip}[例子]
\begin{code}
  \begin{definition}[奇异物品] \label{def: strange} ...
\end{code}
将会生成
\begin{definition}[奇异物品]\label{def: strange}
    这是奇异物品的定义。
\end{definition}

\lstinline|\cref{def: strange}| 会显示为:\cref{def: strange}。

\medskip
使用 \lstinline|\UseLanguage{English}| 后,定理会显示为:

\UseLanguage{English}
\begin{theorem}[Useless]\label{thm}
    A theorem in English.
\end{theorem}

默认情况下,引用时,定理的名称总是与定理所在区域的语言匹配,例如,上面的定义在现在的英文模式下依然显示为中文:\cref{def: strange} 和 \cref{thm}。如果在引用时想让定理的名称与当前语境相匹配,可以在全局选项中加入 \texttt{regionalref}。
\end{tip}

\bigskip
下面是定理类环境的几种主要样式:
\begin{theorem}
    Theorem style: theorem, proposition, lemma, corollary, ...
\end{theorem}

\begin{proof}
    Proof style
\end{proof}

\begin{remark}
    Remark style
\end{remark}

\begin{conjecture}
    Conjecture style
\end{conjecture}

\begin{example*}
    Example style: example, fact, ...
\end{example*}

\begin{problem}
    Problem style: problem, question, ...
\end{problem}

\medskip
为了美观,相邻的定义环境会自动连在一起:
\begin{definition}
    First definition.
\end{definition}

\begin{definition}
    Second definition.
\end{definition}

\UseLanguage{Chinese}


\subsection{定义新的定理型环境}

若需要定义新的定理类环境,首先要定义这个环境在所用语言下的名称:
% \vspace{-.15\baselineskip}%
\begin{itemize}
    \item \lstinline|\NameTheorem[|\meta{language name}\lstinline|]{|\meta{name of environment}\lstinline|}{|\meta{name string}\lstinline|}|
\end{itemize}
% \vspace{-.15\baselineskip}%
其中,\meta{language name} 可参阅关于语言设置的小节。当不指定 \meta{language name}时,则会将该名称设置为所有支持语言下的名称。另外,带星号与不带星号的同名环境共用一个名称,因此 \lstinline|\NameTheorem{envname*}{...}| 与 \lstinline|\NameTheorem{envname}{...}| 效果相同。

\medskip
然后用下面五种方式之一定义这一环境:
\begin{itemize}
    \item \lstinline|\CreateTheorem*{|\meta{name of environment}\lstinline|}|
        \begin{itemize}
            \item 定义不编号的环境 \meta{name of environment}
        \end{itemize}
    \item \lstinline|\CreateTheorem{|\meta{name of environment}\lstinline|}|
        \begin{itemize}
            \item 定义编号环境 \meta{name of environment},按顺序编号
        \end{itemize}
    \item \lstinline|\CreateTheorem{|\meta{name of environment}\lstinline|}[|\meta{numbered like}\lstinline|]|
        \begin{itemize}
            \item 定义编号环境 \meta{name of environment},与 \meta{numbered like} 计数器共用编号
        \end{itemize}
    \item \lstinline|\CreateTheorem{|\meta{name of environment}\lstinline|}<|\meta{numbered within}\lstinline|>|
        \begin{itemize}
            \item 定义编号环境 \meta{name of environment},在 \meta{numbered within} 计数器内编号
        \end{itemize}
    \item \lstinline|\CreateTheorem{|\meta{name of environment}\lstinline|}(|\meta{existed environment}\lstinline|)|\\
    \lstinline|\CreateTheorem*{|\meta{name of environment}\lstinline|}(|\meta{existed environment}\lstinline|)|
        \begin{itemize}
            \item 将 \meta{name of environment} 与 \meta{existed environment} 或 \meta{existed environment}\lstinline|*| 等同。
            \item 这种方式通常在两种情况下比较有用:
                \begin{enumerate}
                    \item 希望定义更简洁的名称。例如,使用 \lstinline|\CreateTheorem{thm}(theorem)|,便可以直接用名称 \texttt{thm} 来撰写定理。
                    \item 希望去除某些环境的编号。例如,使用 \lstinline|\CreateTheorem{remark}(remark*)|,便可以去除 \texttt{remark} 环境的编号。
                \end{enumerate}
        \end{itemize}
\end{itemize}

\begin{tip}
    其内部使用了 \textsf{amsthm},因此传统的 \texttt{theoremstyle} 对其也是适用的,只需在相关定义前标明即可。
\end{tip}

% \def\proofideanameCN{思路}
\NameTheorem[CN]{proofidea}{思路}
\CreateTheorem*{proofidea*}
\CreateTheorem{proofidea}<subsection>

\bigskip
下面提供一个例子。这三行代码:

\begin{code}
  \NameTheorem[CN]{proofidea}{思路}
  \CreateTheorem*{proofidea*}
  \CreateTheorem{proofidea}<subsection>
\end{code}

可以分别定义不编号的环境 \lstinline|proofidea*| 和编号的环境 \lstinline|proofidea| (在 subsection 内编号),它们支持在简体中文语境中使用,效果如下所示:

\begin{proofidea*}
    \lstinline|proofidea*| 环境。
\end{proofidea*}

\begin{proofidea}
    \lstinline|proofidea| 环境。
\end{proofidea}

\subsection{未完成标记}

你可以通过 \lstinline|\dnf| 来标记尚未完成的部分。例如:
\begin{itemize}
    \item \lstinline|\dnf| 或 \lstinline|\dnf<...>|。效果为:\dnf~或 \dnf<...>。\\其提示文字与当前语言相对应,例如,在法语模式下将会显示为 \UseOtherLanguage{French}{\dnf}。
\end{itemize}

类似的,还有 \lstinline|\needgraph| :
\begin{itemize}
    \item \lstinline|\needgraph| 或 \lstinline|\needgraph<...>|。效果为:\needgraph~或 \needgraph<...>其提示文字与当前语言相对应,例如,在法语模式下将会显示为 \UseOtherLanguage{French}{\needgraph}
\end{itemize}


\subsection{关于文章标题、摘要与关键词}

% 由于引入了 \ProjLib{} 工具箱的 \textsf{PJLamssim} 组件,
\lebhart{} 同时具有标准文档类与\AmS{} 文档类的一些特性。

因此,文章的标题部分既可以按照标准文档类 \textsf{article} 的写法来写:

\begin{code}
  \title{(*\meta{title}*)}
  \author{(*\meta{author}*)\thanks{(*\meta{text}*)}}
  \date{(*\meta{date}*)}
  \maketitle
  \begin{abstract}
      (*\meta{abstract}*)
  \end{abstract}
  \begin{keyword}
      (*\meta{keywords}*)
  \end{keyword}
\end{code}

也可以按照 \AmS{} 文档类的方式来写:

\begin{code}
  \title{(*\meta{title}*)}
  \author{(*\meta{author}*)}
  \thanks{(*\meta{text}*)}
  \address{(*\meta{address}*)}
  \email{(*\meta{email}*)}
  \date{(*\meta{date}*)}
  \keywords{(*\meta{keywords}*)}
  \subjclass{(*\meta{subjclass}*)}
  \begin{abstract}
      (*\meta{abstract}*)
  \end{abstract}
  \maketitle
\end{code}

作者信息可以包含多组,输入方式为:

\begin{code}
  \author{(*\meta{author 1}*)}
  \address{(*\meta{address 1}*)}
  \email{(*\meta{email 1}*)}
  \author{(*\meta{author 2}*)}
  \address{(*\meta{address 2}*)}
  \email{(*\meta{email 2}*)}
  ...
\end{code}

其中 \lstinline|\address|、\lstinline|\curraddr|、\lstinline|\email| 的相互顺序是不重要的。


\medskip
\section{目前存在的问题}

\begin{itemize}[itemsep=.6em]
    \item 目录、part 的设计依然有待改良。
    \item 对于字体的设置仍然不够完善。
    \item 由于很多核心功能建立在 \ProjLib{} 工具箱的基础上,因此 \lebhart{} 自然继承了其所有问题。详情可以参阅 \ProjLib{} 用户文档的“目前存在的问题”这一小节。
    \item 错误处理功能不完善,在出现一些问题时没有相应的错误提示。
    \item 代码中仍有许多可优化之处。
\end{itemize}


\UseLanguage{English}

\clearpage
\section{文档示例}

\subsection{标准文档类写法}

如果想采用标准文档类中的写法,可以参考下面的例子:

\begin{code}
\documentclass{lebhart}
\usepackage{PJLtoolkit} % Load ProjLib toolkit

\UseLanguage{French} % Use French from here

\begin{document}

\title{Le Titre}
\author{Auteur}
\date{\PJLdate{2022-04-01}}

\maketitle

\begin{abstract}
    Ceci est un résumé. \dnf<Plus de contenu est nécessaire.>
\end{abstract}
\begin{keyword}
    AAA, BBB, CCC, DDD, EEE
\end{keyword}

\section{Un théorème}

%% Theorem-like environments can be used directly
\begin{theorem}\label{thm:abc}
    Ceci est un théorème.
\end{theorem}

Référence du théorème: \cref{thm:abc} 
    % It is recommended to use clever reference

\end{document}
\end{code}

如果以后想切换到标准文档类,只需要将前两行换为:

\begin{code}
\documentclass{article}
\usepackage[a4paper,margin=1in]{geometry}
\usepackage[hidelinks]{hyperref}
\usepackage[palatino]{PJLtoolkit} % Load ProjLib toolkit
\end{code}


\clearpage
\subsection{\texorpdfstring{\AmS{}}{AMS} 文档类写法}

如果日后有意切换到期刊模版,想采用 \AmS{} 文档类中的写法,可以参考下面的例子:

\begin{code}
\documentclass{lebhart}
\usepackage{PJLtoolkit} % Load ProjLib toolkit

\UseLanguage{French} % Use French from here

\begin{document}

\title{Le Titre}
\author{Auteur 1}
\address{Adresse 1}
\email{\href{Courriel 1}{Courriel 1}}
\author{Auteur 1}
\address{Adresse 1}
\email{\href{Courriel 2}{Courriel 2}}
\date{\PJLdate{2022-04-01}}
\subjclass{*****}
\keywords{...}

\begin{abstract}
    Ceci est un résumé. \dnf<Plus de contenu est nécessaire.>
\end{abstract}

\maketitle

\section{Première section}

%% Theorem-like environments can be used directly
\begin{theorem}\label{thm:abc}
    Ceci est un théorème.
\end{theorem}

Référence du théorème: \cref{thm:abc} 
    % It is recommended to use clever reference

\end{document}
\end{code}

这样,若想切换到 \AmS{} 文档类,只需要将前两行换为:

\begin{code}
\documentclass{amsart}
\usepackage[a4paper,margin=1in]{geometry}
\usepackage[hidelinks]{hyperref}
\usepackage[palatino]{PJLtoolkit} % Load ProjLib toolkit
\end{code}

\end{document}
