% \iffalse meta-comment
%
% Copyright (C) 2021 by Jinwen XU 
% -------------------------------
% 
% This file may be distributed and/or modified under the conditions of the LaTeX
% Project Public License, either version 1.3c of this license or (at your option)
% any later version. The latest version of this license is in:
%
%    http://www.latex-project.org/lppl.txt
%
% \fi
%
%<*driver>
\ProvidesFile{colorist.dtx}
%</driver>
\NeedsTeXFormat{LaTeX2e}[2020-10-01]
%
%<*colorart>
\ProvidesClass{colorart}
    [2021/05/06 A colorful article style]
\def\colorclass@baseclass{article}
%</colorart>
%
%<*colorbook>
\ProvidesClass{colorbook}
    [2021/05/06 A colorful book style]
\def\colorclass@baseclass{book}
%</colorbook>
%
%<*lebhart>
\ProvidesClass{lebhart}
    [2021/05/06 A colorful article style]
\def\colorclass@baseclass{article}
%</lebhart>
%
%<*beaulivre>
\ProvidesClass{beaulivre}
    [2021/05/06 A colorful book style]
\def\colorclass@baseclass{book}
%</beaulivre>
%
%<*colorist>
\ProvidesPackage{colorist}
    [2021/05/06 A colorful style for articles and books]
%</colorist>
%
\RequirePackage{kvoptions}
\RequirePackage{etoolbox}
%
%<*class>
\SetupKeyvalOptions{
    family = @colorclass,
    prefix = @colorclass@,
}
\DeclareBoolOption[false]{draft}
\DeclareBoolOption[false]{fast}
\DeclareDefaultOption{\PassOptionsToClass{\CurrentOption}{\colorclass@baseclass}}
\ProcessKeyvalOptions*\relax
\LoadClass{\colorclass@baseclass}
\if@colorclass@draft
    \@colorclass@fasttrue
\fi

%%================================
%% Page layout
%%================================
\RequirePackage{silence}
\WarningFilter{geometry}{Over-specification in}
\RequirePackage[heightrounded]{geometry}
\geometry{
    % papersize={8in,11in},
    a4paper,
    total={47em,70em},
    hmarginratio=1:1,
    vmarginratio=1:1,
    footnotesep=2em plus 2pt minus 2pt,
}

\RequirePackage{indentfirst}

\if@colorclass@fast
    \PassOptionsToPackage{fast}{colorist}
\fi
%<*lebhart|beaulivre>
\if@colorclass@fast
    \PassOptionsToPackage{polyglossia}{colorist}
\fi
%</lebhart|beaulivre>
\RequirePackage{colorist}

%%================================
%% Fonts
%%================================
%<*colorart|colorbook>
\RequirePackage{iftex}
\ifPDFTeX
\RequirePackage[T1]{fontenc}
\RequirePackage{inputenc}
\fi
\RequirePackage{mathpazo}
\RequirePackage{newpxtext}
%</colorart|colorbook>
%
%<*lebhart|beaulivre>
%% Math fonts in fast mode
\if@colorclass@fast
    \RequirePackage{mathpazo}
\fi

%% English fonts
\PassOptionsToPackage{no-math}{fontspec}
\RequirePackage{fontspec}
\IfFontExistsTF{Palatino Linotype}{%
    \setmainfont{Palatino Linotype}
}{
    \setmainfont{texgyrepagella-regular.otf}[
        BoldFont       = texgyrepagella-bold.otf ,
        ItalicFont     = texgyrepagella-italic.otf ,
        BoldItalicFont = texgyrepagella-bolditalic.otf ]
}
    \setsansfont{SourceSansPro-Regular.otf}[
        Scale          = MatchLowercase,
        BoldFont       = SourceSansPro-Bold.otf ,
        ItalicFont     = SourceSansPro-RegularIt.otf ,
        BoldItalicFont = SourceSansPro-BoldIt.otf ]
    \setmonofont{cmuntt.otf}[
        Scale          = MatchUppercase,
        BoldFont       = cmuntb.otf ,
        ItalicFont     = cmunst.otf ,
        BoldItalicFont = cmuntb.otf ]

%% Chinese fonts
\PassOptionsToPackage{fontset=none,scheme=plain}{ctex}
\RequirePackage{ctex}
\IfFontExistsTF{FZYOUSK_507R--GBK1-0}{%
    \setCJKmainfont{FZYOUSK_507R--GBK1-0}[
        BoldFont       = FZYOUSK_509R--GBK1-0 ,
        BoldFeatures   = {FakeBold=4} ,
        ItalicFont     = * ,
        BoldItalicFont = FZYOUSK_509R--GBK1-0 ,
        BoldItalicFeatures = {FakeBold=4} ,
        SmallCapsFont  = * ]
}{
    \setCJKmainfont{FandolSong-Regular.otf}[
        BoldFont       = FandolSong-Bold.otf ,
        ItalicFont     = FandolKai-Regular.otf ,
        BoldItalicFont = FandolKai-Regular.otf ,
        BoldItalicFeatures = {FakeBold=4} ,
        SmallCapsFont  = * ]
}
\IfFontExistsTF{FZYOUSK_507R--GBK1-0}{%
    \setCJKmonofont{FZYOUSK_507R--GBK1-0}[
        BoldFont       = FZYOUSK_509R--GBK1-0 ,
        BoldFeatures   = {FakeBold=4} ,
        ItalicFont     = * ,
        BoldItalicFont = FZYOUSK_509R--GBK1-0 ,
        BoldItalicFeatures = {FakeBold=4} ,
        SmallCapsFont  = * ]
}{
    \setCJKmonofont{FandolFang-Regular.otf}[
        BoldFont       = * ,
        BoldFeatures   = {FakeBold=4} ,
        ItalicFont     = * ,
        BoldItalicFont = * ,
        BoldItalicFeatures = {FakeBold=4} ,
        SmallCapsFont  = * ]
}
\IfFontExistsTF{FZYOUHK_506L--GBK1-0}{%
    \setCJKsansfont{FZYOUHK_506L--GBK1-0}[
        BoldFont       = FZYOUHK_509R--GBK1-0 ,
        BoldFeatures   = {FakeBold=4} ,
        ItalicFont     = * ,
        BoldItalicFont = FZYOUHK_509R--GBK1-0 ,
        SmallCapsFont  = * ]
}{
    \setCJKsansfont{FandolHei-Regular.otf}[
        BoldFont       = FandolHei-Bold.otf ,
        ItalicFont     = * ,
        BoldItalicFont = FandolHei-Bold.otf ,
        SmallCapsFont  = * ]
}

%% Math font
\if@colorclass@fast
\RequirePackage{amssymb}
\else
\PassOptionsToPackage
    {warnings-off={mathtools-colon,mathtools-overbracket}}{unicode-math}
\RequirePackage{unicode-math}
\unimathsetup{math-style=ISO, partial=upright, nabla=upright}
\setmathfont{Asana-Math.otf}
\IfFontExistsTF{Neo Euler}{%
% From https://tex.stackexchange.com/a/425887
\setmathfont{Neo Euler}
    [range={"0000-"0001,"0020-"007E,
            "00A0,"00A7-"00A8,"00AC,"00AF,"00B1,"00B4-"00B5,"00B7,
            "00D7,"00F7,
            "0131,
            "0237,"02C6-"02C7,"02D8-"02DA,"02DC,
            "0300-"030C,"030F,"0311,"0323-"0325,"032E-"0332,"0338,
            "0391-"0393,"0395-"03A1,"03A3-"03A8,"03B1-"03BB,
            "03BD-"03C1,"03C3-"03C9,"03D1,"03D5-"03D6,"03F5,
            "2016,"2018-"2019,"2021,"2026-"202C,"2032-"2037,"2044,
            "2057,"20D6-"20D7,"20DB-"20DD,"20E1,"20EE-"20EF,
            "210B-"210C,"210E-"2113,"2118,"211B-"211C,"2126-"2128,
            "212C-"212D,"2130-"2131,"2133,"2135,"2190-"2199,
            "21A4,"21A6,"21A9-"21AA,"21BC-"21CC,"21D0-"21D5,
            "2200,"2202-"2209,"220B-"220C,"220F-"2213,"2215-"221E,
            "2223,"2225,"2227-"222E,"2234-"2235,"2237-"223D,
            "2240-"224C,"2260-"2269,"226E-"2279,"2282-"228B,"228E,
            "2291-"2292,"2295-"2299,"22A2-"22A5,"22C0-"22C5,
            "22DC-"22DD,"22EF,"22F0-"22F1,
            "2308-"230B,"2320-"2321,"2329-"232A,"239B-"23AE,
            "23DC-"23DF,
            "27E8-"27E9,"27F5-"27FE,"2A0C,"2B1A,
            "1D400-"1D433,"1D49C,"1D49E-"1D49F,"1D4A2,"1D4A5-"1D4A6,
            "1D4A9-"1D4AC,"1D4AE-"1D4B5,"1D4D0-"1D4E9,"1D504-"1D505,
            "1D507-"1D50A,"1D50D-"1D514,"1D516-"1D51C,"1D51E-"1D537,
            "1D56C-"1D59F,"1D6A8-"1D6B8,"1D6BA-"1D6D2,"1D6D4-"1D6DD,
            "1D6DF,"1D6E1,"1D7CE-"1D7D7 }]
\setmathfont[range=up/{greek,Greek}, script-features={}, sscript-features={}
            ]{Neo Euler}
\setmathfont[range=up/{latin,Latin}, script-features={}, sscript-features={}
            ]{Neo Euler}
\setmathfont[range={bfup/{latin, Latin, greek, Greek}, frak, bffrak, cal},
             script-features={}, sscript-features={}
            ]{Neo Euler}
\setmathfont[range={up/num, bfup/num, it, bfit, scr, bfscr,
                    sfup, sfit, bfsfup, bfsfit, tt}
            ]{Asana-Math.otf}
\setmathfont[range=bfcal, Scale=MatchUppercase, Alternate]{Asana-Math.otf}
}{}
\fi
%</lebhart|beaulivre>

\RequirePackage[verbose=silent]{microtype}

%%================================
%% Graphics
%%================================
\RequirePackage{graphicx}
\graphicspath{{images/}}
\RequirePackage{wrapfig}
\RequirePackage{float}
\RequirePackage{caption}
\captionsetup{font=small}

%%================================
%% Index
%%================================
\RequirePackage{imakeidx}
%</class>
%
%
%<*colorist>
\SetupKeyvalOptions{%
    family = @colorist,
    prefix = @colorist@
}
\DeclareBoolOption[false]{draft}
\DeclareBoolOption[false]{fast}
\ProcessKeyvalOptions*\relax

\if@colorist@draft
  \@colorist@fasttrue
\fi

\newif\ifIsBook
\ifdefined\chapter\IsBooktrue\else\IsBookfalse\fi

%%================================
%% Title fonts
%%================================
\RequirePackage{anyfontsize}
\newcommand{\partfont}{\sffamily}
\newcommand{\chapfont}{\sffamily}
\newcommand{\secfont}{\sffamily}
\newcommand{\subsecfont}{}
\newcommand{\subsubsecfont}{}

%%================================
%% Color
%%================================
\RequirePackage{xcolor}
\definecolor{paper}{RGB}{255,255,255}
\definecolor{skyblue}{RGB}{60,120,234}
\definecolor{steelblue}{RGB}{70,130,180}
\definecolor{forestgreen}{RGB}{21,122,81}
\definecolor{lightorange}{RGB}{255,185,88}
\definecolor{lightskyblue}{RGB}{35,198,255}

%%================================
%% Footer
%%================================
\RequirePackage{geometry}
\RequirePackage{fancyhdr}
\RequirePackage{extramarks}
\AtEndPreamble{\fancyhfoffset{0pt}}
\fancypagestyle{fancy}{
    \fancyhf{}
    \if@twoside
        \fancyfoot[RO]{\small\textcolor{black!30!paper}{\lastrightmark}%
            ~~\rlap{\textcolor{gray!55!paper}{$|$}~~\thepage}}
        \fancyfoot[LE]{\small\leavevmode\llap{\thepage%
            ~~\textcolor{gray!55!paper}{$|$}}%
            ~~\textcolor{black!30!paper}{\lastleftmark}}
    \else
        \fancyfoot[R]{\small\textcolor{black!30!paper}{\lastrightmark}%
            ~~\rlap{\textcolor{gray!55!paper}{$|$}~~\thepage}}
    \fi
    \renewcommand{\headrulewidth}{0pt}
}
\pagestyle{fancy}
\fancypagestyle{plain}{
    \fancyhf{}
    \if@twoside
        \fancyfoot[RO]{\small%
            ~\rlap{\textcolor{gray!55!paper}{$|$}~~\thepage}}
        \fancyfoot[LE]{\small\leavevmode\llap{\thepage%
            ~~\textcolor{gray!55!paper}{$|$}}}
    \else
        \fancyfoot[R]{\small%
            ~\rlap{\textcolor{gray!55!paper}{$|$}~~\thepage}}
    \fi
    \renewcommand{\headrulewidth}{0pt}
}
\ifbool{IsBook}{
% For book
    \if@twoside
        \renewcommand{\chaptermark}[1]{\markboth{\textsc{#1}}{}}
    \else
        \renewcommand{\chaptermark}[1]{\markboth{\textsc{#1}}{\textsc{#1}}}
    \fi
    \renewcommand*{\sectionmark}[1]{%
        \markright{\thesection~~#1}}
}{
% For article
    \if@twoside
        \renewcommand*{\sectionmark}[1]{\markboth{\textsc{#1}}{}}
    \else
        \renewcommand*{\sectionmark}[1]{\markboth{\textsc{#1}}{\textsc{#1}}}
    \fi
}
%
%%================================
%% Line spacing
%%================================
\RequirePackage{setspace}
\setstretch{1.07}
% To avoid `Underfull \vbox (badness 10000)`
\raggedbottom

%%================================
%% Title format
%%================================
\RequirePackage[explicit,newparttoc]{titlesec}
\PassOptionsToPackage{normalem}{ulem}
\RequirePackage{ulem}
\RequirePackage{PLlang}

\newcommand{\partstring}{\MakeUppercase{{\partname~\protect\thepart}}}
\gappto{\PLlang@langconfig@common}{%
\renewcommand{\partstring}{\MakeUppercase{{\partname~\protect\thepart}}}%
}
\gappto{\PLlang@langconfig@chinese}{%
\renewcommand{\partstring}{第~\thepart~部分}%
}
\gappto{\PLlang@langconfig@tchinese}{%
\renewcommand{\partstring}{第~\thepart~部分}%
}
\gappto{\PLlang@langconfig@japanese}{%
\renewcommand{\partstring}{第~\thepart~部}%
}

\ifbool{IsBook}{
% For book
    %% Part
    \titleclass{\part}{top} % make part like a chapter
    \titleformat{\part}[display]
        {\partfont\filleft}
        {\partstring}
        {1em}
        {\fontsize{20}{0}\selectfont\MakeUppercase{#1}}
    \titleformat{name=\part,numberless}[display]
        {% \phantomsection\addcontentsline{toc}{part}{#1}%
        \partfont\filleft}
        {\phantom{\MakeUppercase{\partname}}}
        {1em}
        {\fontsize{20}{0}\selectfont\MakeUppercase{#1}}
    \titlespacing*{\part}{0pt}{5em}{6em}
    %% Text after part
    \newcommand{\parttext}[1]{%
        \vfill%
        \begin{flushright}%
            \begin{minipage}{0.833\textwidth}%
                \color{black!80!paper}\raggedleft#1%
            \end{minipage}%
        \end{flushright}%
        \vfill\vfill%
        \cleardoublepage%
    }

    %% Chapter
    \titleformat{\chapter}
        {\thispagestyle{fancy}%
        \color{black!80!paper}\chapfont\fontsize{16}{0}\selectfont}{}{0em}
        {\rlap{\hspace*{-.5em}{\color{gray!25!paper}%
            \fontsize{80}{0}\selectfont\raisebox{-7pt}{\thechapter}}}#1}
    \titleformat{name=\chapter,numberless}
        {\thispagestyle{fancy}% \phantomsection\addcontentsline{toc}{chapter}{#1}%
        \color{black!80!paper}\chapfont\fontsize{16}{0}\selectfont}{}{0em}
        {\rlap{\hspace*{-.5em}{\color{gray!25!paper}%
            \fontsize{80}{0}\selectfont\normalfont\raisebox{-7pt}{*}}}#1}

    %% Section
    \titleformat{\section}
    {\secfont\large\color{steelblue}}
    {\thesection}{.75em}{#1}
    [{\titlerule[.75pt]}]
}{
% For article
    %% Part
    \titleformat{\part}[display]
        {%
        \partfont\filleft}
        {\partstring}
        {.3em}
        {\fontsize{16}{0}\selectfont\MakeUppercase{#1}}
    \titleformat{name=\part,numberless}[display]
        {% \phantomsection\addcontentsline{toc}{part}{#1}%
        \partfont\filleft}
        {\phantom{\MakeUppercase{\partname}}}
        {.3em}
        {\fontsize{16}{0}\selectfont\MakeUppercase{#1}}
    %% Text after part
    \newcommand{\parttext}[1]{%
        \begin{flushright}%
            \begin{minipage}{0.833\textwidth}%
                \color{black!80!paper}\raggedleft#1%
            \end{minipage}%
        \end{flushright}%
    }

    %% Section
    \titleformat{\section}
    {\secfont\large\color{steelblue}}
    {\thesection}{.75em}{\scshape #1}
    [{\titlerule[.75pt]}]
}

%% Subsection
\titleformat{\subsection}
    {\subsecfont}{}{0em}
    {\textsf{\thesubsection}~~\textcolor{gray!55!paper}{$|$}~~#1}
\titleformat{name=\subsection,numberless}
    {\subsecfont}{}{0em}
    {#1}

%% Subsubsection
\titleformat{\subsubsection}
    {\subsubsecfont}{\sffamily\thesubsubsection}{.5em}
    {#1}
\titlespacing{\subsubsection}{0pt}{.8\baselineskip}{.5\baselineskip}

%%================================
%% TOC format
%%================================
\RequirePackage{titletoc}
\titlecontents{part}
    [0em]
    {\addvspace{1.5pc}\filcenter\partfont}
    {\thecontentslabel\\\uppercase}
    {}
    {} % without page number
    [\addvspace{.5pc}]
\ifbool{IsBook}{
% For book
    \titlecontents{chapter}
        [2em] % i.e., 0em (part) + 2em
        {\addvspace{.5pc}\chapfont}
        {\contentslabel{2em}}
        {\hspace*{-2em}}
        {\normalfont\titlerule*[1em]{\textcolor{gray!30!paper}{.}}\contentspage}
    \titlecontents{section}
        [4em] % i.e., 2em (chapter) + 2em
        {\secfont}
        {\contentslabel{1.75em}}
        {\hspace*{-1.75em}}
        {\titlerule*[1em]{\textcolor{gray!30!paper}{.}}\contentspage}
    \titlecontents{subsection}
        [7em] % i.e., 4em (section) + 3em
        {\subsecfont}
        {\contentslabel{2.75em}}
        {\hspace*{-2.75em}}
        {\titlerule*[1em]{\textcolor{gray!30!paper}{.}}\contentspage}
}{
% For article
    \titlecontents{section}
        [2em] % i.e., 0em (part) + 2em
        {\secfont}
        {\contentslabel{1.75em}}
        {\hspace*{-1.75em}}
        {\titlerule*[1em]{\textcolor{gray!30!paper}{.}}\contentspage}
    \titlecontents{subsection}
        [5em] % i.e., 2em (section) + 3em
        {\subsecfont}
        {\contentslabel{2.75em}}
        {\hspace*{-2.75em}}
        {\titlerule*[1em]{\textcolor{gray!30!paper}{.}}\contentspage}
}

%%================================
%% Lists
%%================================
\RequirePackage{enumitem}
\setlist{noitemsep,leftmargin=2em}
\renewcommand\labelitemi{\color{gray!50}$\bullet$} 

%%================================
%% Blank page
%%================================
\newcommand{\blinkpagetext}{This page is intentionally left blank}
\renewcommand{\cleardoublepage}{\relax
    \clearpage
    \if@twoside\ifodd\c@page\relax\else
    \thispagestyle{empty}
    \AddToHookNext{shipout/background}
      {% 
       \put(0.5\paperwidth,-0.5\paperheight){%
       \makebox[0pt]{\large\color{gray!20!paper}\blinkpagetext}}}
    \null\newpage\fi\fi}

%%================================
%% Draft mark
%%================================
\RequirePackage{PLdraft}

%%================================
%% Icons
%%================================
\RequirePackage{tikz}
\newcommand{\ideabulb}[2][0.15]{%
    \scalebox{#1}{%
    \begin{tikzpicture}
        \filldraw[draw=#2,fill=#2] (0,0) circle [radius=1cm];
        \filldraw[draw=paper,fill=paper,rounded corners=0.8pt]
            [rotate=20] (-0.26,-0.66) rectangle (-0.06,-0.6)
            [xshift=-0.4mm,yshift=1mm] (-0.26,-0.66) rectangle (0.02,-0.6)
            [xshift=0.4mm,yshift=1mm] (-0.26,-0.66) rectangle (-0.06,-0.6);
        \draw[draw=paper,line width=0.7mm] (-0.18,-0.46)
            .. controls (-0.18,-0.28) and (-0.28,-0.12) ..(-0.4,0.1)
            .. controls (-0.55,0.4) and (-0.3,0.64) ..(0,0.64)
            .. controls (0.3,0.64) and (0.55,0.4) ..(0.4,0.1)
            .. controls (0.28,-0.12) and (0.18,-0.28) ..(0.18,-0.46);
    \end{tikzpicture}}}

\newcommand{\questionmark}[2][0.15]{%
    \scalebox{#1}{%
    \begin{tikzpicture}
        \filldraw[draw=#2,fill=#2] (0,0) circle [radius=1cm];
        \filldraw[paper,yshift=0.5mm,scale=0.9] (-0.4,0.1) circle [radius=0.77mm];
        \draw[draw=paper,line width=1.5mm,yshift=0.5mm,scale=0.9] (-0.4,0.1)
            .. controls (-0.55,0.4) and (-0.3,0.64) ..(0,0.64)
            .. controls (0.3,0.64) and (0.55,0.4) ..(0.4,0.1)
            .. controls (0.28,-0.12) and (0.05,-0.28) ..(0.05,-0.3)
            .. controls (0,-0.36) and (0.0,-0.45) ..(0.0,-0.5);
        \fill[fill=paper,rounded corners=0.6mm]
            (-0.09,-0.75) rectangle (0.09,-0.53);
    \end{tikzpicture}}}

%%================================
%% Theorems
%%================================
\RequirePackage{mathtools}
\RequirePackage{amsthm}
\newtheoremstyle{simple}%
    {}{}%
    {\normalfont}{}%
    {\normalfont}{}%
    {0pt}%
    {\thmname{\textsc{#1}}\thmnumber{ #2}\hspace{.4em}%
        \textcolor{gray!55!paper}{$|$}\hspace{.4em}%
        \color{gray}\thmnote{\ensuremath{(\text{#3})}~~}\pushQED{\qed}}
\def\@endtheorem{\popQED\endtrivlist\@endpefalse }

\renewcommand{\qedsymbol}{%
    \makebox[1em]{\color{gray!55!paper}\rule[-0.1em]{.95em}{.95em}}}

\newtheoremstyle{basic}
    {0pt}{0pt}{\normalfont}{0pt}
    {}{\;}{0.25em}
    {\thmname{#1}~\thmnumber{\textup{#2}}
    \thmnote{\normalfont\sffamily\color{black}~(#3)}}

\newtheoremstyle{emphasis}
    {0pt}{0pt}{\itshape}{0pt}{}{}{0pt}
    {\thmnote{\normalfont\sffamily\color{black}#3\hspace*{0.25em}}}

\if@colorist@fast
    \providecommand{\phantomsection}{}
    \RequirePackage{url}
\else
    \PassOptionsToPackage{hidelinks,linktoc=all}{hyperref}
% To solve `Difference between bookmark levels is greater than one`
    \RequirePackage{bookmark}
    \RequirePackage{hyperref}
\fi

% Should be placed after "hyperref"
\RequirePackage[nothms]{PLthm}

%% Redefine English theorems names
\def\theoremnameEN{\sffamily\color{orange}\textsc{Theorem}}
\def\lemmanameEN{\sffamily\color{orange}\textsc{Lemma}}
\def\propositionnameEN{\sffamily\color{orange}\textsc{Proposition}}
\def\corollarynameEN{\sffamily\color{orange}\textsc{Corollary}}
\def\factnameEN{\sffamily\color{black}\textsc{Fact}}
\def\conjecturenameEN{\sffamily\color{purple}\textsc{Conjecture}}
\def\definitionnameEN{\sffamily\color{forestgreen}\textsc{Definition}}
\def\axiomnameEN{\sffamily\color{orange}\textsc{Axiom}}
\def\assumptionnameEN{\sffamily\color{forestgreen}\textsc{Assumption}}
\def\examplenameEN{\sffamily\color{black}\textsc{Example}}
\def\problemnameEN{\sffamily\color{black}\textsc{Problem}}
\def\remarknameEN{\sffamily\color{black}\textsc{Remark}}

%% Redefine French theorems names
\def\theoremnameFR{\sffamily\color{orange}\textsc{Théorème}}
\def\lemmanameFR{\sffamily\color{orange}\textsc{Lemme}}
\def\propositionnameFR{\sffamily\color{orange}\textsc{Proposition}}
\def\corollarynameFR{\sffamily\color{orange}\textsc{Corollaire}}
\def\factnameFR{\sffamily\color{black}\textsc{Fait}}
\def\conjecturenameFR{\sffamily\color{purple}\textsc{Conjecture}}
\def\definitionnameFR{\sffamily\color{forestgreen}\textsc{Définition}}
\def\axiomnameFR{\sffamily\color{orange}\textsc{Axiome}}
\def\assumptionnameFR{\sffamily\color{forestgreen}\textsc{Supposition}}
\def\examplenameFR{\sffamily\color{black}\textsc{Exemple}}
\def\problemnameFR{\sffamily\color{black}\textsc{Problème}}
\def\remarknameFR{\sffamily\color{black}\textsc{Remarque}}

%% Redefine German theorems names
\def\theoremnameDE{\sffamily\color{orange}\textsc{Satz}}
\def\lemmanameDE{\sffamily\color{orange}\textsc{Lemma}}
\def\propositionnameDE{\sffamily\color{orange}\textsc{Proposition}}
\def\corollarynameDE{\sffamily\color{orange}\textsc{Korollar}}
\def\factnameDE{\sffamily\color{black}\textsc{Fakt}}
\def\conjecturenameDE{\sffamily\color{purple}\textsc{Vermutung}}
\def\definitionnameDE{\sffamily\color{forestgreen}\textsc{Definition}}
\def\axiomnameDE{\sffamily\color{orange}\textsc{Axiom}}
\def\assumptionnameDE{\sffamily\color{forestgreen}\textsc{Annahme}}
\def\examplenameDE{\sffamily\color{black}\textsc{Beispiel}}
\def\problemnameDE{\sffamily\color{black}\textsc{Problem}}
\def\remarknameDE{\sffamily\color{black}\textsc{Bemerkung}}

%% Redefine Chinese theorems names
\def\theoremnameCN{\sffamily\color{orange}定理}
\def\lemmanameCN{\sffamily\color{orange}引理}
\def\propositionnameCN{\sffamily\color{orange}命题}
\def\corollarynameCN{\sffamily\color{orange}推论}
\def\factnameCN{\sffamily\color{black}事实}
\def\conjecturenameCN{\sffamily\color{purple}猜想}
\def\definitionnameCN{\sffamily\color{forestgreen}定义}
\def\axiomnameCN{\sffamily\color{orange}公理}
\def\assumptionnameCN{\sffamily\color{forestgreen}假设}
\def\examplenameCN{\sffamily\color{black}例}
\def\problemnameCN{\sffamily\color{black}问题}
\def\remarknameCN{\sffamily\color{black}备注}

\def\theoremnameTC{\sffamily\color{orange}定理}
\def\lemmanameTC{\sffamily\color{orange}引理}
\def\propositionnameTC{\sffamily\color{orange}命題}
\def\corollarynameTC{\sffamily\color{orange}推論}
\def\factnameTC{\sffamily\color{black}事實}
\def\conjecturenameTC{\sffamily\color{purple}猜想}
\def\definitionnameTC{\sffamily\color{forestgreen}定義}
\def\axiomnameTC{\sffamily\color{orange}公理}
\def\assumptionnameTC{\sffamily\color{forestgreen}假設}
\def\examplenameTC{\sffamily\color{black}例}
\def\problemnameTC{\sffamily\color{black}問題}
\def\remarknameTC{\sffamily\color{black}備註}

%% Redefine Japanese theorems names
\def\theoremnameJP{\sffamily\color{orange}定理}
\def\lemmanameJP{\sffamily\color{orange}補題}
\def\propositionnameJP{\sffamily\color{orange}命題}
\def\corollarynameJP{\sffamily\color{orange}推論}
\def\factnameJP{\sffamily\color{black}事実}
\def\conjecturenameJP{\sffamily\color{purple}予想}
\def\definitionnameJP{\sffamily\color{forestgreen}定義}
\def\axiomnameJP{\sffamily\color{orange}公理}
\def\assumptionnameJP{\sffamily\color{forestgreen}仮定}
\def\examplenameJP{\sffamily\color{black}例}
\def\problemnameJP{\sffamily\color{black}問題}
\def\remarknameJP{\sffamily\color{black}注釈}

%% Redefine Russian theorems names
\def\theoremnameRU{\sffamily\color{orange}Теорема}
\def\lemmanameRU{\sffamily\color{orange}Лемма}
\def\propositionnameRU{\sffamily\color{orange}Предложение}
\def\corollarynameRU{\sffamily\color{orange}Следствие}
\def\factnameRU{\sffamily\color{black}Факт}
\def\conjecturenameRU{\sffamily\color{purple}Гипотеза}
\def\definitionnameRU{\sffamily\color{forestgreen}Определение}
\def\axiomnameRU{\sffamily\color{orange}Аксиома}
\def\assumptionnameRU{\sffamily\color{forestgreen}Предположение}
\def\examplenameRU{\sffamily\color{black}Пример}
\def\problemnameRU{\sffamily\color{black}Вопрос}
\def\remarknameRU{\sffamily\color{black}Замечание}

%% Theorem environments
\theoremstyle{basic}
\ifbool{IsBook}{
    \newaliascnt{highest}{chapter}
}{
    \newaliascnt{highest}{section}
}
\CreateTheorem{theorem}<highest>
\CreateTheorem{lemma}[theorem]
\CreateTheorem{proposition}[theorem]
\CreateTheorem{corollary}[theorem]
\CreateTheorem{fact}[theorem]
\CreateTheorem{conjecture}<highest>
\CreateTheorem*{theorem*}
\CreateTheorem*{lemma*}
\CreateTheorem*{proposition*}
\CreateTheorem*{corollary*}
\CreateTheorem*{fact*}
\CreateTheorem*{conjecture*}
\CreateTheorem{definition}[theorem]
\CreateTheorem{axiom}[theorem]
\CreateTheorem{assumption}[theorem]
\CreateTheorem{example}<highest>
\CreateTheorem{problem}<highest>
\CreateTheorem*{definition*}
\CreateTheorem*{axiom*}
\CreateTheorem*{assumption*}
\CreateTheorem*{example*}
\CreateTheorem*{problem*}

\theoremstyle{emphasis}
\CreateTheorem{remark}<highest>
\CreateTheorem*{remark*}

\RequirePackage{marginnote}
\newcommand{\mparadjust}[1]{\renewcommand*{\marginnotevadjust}{#1}}
\apptocmd{\remark}{\reversemarginpar\mparadjust{-.25em}\marginnote{\ideabulb[0.3]{orange}\hspace*{-.5em}}\normalmarginpar}{}{\FAIL}
\apptocmd{\conjecture}{\reversemarginpar\mparadjust{-.25em}\marginnote{\questionmark[0.3]{purple}\hspace*{-.5em}}\normalmarginpar}{}{\FAIL}

\RequirePackage{iftex}
\ifXeTeX
\def\pgfsys@hboxsynced#1{%
{%
    \pgfsys@beginscope%
    \setbox\pgf@hbox=\hbox{%
    \hskip\pgf@pt@x%
    \raise\pgf@pt@y\hbox{%
        \pgf@pt@x=0pt%
        \pgf@pt@y=0pt%
        \special{pdf: content q}%
        \pgflowlevelsynccm%
        \pgfsys@invoke{q -1 0 0 -1 0 0 cm}%
        \special{pdf: content -1 0 0 -1 0 0 cm q}
        % translate to original coordinate system
        \pgfsys@invoke{0 J [] 0 d}% reset line cap and dash
        \wd#1=0pt%
        \ht#1=0pt%
        \dp#1=0pt%
        \box#1%
        \pgfsys@invoke{n Q Q Q}%
    }%
    \hss%
    }%
    \wd\pgf@hbox=0pt%
    \ht\pgf@hbox=0pt%
    \dp\pgf@hbox=0pt%
    \pgfsys@hbox\pgf@hbox%
    \pgfsys@endscope%
}}
\fi

\theoremstyle{simple}% as the default style for user-defined environments

\let\proof\relax
\let\endproof\relax
\def\proofnameCN{\proofname}
\def\proofnameEN{\proofname}
\def\proofnameFR{\proofname}
\CreateTheorem*{proof}

\RequirePackage[many]{tcolorbox}
\if@colorclass@fast
    \tcbstartdraftmode
\fi
\tcolorboxenvironment{theorem}
    {enhanced jigsaw,pad at break*=1mm,breakable,colback=black!3!paper,
    left=3.5mm,right=3.5mm,
    opacityframe=0.9,colframe=orange,arc=.7mm}
\tcolorboxenvironment{theorem*}
    {enhanced jigsaw,pad at break*=1mm,breakable,colback=black!3!paper,
    left=3.5mm,right=3.5mm,
    opacityframe=0.9,colframe=orange,arc=.7mm}
\tcolorboxenvironment{lemma}
    {enhanced jigsaw,pad at break*=1mm,breakable,colback=black!3!paper,
    left=3.5mm,right=3.5mm,
    opacityframe=0.9,colframe=orange,arc=.7mm}
\tcolorboxenvironment{lemma*}
    {enhanced jigsaw,pad at break*=1mm,breakable,colback=black!3!paper,
    left=3.5mm,right=3.5mm,
    opacityframe=0.9,colframe=orange,arc=.7mm}
\tcolorboxenvironment{proposition}
    {enhanced jigsaw,pad at break*=1mm,breakable,colback=black!3!paper,
    left=3.5mm,right=3.5mm,
    opacityframe=0.9,colframe=orange,arc=.7mm}
\tcolorboxenvironment{proposition*}
    {enhanced jigsaw,pad at break*=1mm,breakable,colback=black!3!paper,
    left=3.5mm,right=3.5mm,
    opacityframe=0.9,colframe=orange,arc=.7mm}
\tcolorboxenvironment{corollary}
    {enhanced jigsaw,pad at break*=1mm,breakable,colback=black!3!paper,
    left=3.5mm,right=3.5mm,
    opacityframe=0.9,colframe=orange,arc=.7mm}
\tcolorboxenvironment{corollary*}
    {enhanced jigsaw,pad at break*=1mm,breakable,colback=black!3!paper,
    left=3.5mm,right=3.5mm,
    opacityframe=0.9,colframe=orange,arc=.7mm}
\tcolorboxenvironment{axiom}
    {enhanced jigsaw,pad at break*=1mm,breakable,colback=black!3!paper,
    left=3.5mm,right=3.5mm,
    opacityframe=0.9,colframe=orange,arc=.7mm}
\tcolorboxenvironment{axiom*}
    {enhanced jigsaw,pad at break*=1mm,breakable,colback=black!3!paper,
    left=3.5mm,right=3.5mm,
    opacityframe=0.9,colframe=orange,arc=.7mm}

\tcolorboxenvironment{fact}
    {enhanced jigsaw,pad at break*=1mm,breakable,colback=gray!10!paper,
    boxrule=0pt,frame hidden,arc=.7mm}
\tcolorboxenvironment{fact*}
    {enhanced jigsaw,pad at break*=1mm,breakable,colback=gray!10!paper,
    boxrule=0pt,frame hidden,arc=.7mm}

\tcolorboxenvironment{conjecture}
    {enhanced jigsaw,pad at break*=1mm,breakable,colback=black!3!paper,
    left=3.5mm,right=3.5mm,
    opacityframe=0.7,colframe=purple,arc=.7mm}
\tcolorboxenvironment{conjecture*}
    {enhanced jigsaw,pad at break*=1mm,breakable,colback=black!3!paper,
    left=3.5mm,right=3.5mm,
    opacityframe=0.7,colframe=purple,arc=.7mm}

\tcolorboxenvironment{definition}
    {enhanced jigsaw,pad at break*=1mm,breakable,
    left=4mm,right=4mm,top=1mm,bottom=1mm,
    colback=lightorange!10!paper,boxrule=0pt,frame hidden,
    borderline west={1.5mm}{-1mm}{forestgreen},arc=.7mm}
\tcolorboxenvironment{definition*}
    {enhanced jigsaw,pad at break*=1mm,breakable,
    left=4mm,right=4mm,top=1mm,bottom=1mm,
    colback=lightorange!10!paper,boxrule=0pt,frame hidden,
    borderline west={1.5mm}{-1mm}{forestgreen},arc=.7mm}
\tcolorboxenvironment{assumption}
    {enhanced jigsaw,pad at break*=1mm,breakable,
    left=4mm,right=4mm,top=1mm,bottom=1mm,
    colback=lightorange!10!paper,boxrule=0pt,frame hidden,
    borderline west={1.5mm}{-1mm}{forestgreen},arc=.7mm}
\tcolorboxenvironment{assumption*}
    {enhanced jigsaw,pad at break*=1mm,breakable,
    left=4mm,right=4mm,top=1mm,bottom=1mm,
    colback=lightorange!10!paper,boxrule=0pt,frame hidden,
    borderline west={1.5mm}{-1mm}{forestgreen},arc=.7mm}

\tcolorboxenvironment{example}
    {enhanced jigsaw,pad at break*=1mm,breakable,colback=gray!10!paper,
    boxrule=0pt,frame hidden,arc=.7mm}
\tcolorboxenvironment{example*}
    {enhanced jigsaw,pad at break*=1mm,breakable,colback=gray!10!paper,
    boxrule=0pt,frame hidden,arc=.7mm}

\tcolorboxenvironment{problem}
    {enhanced jigsaw,pad at break*=1mm,breakable,colback=yellow!25!paper,
    boxrule=0pt,frame hidden,arc=.7mm}
\tcolorboxenvironment{problem*}
    {enhanced jigsaw,pad at break*=1mm,breakable,colback=yellow!25!paper,
    boxrule=0pt,frame hidden,arc=.7mm}

% Connect definitions
% From https://tex.stackexchange.com/a/587023
\ExplSyntaxOn
\NewDocumentCommand \AfterEnvEnd { +m }
  { \colorist_after_env_end:nw {#1} }
\cs_new_protected:Npn \colorist_after_env_end:nw #1 #2
       \if@ignore\@ignorefalse\ignorespaces\fi
  { #2 \if@ignore\@ignorefalse\ignorespaces\fi #1 }
  \NewDocumentCommand \ScanEnv { s m +m+m }
  {
    \IfBooleanTF {#1}
      { \jinwen_scan_env_ignore_par:nTF }
      { \jinwen_scan_env:nTF }
          {#2} {#3} {#4}
  }
\cs_new_protected:Npn \jinwen_scan_env:nTF
  { \__jinwen_scan_env:NnTF \c_false_bool }
\cs_new_protected:Npn \jinwen_scan_env_ignore_par:nTF
  { \__jinwen_scan_env:NnTF \c_true_bool }
\tl_new:N \l__jinwen_collected_tl
\cs_new_protected:Npn \__jinwen_scan_env:NnTF #1 #2 #3 #4
  {
    \tl_clear:N \l__jinwen_collected_tl
    \peek_analysis_map_inline:n
      {
        \tl_put_right:Nn \l__jinwen_collected_tl {##1}
        \int_compare:nNnTF { "##3 } = { 0 }
          {
            \exp_args:No \token_if_eq_meaning:NNTF {##1} \begin
              { \peek_analysis_map_break:n { \__jinwen_chk_env:nTFn {#2} {#3} {#4} } }
              {
                \bool_lazy_and:nnF {#1}
                    { \exp_args:No \token_if_eq_meaning_p:NN {##1} \par }
                  { \__jinwen_scan_env_end:n {#4} }
              }
          }
          { \int_compare:nNnF { "##3 } = { 10 } { \__jinwen_scan_env_end:n {#4} } }
      }
  }
\cs_new_protected:Npn \__jinwen_scan_env_end:n #1
  { \peek_analysis_map_break:n { \__jinwen_reinsert_tokens:nn {#1} { } } }
\cs_new_protected:Npn \__jinwen_reinsert_tokens:nn #1 #2
  {
    \use:x
      {
        \tl_clear:N \exp_not:N \l__jinwen_collected_tl
        \exp_not:n {#1} \l__jinwen_collected_tl #2
      }
  }
\cs_new_protected:Npn \__jinwen_chk_env:nTFn #1 #2 #3 #4
  {
    \exp_args:Nx \__jinwen_reinsert_tokens:nn
      { \str_if_eq:nnTF {#1} {#4} { \exp_not:n {#2} } { \exp_not:n {#3} } } { {#4} }
  }
\ExplSyntaxOff

\def\scandefinitionenv{%
  \AfterEnvEnd{%
    \ScanEnv*{definition}%
      {\vspace{-1.05\baselineskip}}%
      {\ScanEnv*{definition*}%
        {\vspace{-1.05\baselineskip}}%
        {}}}}
\AddToHook{env/definition/end}{\scandefinitionenv}%
\AddToHook{env/definition*/end}{\scandefinitionenv}%

\def\scanassumptionenv{%
  \AfterEnvEnd{%
    \ScanEnv*{assumption}%
      {\vspace{-1.05\baselineskip}}%
      {\ScanEnv*{assumption*}%
        {\vspace{-1.05\baselineskip}}%
        {}}}}
\AddToHook{env/assumption/end}{\scanassumptionenv}%
\AddToHook{env/assumption*/end}{\scanassumptionenv}%

\ifbool{IsBook}{}{%

%%================================
%% Title block style
%%================================
\renewcommand{\@maketitle}{%
    \noindent%
    {\textcolor{gray!55!paper}{\rule{\textwidth}{0.75pt}}}%
    \vspace{-\parskip}%
    \begin{center}%
        {\large\@title}\\\medskip%
        \color{black!80!paper}%
        {\scshape\@author}\\\smallskip%
        {\@date}%
    \end{center}%
    \vspace{-\parskip}%
    \ifx\@date\@empty%
        \vspace{-\baselineskip}%
    \else%
        \vspace{-.8\baselineskip}%
    \fi%
    {\textcolor{gray!55!paper}{\rule{\textwidth}{0.75pt}}\par}%
    \medskip%
}
\apptocmd{\maketitle}{\thispagestyle{fancy}}{}{\FAIL}

%%================================
%% Abstract style
%%================================
\renewenvironment{abstract}
{\small{\centerline{\textsc{\abstractname}}\vspace{-0.3\baselineskip}}
    \color{black!80!paper}\begin{quotation}}
{\end{quotation}\medskip}

}
%</colorist>

\endinput