%<*driver>
\ProvidesFile{colorist.dtx}
%</driver>
\NeedsTeXFormat{LaTeX2e}[2020-10-01]
%<*lebhart>
\ProvidesClass{lebhart}
    [2021/03/10 A colorful article style]
%</lebhart>
%<*lebhartfast>
\ProvidesClass{lebhartfast}
    [2021/03/10 A faster but rougher version of lebhart]
%</lebhartfast>
%<*beaulivre>
\ProvidesClass{beaulivre}
    [2021/03/10 A colorful book style]
%</beaulivre>
%<*beaulivrefast>
\ProvidesClass{beaulivrefast}
    [2021/03/10 A faster but rougher version of beaulivre]
%</beaulivrefast>

%<*lebhart|lebhartfast>
\DeclareOption*{%
    \PassOptionsToClass{\CurrentOption}{article}}
\ProcessOptions\relax
\LoadClass{article}
%</lebhart|lebhartfast>
%<*beaulivre|beaulivrefast>
\DeclareOption*{%
    \PassOptionsToClass{\CurrentOption}{book}}
\ProcessOptions\relax
\LoadClass{book}
%</beaulivre|beaulivrefast>

\RequirePackage{etoolbox}

%%================================
%% Fonts
%%================================
\RequirePackage{anyfontsize}

%% Title fonts
\newcommand{\partfont}{\sffamily}
%<beaulivre|beaulivrefast>\newcommand{\chapfont}{\sffamily}
\newcommand{\secfont}{\sffamily}
\newcommand{\subsecfont}{\sffamily}
%<*lebhartfast|beaulivrefast>

%% Math fonts
\RequirePackage{mathpazo}
%</lebhartfast|beaulivrefast>

%% English fonts
\PassOptionsToPackage{no-math}{fontspec}
\RequirePackage{fontspec}
\IfFontExistsTF{Palatino Linotype}{%
    \setmainfont{Palatino Linotype}
}{
    \setmainfont{texgyrepagella-regular.otf}[
        BoldFont       = texgyrepagella-bold.otf ,
        ItalicFont     = texgyrepagella-italic.otf ,
        BoldItalicFont = texgyrepagella-bolditalic.otf ]
}
    \setsansfont{SourceSansPro-Regular.otf}[
        Scale          = MatchLowercase,
        BoldFont       = SourceSansPro-Bold.otf ,
        ItalicFont     = SourceSansPro-RegularIt.otf ,
        BoldItalicFont = SourceSansPro-BoldIt.otf ]

%% Chinese fonts
\PassOptionsToPackage{fontset=none,scheme=plain}{ctex}
\RequirePackage{ctex}
\IfFontExistsTF{FZYouSongS 507R}{%
    \setCJKmainfont{FZYouSongS 507R}[
        BoldFont       = FZYouSongS 509R ,
        BoldFeatures   = {FakeBold=2} ,
        ItalicFont     = * ,
        BoldItalicFont = FZYouSongS 509R ,
        BoldItalicFeatures = {FakeBold=2} ,
        SmallCapsFont  = * ]
}{
    \setCJKmainfont{FandolSong-Regular.otf}[
        BoldFont       = FandolSong-Bold.otf ,
        ItalicFont     = FandolKai-Regular.otf ,
        BoldItalicFont = FandolKai-Regular.otf ,
        BoldItalicFeatures = {FakeBold=4} ,
        SmallCapsFont  = * ]
}
\IfFontExistsTF{FZFSJW--GB1-0}{%
    \setCJKmonofont{FZFSJW--GB1-0}[
        BoldFont       = * ,
        BoldFeatures   = {FakeBold=4} ,
        ItalicFont     = * ,
        BoldItalicFont = * ,
        BoldItalicFeatures = {FakeBold=4} ,
        SmallCapsFont  = * ]
}{
    \setCJKmonofont{FandolFang-Regular.otf}[
        BoldFont       = * ,
        BoldFeatures   = {FakeBold=4} ,
        ItalicFont     = * ,
        BoldItalicFont = * ,
        BoldItalicFeatures = {FakeBold=4} ,
        SmallCapsFont  = * ]
}
\IfFontExistsTF{FZYouHeiS 506L}{%
    \setCJKsansfont{FZYouHeiS 506L}[
        BoldFont       = FZYouHeiS 509R,
        ItalicFont     = * ,
        BoldItalicFont = FZYouHeiS 509R ,
        SmallCapsFont  = * ]
}{
    \setCJKsansfont{FandolHei-Regular.otf}[
        BoldFont       = FandolHei-Bold.otf ,
        ItalicFont     = * ,
        BoldItalicFont = FandolHei-Bold.otf ,
        SmallCapsFont  = * ]
}
%<*lebhart|beaulivre>

%% Math fonts
\PassOptionsToPackage
    {warnings-off={mathtools-colon,mathtools-overbracket}}{unicode-math}
\RequirePackage{unicode-math}
\unimathsetup{math-style=ISO}
\setmathfont{Asana-Math.otf}
\IfFontExistsTF{Neo Euler}{%
\setmathfont{Neo Euler} % From https://tex.stackexchange.com/a/425887
    [range={"0000-"0001,"0020-"007E,
            "00A0,"00A7-"00A8,"00AC,"00AF,"00B1,"00B4-"00B5,"00B7,
            "00D7,"00F7,
            "0131,
            "0237,"02C6-"02C7,"02D8-"02DA,"02DC,
            "0300-"030C,"030F,"0311,"0323-"0325,"032E-"0332,"0338,
            "0391-"0393,"0395-"03A1,"03A3-"03A8,"03B1-"03BB,
            "03BD-"03C1,"03C3-"03C9,"03D1,"03D5-"03D6,"03F5,
            "2016,"2018-"2019,"2021,"2026-"202C,"2032-"2037,"2044,
            "2057,"20D6-"20D7,"20DB-"20DD,"20E1,"20EE-"20EF,
            "210B-"210C,"210E-"2113,"2118,"211B-"211C,"2126-"2128,
            "212C-"212D,"2130-"2131,"2133,"2135,"2190-"2199,
            "21A4,"21A6,"21A9-"21AA,"21BC-"21CC,"21D0-"21D5,
            "2200,"2202-"2209,"220B-"220C,"220F-"2213,"2215-"221E,
            "2223,"2225,"2227-"222E,"2234-"2235,"2237-"223D,
            "2240-"224C,"2260-"2269,"226E-"2279,"2282-"228B,"228E,
            "2291-"2292,"2295-"2299,"22A2-"22A5,"22C0-"22C5,
            "22DC-"22DD,"22EF,"22F0-"22F1,
            "2308-"230B,"2320-"2321,"2329-"232A,"239B-"23AE,
            "23DC-"23DF,
            "27E8-"27E9,"27F5-"27FE,"2A0C,"2B1A,
            "1D400-"1D433,"1D49C,"1D49E-"1D49F,"1D4A2,"1D4A5-"1D4A6,
            "1D4A9-"1D4AC,"1D4AE-"1D4B5,"1D4D0-"1D4E9,"1D504-"1D505,
            "1D507-"1D50A,"1D50D-"1D514,"1D516-"1D51C,"1D51E-"1D537,
            "1D56C-"1D59F,"1D6A8-"1D6B8,"1D6BA-"1D6D2,"1D6D4-"1D6DD,
            "1D6DF,"1D6E1,"1D7CE-"1D7D7 }]
}{}
%</lebhart|beaulivre>

\RequirePackage{mathtools}
\RequirePackage[verbose=silent]{microtype}

%%================================
%% Page layout
%%================================
\RequirePackage[heightrounded]{geometry}
\geometry{
    papersize={7in,10in},
    total={40em,60em},
    hmarginratio=1:1,
    vmarginratio=1:1,
    footnotesep=2em plus 2pt minus 2pt,
}

\RequirePackage{indentfirst}

\RequirePackage{xcolor}
\definecolor{paper}{RGB}{255,255,255}
\definecolor{skyblue}{RGB}{60,120,234}
\definecolor{steelblue}{RGB}{70,130,180}
\definecolor{forestgreen}{RGB}{21,122,81}
\definecolor{lightorange}{RGB}{255,185,88}
\definecolor{lightskyblue}{RGB}{35,198,255}

\RequirePackage{fancyhdr}
\RequirePackage{extramarks}
\fancypagestyle{fancy}{
    \fancyhf{}
    \if@twoside
        \fancyfoot[RO]{\small\textcolor{black!30!paper}{\lastrightmark}%
            ~~\rlap{\textcolor{gray!55!paper}{$|$}~~\thepage}}
        \fancyfoot[LE]{\small\leavevmode\llap{\thepage%
            ~~\textcolor{gray!55!paper}{$|$}}%
            ~~\textcolor{black!30!paper}{\lastleftmark}}
    \else
        \fancyfoot[R]{\small\textcolor{black!30!paper}{\lastrightmark}%
            ~~\rlap{\textcolor{gray!55!paper}{$|$}~~\thepage}}
    \fi
    \renewcommand{\headrulewidth}{0pt}
}
\pagestyle{fancy}

\fancypagestyle{plain}{
    \fancyhf{}
    \if@twoside
        \fancyfoot[RO]{\small%
            ~\rlap{\textcolor{gray!55!paper}{$|$}~~\thepage}}
        \fancyfoot[LE]{\small\leavevmode\llap{\thepage%
            ~~\textcolor{gray!55!paper}{$|$}}}
    \else
        \fancyfoot[R]{\small%
            ~\rlap{\textcolor{gray!55!paper}{$|$}~~\thepage}}
    \fi
    \renewcommand{\headrulewidth}{0pt}
}

%<*lebhart|lebhartfast>
\if@twoside
    \renewcommand*{\sectionmark}[1]{\markboth{\uppercase{#1}}{}}
\else
    \renewcommand*{\sectionmark}[1]{\markboth{\uppercase{#1}}{\uppercase{#1}}}
\fi
%</lebhart|lebhartfast>
%<*beaulivre|beaulivrefast>
\if@twoside
    \renewcommand{\chaptermark}[1]{\markboth{\uppercase{#1}}{}}
\else
    \renewcommand{\chaptermark}[1]{\markboth{\uppercase{#1}}{\uppercase{#1}}}
\fi
\renewcommand*{\sectionmark}[1]{%
    \markright{\raisebox{.03em}{\footnotesize/}%
    ~\thesection~\raisebox{.03em}{\footnotesize/}~~~#1}}
%</beaulivre|beaulivrefast>

%%================================
%% Line spacing
%%================================
\RequirePackage{setspace}
\linespread{1.07}
% To avoid `Underfull \vbox (badness 10000)`
\raggedbottom

%%================================
%% Title format
%%================================
\RequirePackage[explicit,newparttoc]{titlesec}
\PassOptionsToPackage{normalem}{ulem}
\RequirePackage{ulem}

%<*lebhart|lebhartfast>
%% Part
\titleformat{\part}[display]
    {%
    \partfont\filleft}
    {\MakeUppercase{\partname~\protect\thepart}}
    {.3em}
    {\fontsize{16}{0}\selectfont\MakeUppercase{#1}}
\titleformat{name=\part,numberless}[display]
    {% \phantomsection\addcontentsline{toc}{part}{#1}%
    \partfont\filleft}
    {\phantom{\MakeUppercase{\partname}}}
    {.3em}
    {\fontsize{16}{0}\selectfont\MakeUppercase{#1}}
%% Text after part
\newcommand{\parttext}[1]{%
    \begin{flushright}%
        \begin{minipage}{0.833\textwidth}%
            \color{black!80!paper}\raggedleft#1%
        \end{minipage}%
    \end{flushright}%
}
%</lebhart|lebhartfast>
%<*beaulivre|beaulivrefast>
%% Part
\titleclass{\part}{top} % make part like a chapter
\titleformat{\part}[display]
    {\partfont\filleft}
    {\MakeUppercase{\partname~\protect\thepart}}
    {1em}
    {\fontsize{20}{0}\selectfont\MakeUppercase{#1}}
\titleformat{name=\part,numberless}[display]
    {% \phantomsection\addcontentsline{toc}{part}{#1}%
    \partfont\filleft}
    {\phantom{\MakeUppercase{\partname}}}
    {1em}
    {\fontsize{20}{0}\selectfont\MakeUppercase{#1}}
\titlespacing*{\part}{0pt}{5em}{6em}
%% Text after part
\newcommand{\parttext}[1]{%
\vfill%
%
\begin{flushright}%
    \begin{minipage}{0.833\textwidth}%
        \color{black!80!paper}\raggedleft#1%
    \end{minipage}%
\end{flushright}%
%
\vfill\vfill%
\cleardoublepage%
}

%% Chapter
\titleformat{\chapter}
    {\thispagestyle{fancy}%
    \color{black!80!paper}\chapfont\fontsize{16}{0}\selectfont}{}{0em}
    {\rlap{\hspace*{-.5em}{\color{gray!25!paper}%
        \fontsize{80}{0}\selectfont\raisebox{-7pt}{\thechapter}}}#1}
\titleformat{name=\chapter,numberless}
    {\thispagestyle{fancy}% \phantomsection\addcontentsline{toc}{chapter}{#1}%
    \color{black!80!paper}\chapfont\fontsize{16}{0}\selectfont}{}{0em}
    {\rlap{\hspace*{-.5em}{\color{gray!25!paper}%
        \fontsize{80}{0}\selectfont\normalfont\raisebox{-7pt}{*}}}#1}
%</beaulivre|beaulivrefast>

%% Section
\titleformat{\section}
    {\secfont\color{steelblue}}
    {\thesection}{1em}{#1}
    [{\titlerule[.75pt]}]

%% Subsection
\titleformat{\subsection}
    {\subsecfont}{}{0em}
    {\thesubsection~~\textcolor{gray!55!paper}{$|$}~~#1}
\titleformat{name=\subsection,numberless}
    {\subsecfont}{}{0em}
    {#1}

%%================================
%% TOC format
%%================================
\RequirePackage{titletoc}
\titlecontents{part}
    [0em]
    {\addvspace{1.5pc}\filcenter\partfont}
    {\thecontentslabel\\\uppercase}
    {}
    {} % without page number
    [\addvspace{.5pc}]
%<*lebhart|lebhartfast>
\titlecontents{section}
    [2em] % i.e., 0em (part) + 2em
    {\secfont}
    {\contentslabel{1.75em}}
    {\hspace*{-1.75em}}
    {\titlerule*[1em]{\textcolor{gray!30!paper}{.}}\contentspage}
\titlecontents{subsection}
    [5em] % i.e., 2em (section) + 3em
    {\subsecfont}
    {\contentslabel{2.75em}}
    {\hspace*{-2.75em}}
    {\titlerule*[1em]{\textcolor{gray!30!paper}{.}}\contentspage}
%</lebhart|lebhartfast>
%<*beaulivre|beaulivrefast>
\titlecontents{chapter}
    [2em] % i.e., 0em (part) + 2em
    {\addvspace{.5pc}\chapfont}
    {\contentslabel{2em}}
    {\hspace*{-2em}}
    {\normalfont\titlerule*[1em]{\textcolor{gray!30!paper}{.}}\contentspage}
\titlecontents{section}
    [4em] % i.e., 2em (chapter) + 2em
    {\secfont}
    {\contentslabel{1.75em}}
    {\hspace*{-1.75em}}
    {\titlerule*[1em]{\textcolor{gray!30!paper}{.}}\contentspage}
\titlecontents{subsection}
    [7em] % i.e., 4em (section) + 3em
    {\subsecfont}
    {\contentslabel{2.75em}}
    {\hspace*{-2.75em}}
    {\titlerule*[1em]{\textcolor{gray!30!paper}{.}}\contentspage}
%</beaulivre|beaulivrefast>

%%================================
%% Graphics
%%================================
\RequirePackage{graphicx}
\graphicspath{{images/}}
\RequirePackage{wrapfig}
\RequirePackage{caption}

%%================================
%% Lists
%%================================
\RequirePackage{enumitem}
\setlist{noitemsep,leftmargin=2em}
\renewcommand\labelitemi{\color{gray!50}$\bullet$} 

%%================================
%% Blank page
%%================================
\newcommand{\blinkpagetext}{This page is intentionally left blank}
\renewcommand{\cleardoublepage}{\relax
    \clearpage
    \if@twoside\ifodd\c@page\relax\else
    \thispagestyle{empty}
    \newgeometry{centering}
    \null\vfill
    \centerline{\large\color{gray!20!paper}\blinkpagetext}
    \vfill\restoregeometry\newpage\fi\fi}

%%================================
%% Index
%%================================
\RequirePackage{imakeidx}
% switch off the line numbers of index
\pretocmd{\printindex}{}{}{\FAIL}
\apptocmd{\printindex}{}{}{\FAIL}

%%================================
%% Icons
%%================================
\RequirePackage{tikz}
\newcommand{\ideabulb}[2][0.15]{%
  \scalebox{#1}{%
  \begin{tikzpicture}
    \filldraw[draw=#2,fill=#2] (0,0) circle [radius=1cm];
    \filldraw[draw=paper,fill=paper,rounded corners=0.8pt]
        [rotate=20] (-0.26,-0.66) rectangle (-0.06,-0.6)
        [xshift=-0.4mm,yshift=1mm] (-0.26,-0.66) rectangle (0.02,-0.6)
        [xshift=0.4mm,yshift=1mm] (-0.26,-0.66) rectangle (-0.06,-0.6);
    \draw[draw=paper,line width=0.7mm] (-0.18,-0.46)
        .. controls (-0.18,-0.28) and (-0.28,-0.12) ..(-0.4,0.1)
        .. controls (-0.55,0.4) and (-0.3,0.64) ..(0,0.64)
        .. controls (0.3,0.64) and (0.55,0.4) ..(0.4,0.1)
        .. controls (0.28,-0.12) and (0.18,-0.28) ..(0.18,-0.46);
  \end{tikzpicture}}}

\newcommand{\questionmark}[2][0.15]{%
  \scalebox{#1}{%
  \begin{tikzpicture}
    \filldraw[draw=#2,fill=#2] (0,0) circle [radius=1cm];
    \filldraw[paper,yshift=0.5mm,scale=0.9] (-0.4,0.1) circle [radius=0.77mm];
    \draw[draw=paper,line width=1.5mm,yshift=0.5mm,scale=0.9] (-0.4,0.1)
        .. controls (-0.55,0.4) and (-0.3,0.64) ..(0,0.64)
        .. controls (0.3,0.64) and (0.55,0.4) ..(0.4,0.1)
        .. controls (0.28,-0.12) and (0.05,-0.28) ..(0.05,-0.3)
        .. controls (0,-0.36) and (0.0,-0.45) ..(0.0,-0.5);
    \fill[fill=paper,rounded corners=0.6mm]
          (-0.09,-0.75) rectangle (0.09,-0.53);
  \end{tikzpicture}}}

%%================================
%% Theorems
%%================================
\RequirePackage{amsthm}
\newtheoremstyle{simple}%
    {}{}%
    {\normalfont}{}%
    {\normalfont}{}%
    {0pt}%
    {\thmname{\textsc{#1}}\thmnumber{ #2}\hspace{.4em}%
        \textcolor{gray!55!paper}{$|$}\hspace{.4em}%
        \color{gray}\thmnote{\ensuremath{(\text{#3})}~~}\pushQED{\qed}}
\def\@endtheorem{\popQED\endtrivlist\@endpefalse }

\renewcommand{\qedsymbol}{%
    \makebox[1em]{\color{gray!55!paper}\rule[-0.1em]{.95em}{.95em}}}

\newtheoremstyle{basic}
    {0pt}{0pt}{\normalfont}{0pt}
    {}{\;}{0.25em}
    {\thmname{#1}~\thmnumber{\textup{#2}}
    \thmnote{\normalfont\sffamily\color{black}~(#3)}}

\newtheoremstyle{emphasis}
    {0pt}{0pt}{\itshape}{0pt}{}{}{0pt}
    {\thmnote{\normalfont\sffamily\color{black}#3\hspace*{0.25em}}}

%<lebhart|beaulivre>\PassOptionsToPackage{hidelinks,linktoc=all}{hyperref}
\RequirePackage{aliascnt}
% To solve `Difference between bookmark levels is greater than one`
%<lebhart|beaulivre>\RequirePackage{bookmark}
%<lebhart|beaulivre>\RequirePackage{hyperref}
\PassOptionsToPackage{nameinlink}{cleveref}
\RequirePackage{cleveref}
\crefdefaultlabelformat{#2#1#3~\aftergroup\ignorespaces}

\newcommand\englishABBR{EN}
\newcommand\frenchABBR{FR}
\newcommand\chineseABBR{CN}

%% Macro for creating theorems
\RequirePackage{xstring}
\newcommand\PassFirstToSecond[2]{#2{#1}}%
\NewDocumentCommand{\CreateTheorem}{sm}{%
    \begingroup
    \protected@edef\temp{#2}%
    \expandafter\IfEndWith\expandafter{\temp}{*}{%
        \expandafter\StrGobbleRight\expandafter{\temp}{1}[\temp]%
        \PassFirstToSecond{*}%
    }{%
        \PassFirstToSecond{}%
    }%
    {\expandafter\PassFirstToSecond%
        \expandafter{\temp}{\endgroup\InnerCreateTheorem{#1}}}%
}%
\NewDocumentCommand{\InnerCreateTheorem}{mmmod<>}{% #1 = star or no star #2 = name of environment #3 = emptiness or star to append to name of environment #4 = numbered like #5 = numbered within
    \IfBooleanTF{#1}{%
        \IfValueTF{#4}
            {\@firstoftwo}
            {\IfValueTF{#5}{\@firstoftwo}{\@secondoftwo}}%
    }{%
        \IfValueTF{#4}
            {\IfValueTF{#5}{\@firstoftwo}{\@secondoftwo}}{
            \@secondoftwo}
    }%
    {%
        \GenericError{}%
        {\string\CreateTheorem\space syntax error\on@line}{%
        You cannot call the starred variant with optional argument,\MessageBreak
        nor call the unstarred variant with several optional arguments.}%
        {}%
    }{%
        \IfBooleanTF{#1}{%
            \newtheorem*{#2EN#3}{\csname#2nameEN\endcsname}
            \newtheorem*{#2FR#3}{\csname#2nameFR\endcsname}
            \newtheorem*{#2CN#3}{\csname#2nameCN\endcsname}
        }{%
            \IfValueTF{#5}{%
                \newcounter{#2#3}[{#5}]%
                \expandafter\renewcommand\expandafter*%
                    \csname the#2#3\expandafter\endcsname%
                    \expandafter{\csname the#5\endcsname.\arabic{#2#3}}%
            }{%
                \IfValueTF{#4}
                    {\newaliascnt{#2#3}{#4}}
                    {\newcounter{#2#3}}%
            }%
            %-------------------------------------------------------------------
            \CreateTheoremNumberedLikeAliasCounter{#2}{EN}{#3}%
            \CreateTheoremNumberedLikeAliasCounter{#2}{FR}{#3}%
            \CreateTheoremNumberedLikeAliasCounter{#2}{CN}{#3}%
            %-------------------------------------------------------------------
        }%
        \NewDocumentEnvironment{#2#3}{}
            {\csname#2\csname\languagename ABBR\endcsname#3\endcsname}%
            {\csname end#2\csname\languagename ABBR\endcsname#3\endcsname}%
    }%
}%
\NewDocumentCommand{\CreateTheoremNumberedLikeAliasCounter}{mmm}{%
    \newaliascnt{#1#2#3}{#1#3}%
    \newtheorem{#1#2#3}[{#1#2#3}]{\csname#1name#2\endcsname}%
    \aliascntresetthe{#1#2#3}%
    \crefname{#1#2#3}%
        {\csname#1name#2\endcsname}%
        {\csname#1name#2\endcsname}%
}%

%% English theorems names
\def\theoremnameEN{\unskip~\sffamily\color{orange}\textsc{Theorem}}
\def\lemmanameEN{\unskip~\sffamily\color{orange}\textsc{Lemma}}
\def\propositionnameEN{\unskip~\sffamily\color{orange}\textsc{Proposition}}
\def\corollarynameEN{\unskip~\sffamily\color{orange}\textsc{Corollary}}
\def\factnameEN{\unskip~\sffamily\color{black}\textsc{Fact}}
\def\conjecturenameEN{\unskip~\sffamily\color{purple}\textsc{Conjecture}}
\def\definitionnameEN{\unskip~\sffamily\color{forestgreen}\textsc{Definition}}
\def\examplenameEN{\unskip~\sffamily\color{black}\textsc{Example}}
\def\problemnameEN{\unskip~\sffamily\color{black}\textsc{Problem}}
\def\remarknameEN{\unskip~\sffamily\color{black}\textsc{Remark}}

%% French theorems names
\def\theoremnameFR{\unskip~\sffamily\color{orange}\textsc{Théorème}}
\def\lemmanameFR{\unskip~\sffamily\color{orange}\textsc{Lemme}}
\def\propositionnameFR{\unskip~\sffamily\color{orange}\textsc{Proposition}}
\def\corollarynameFR{\unskip~\sffamily\color{orange}\textsc{Corollaire}}
\def\factnameFR{\unskip~\sffamily\color{black}\textsc{Fait}}
\def\conjecturenameFR{\unskip~\sffamily\color{purple}\textsc{Conjecture}}
\def\definitionnameFR{\unskip~\sffamily\color{forestgreen}\textsc{Définition}}
\def\examplenameFR{\unskip~\sffamily\color{black}\textsc{Exemple}}
\def\problemnameFR{\unskip~\sffamily\color{black}\textsc{Problème}}
\def\remarknameFR{\unskip~\sffamily\color{black}\textsc{Remarque}}

%% Chinese theorems names
\def\theoremnameCN{\sffamily\color{orange}定理}
\def\lemmanameCN{\sffamily\color{orange}引理}
\def\propositionnameCN{\sffamily\color{orange}命题}
\def\corollarynameCN{\sffamily\color{orange}推论}
\def\factnameCN{\sffamily\color{black}事实}
\def\conjecturenameCN{\sffamily\color{purple}猜想}
\def\definitionnameCN{\sffamily\color{forestgreen}定义}
\def\examplenameCN{\sffamily\color{black}例}
\def\problemnameCN{\sffamily\color{black}问题}
\def\remarknameCN{\sffamily\color{black}备注}

%% Theorem environments
\theoremstyle{basic}
%<lebhart|lebhartfast>\CreateTheorem{theorem}<section>
%<beaulivre|beaulivrefast>\CreateTheorem{theorem}<chapter>
\CreateTheorem{lemma}[theorem]
\CreateTheorem{proposition}[theorem]
\CreateTheorem{corollary}[theorem]
\CreateTheorem{fact}[theorem]
%<lebhart|lebhartfast>\CreateTheorem{conjecture}<section>
%<beaulivre|beaulivrefast>\CreateTheorem{conjecture}<chapter>
\CreateTheorem*{theorem*}
\CreateTheorem*{lemma*}
\CreateTheorem*{proposition*}
\CreateTheorem*{corollary*}
\CreateTheorem*{fact*}
\CreateTheorem*{conjecture*}
\CreateTheorem{definition}[theorem]
%<lebhart|lebhartfast>\CreateTheorem{example}<section>
%<beaulivre|beaulivrefast>\CreateTheorem{example}<chapter>
%<lebhart|lebhartfast>\CreateTheorem{problem}<section>
%<beaulivre|beaulivrefast>\CreateTheorem{problem}<chapter>
\CreateTheorem*{definition*}
\CreateTheorem*{example*}
\CreateTheorem*{problem*}

\theoremstyle{emphasis}
%<lebhart|lebhartfast>\CreateTheorem{remark}<section>
%<beaulivre|beaulivrefast>\CreateTheorem{remark}<chapter>
\CreateTheorem*{remark*}

\RequirePackage{marginnote}
\newcommand{\mparadjust}[1]{\renewcommand*{\marginnotevadjust}{#1}}
\apptocmd{\remark}{\reversemarginpar\mparadjust{-.3em}\marginnote{\ideabulb[0.3]{orange}}\normalmarginpar}{}{\FAIL}
\apptocmd{\conjecture}{\reversemarginpar\mparadjust{-.3em}\marginnote{\questionmark[0.3]{purple}}\normalmarginpar}{}{\FAIL}

\RequirePackage{iftex}
\ifXeTeX
\def\pgfsys@hboxsynced#1{%
{%
    \pgfsys@beginscope%
    \setbox\pgf@hbox=\hbox{%
    \hskip\pgf@pt@x%
    \raise\pgf@pt@y\hbox{%
        \pgf@pt@x=0pt%
        \pgf@pt@y=0pt%
        \special{pdf: content q}%
        \pgflowlevelsynccm%
        \pgfsys@invoke{q -1 0 0 -1 0 0 cm}%
        \special{pdf: content -1 0 0 -1 0 0 cm q}
        % translate to original coordinate system
        \pgfsys@invoke{0 J [] 0 d}% reset line cap and dash
        \wd#1=0pt%
        \ht#1=0pt%
        \dp#1=0pt%
        \box#1%
        \pgfsys@invoke{n Q Q Q}%
    }%
    \hss%
    }%
    \wd\pgf@hbox=0pt%
    \ht\pgf@hbox=0pt%
    \dp\pgf@hbox=0pt%
    \pgfsys@hbox\pgf@hbox%
    \pgfsys@endscope%
}}
\fi

\theoremstyle{simple}% as the default style for user-defined environments

\RequirePackage[many]{tcolorbox}
%<lebhartfast|beaulivrefast>\tcbstartdraftmode
\tcolorboxenvironment{theorem}
    {enhanced jigsaw,pad at break*=1mm,breakable,colback=black!3!paper,
    colframe=orange}
\tcolorboxenvironment{theorem*}
    {enhanced jigsaw,pad at break*=1mm,breakable,colback=black!3!paper,
    colframe=orange}
\tcolorboxenvironment{lemma}
    {enhanced jigsaw,pad at break*=1mm,breakable,colback=black!3!paper,
    colframe=orange}
\tcolorboxenvironment{lemma*}
    {enhanced jigsaw,pad at break*=1mm,breakable,colback=black!3!paper,
    colframe=orange}
\tcolorboxenvironment{proposition}
    {enhanced jigsaw,pad at break*=1mm,breakable,colback=black!3!paper,
    colframe=orange}
\tcolorboxenvironment{proposition*}
    {enhanced jigsaw,pad at break*=1mm,breakable,colback=black!3!paper,
    colframe=orange}
\tcolorboxenvironment{corollary}
    {enhanced jigsaw,pad at break*=1mm,breakable,colback=black!3!paper,
    colframe=orange}
\tcolorboxenvironment{corollary*}
    {enhanced jigsaw,pad at break*=1mm,breakable,colback=black!3!paper,
    colframe=orange}

\tcolorboxenvironment{fact}
    {enhanced jigsaw,pad at break*=1mm,breakable,colback=gray!10!paper,
    boxrule=0pt,frame hidden}
\tcolorboxenvironment{fact*}
    {enhanced jigsaw,pad at break*=1mm,breakable,colback=gray!10!paper,
    boxrule=0pt,frame hidden}

\tcolorboxenvironment{conjecture}
    {enhanced jigsaw,pad at break*=1mm,breakable,colback=black!3!paper,
    colframe=purple}
\tcolorboxenvironment{conjecture*}
    {enhanced jigsaw,pad at break*=1mm,breakable,colback=black!3!paper,
    colframe=purple}

\tcolorboxenvironment{definition}
    {enhanced jigsaw,pad at break*=1mm,breakable,
    colback=lightorange!10!paper,boxrule=0pt,frame hidden,
    borderline west={1.5mm}{-1mm}{forestgreen}}
\tcolorboxenvironment{definition*}
    {enhanced jigsaw,pad at break*=1mm,breakable,
    colback=lightorange!10!paper,boxrule=0pt,frame hidden,
    borderline west={1.5mm}{-1mm}{forestgreen}}
    
\tcolorboxenvironment{example}
    {enhanced jigsaw,pad at break*=1mm,breakable,colback=gray!10!paper,
    boxrule=0pt,frame hidden}
\tcolorboxenvironment{example*}
    {enhanced jigsaw,pad at break*=1mm,breakable,colback=gray!10!paper,
    boxrule=0pt,frame hidden}

\tcolorboxenvironment{problem}
    {enhanced jigsaw,pad at break*=1mm,breakable,colback=yellow!25!paper,
    boxrule=0pt,frame hidden}
\tcolorboxenvironment{problem*}
    {enhanced jigsaw,pad at break*=1mm,breakable,colback=yellow!25!paper,
    boxrule=0pt,frame hidden}

%%================================
%% Language configuration
%%================================
%<*lebhart|beaulivre>
\PassOptionsToPackage{french,english}{babel}
\RequirePackage{babel}
\frenchsetup{PartNameFull=false}
%</lebhart|beaulivre>
%<*lebhartfast|beaulivrefast>
\RequirePackage{polyglossia}
\setdefaultlanguage{english}
\setotherlanguage[frenchpart=false]{french}
%</lebhartfast|beaulivrefast>

\newcommand{\colorist@langconfig@chinese}{%
    \def\abstractname{摘要}%
    \def\proofname{证明}%
    \def\contentsname{目录}%
    \def\listfigurename{插图}%
    \def\listtablename{表格}%
    \def\figurename{图}%
    \def\tablename{表}%
    \def\indexname{索引}%
    \def\appendixname{附录}%
    \def\bibname{参考文献}%
    \renewcommand{\languagename}{chinese}%
}
\newcommand{\colorist@langconfig@english}{%
    \selectlanguage{english}%
}
\newcommand{\colorist@langconfig@french}{%
    \selectlanguage{french}%
% The line below is only needed for 'babel'
%<lebhart|beaulivre>    \def\frenchpartname{Partie}%
}

\newcommand{\UseLanguageCORE}[1]{%
    \ifstrequal{#1}{chinese}{\colorist@langconfig@chinese}{}%
    \ifstrequal{#1}{Chinese}{\colorist@langconfig@chinese}{}%
    \ifstrequal{#1}{english}{\colorist@langconfig@english}{}%
    \ifstrequal{#1}{English}{\colorist@langconfig@english}{}%
    \ifstrequal{#1}{french}{\colorist@langconfig@french}{}
    \ifstrequal{#1}{French}{\colorist@langconfig@french}{}
}
\newcommand{\UseLanguage}[1]{%
    \ifx\@onlypreamble\@notprerr%
        \UseLanguageCORE{#1}%
    \else%
        \AfterEndPreamble{\UseLanguageCORE{#1}}%
    \fi%
}

%%================================
%% Draft mark
%%================================
\def\dnfFont{\ttfamily}
\def\needgraphFont{\ttfamily}

\def\dnfTextEN{To be finished here}
\def\needgraphTextEN{A graph is needed here}
\def\dnfTextFR{À terminer ici}
\def\needgraphTextFR{Il manque encore un graphique ici}
\def\dnfTextCN{这里的内容尚未完成}
\def\needgraphTextCN{这里需要一张图片}

\definecolor{dnfColor}{RGB}{21,122,20}
\definecolor{needgraphColor}{RGB}{70,130,180}

%<*lebhart|beaulivre>
\PassOptionsToPackage{many}{tcolorbox}
\RequirePackage{tcolorbox}
\newtcbox{\plainBox}[1][-paper]{enhanced jigsaw,%
    on line, arc = 1.2pt, outer arc = 1pt,breakable,%
    colframe = #1,colupper=#1,opacityback=0,%
    boxsep = 1pt,boxrule = 1.2pt,%
    left = 1pt, right = 1pt, top = 0pt, bottom = 0pt,%
}
%</lebhart|beaulivre>
%<*lebhartfast|beaulivrefast>
\newcommand{\plainBox}[2][-paper]{\textcolor{#1}{%
    \setlength{\fboxsep}{1.5pt}%
    \setlength{\fboxrule}{1.2pt}%
    \fbox{#2}}}
%</lebhartfast|beaulivrefast>

\NewDocumentCommand{\dnf}{d<>}{%
    \noindent\plainBox[dnfColor]%
    {\normalfont\dnfFont\small%
    \csname dnfText\csname\languagename ABBR\endcsname\endcsname%
    \IfNoValueF{#1}{ : #1}}%
}
\NewDocumentCommand{\needgraph}{d<>}{%
    \par%
    \centerline{\plainBox[needgraphColor]%
    {\normalfont\needgraphFont\small%
    \csname needgraphText\csname\languagename ABBR\endcsname\endcsname%
    \IfNoValueF{#1}{ : #1}}}%
    \par%
}
%<*lebhart|lebhartfast>

%%================================
%% Title block style
%%================================
\renewcommand{\@maketitle}{%
    \noindent%
    {\textcolor{gray!55!paper}{\rule{\textwidth}{0.75pt}}}%
    \vspace{-\parskip}%
    \begin{center}%
        {\large\@title}\\\medskip%
        \color{black!80!paper}%
        {\scshape\@author}\\%
        {\@date}%
    \end{center}%
    \vspace{-\parskip}%
    \vspace{-.5\baselineskip}%
    {\textcolor{gray!55!paper}{\rule{\textwidth}{0.75pt}}\par}%
}
\apptocmd{\maketitle}{\thispagestyle{fancy}}{}{\FAIL}
\apptocmd{\abstract}{\color{black!80!paper}}{}{\FAIL}
%</lebhart|lebhartfast>

\endinput